\vspace{-1ex}
\section{Introduction}
\vspace{-1ex}

Our goal is to analyze the inter-individual variability within a time series dataset, an approach of significant interest in physiological contexts \cite{guscelli2019importance, wang2016research, bar2012studying, germain2023unsupervised}. Specifically, we aim to develop an unsupervised feature representation method that encodes the specificities of individual time series in comparison to a reference time series. In physiology, examining the various "shapes" in a time series related to biological phenomena and their variations due to individual differences or pathological conditions is common. However, the term "shape" lacks a precise definition and is more intuitively understood as the silhouette of a pattern in a time series. In this paper, we refer to the shape of a time series as the graph of this signal.

Although community structures with representatives can be learned in an unsupervised manner \cite{trirat2024universal, meng2023unsupervised} using contrastive loss \cite{franceschi2019unsupervised, tonekaboni2021unsupervised, meng2023unsupervised} or similarity measures \cite{asgari2023clustering, germain2023unsupervised, paparrizos2015k, ye2009time}, the study of inter-individual variability of shapes within a cluster \cite{niennattrakul2007inaccuracies, shirato2023identifying} remains an open problem in unsupervised representation learning (URL), particularly for \textit{irregularly sampled} time series with \textit{variable lengths}.
   
Our work explicitly focuses on learning shape-based representation of time series. First, we propose to view the shape of a time series not merely as its curve $\{s_t:\eqsp t\in\msi\}$, but as its graph $\msg(s)=\{(t,s(t)):\eqsp t\in \msi\}$. Then, building on the shape analysis literature \cite{beg2005computing,vaillant2004statistics}, we adopt the Large Deformation Diffeomorphic Metric Mapping (LDDMM) framework \cite{beg2005computing,vaillant2004statistics} to analyze these graphs. The core idea is to represent each element $\msg(s^j)$ of a dataset $(s^j)_{j\in[N]}$ as the transformation of a reference graph $\msg(\mathbf{s}_0)$ by a diffeomorphism $\phi_j$, i.e. $\msg(s^j) \sim \phi_j . \msg(\mathbf{s}_0)$. The diffeomorphism $\phi_j$ is learned by integrating an ordinary differential equation parameterized by a Reproducing Kernel Hilbert Space (RKHS). The parameters $(\alpha_j)_{j\in[N]}$ encoding the diffemorphisms $(\phi_j)_{j\in[N]}$ yield the representation features of the graphs $(\msg(s^j))_{j\in[N]}$. Finally, these shape-encoding features can be used as inputs to any statistical or machine-learning model.

     
However, a graph time series transformation by a general diffeomorphism is not always a graph time series, see e.g. \Cref{fig:diffeo}, thus a graph time series is more than a simple curve \cite{glaunes2008large}. Our contributions arise from this observation: we specify the class of diffeomorphisms to consider and show how to learn them. This change is fruitful in representing transformations of time series graphs as illustrated in \Cref{fig:transport}.

Our contributions can be summarized as follows:
\begin{itemize}
  \item We propose an unsupervised method (TS-LDDMM) to analyze the inter-individual variability of shapes in a time series dataset (\Cref{section:methodology}). In particular, the method can handle multivariate time series \textit{irregularly sampled} and with \textit{variable sizes}.
  
  \item We motivate our extension of LDDMM to time series by introducing a theoretical framework with a representation theorem for time series graph (\Cref{theorem:representation}) and kernels related to their structure (\Cref{lemma:choice_of_kernel_V}).
  
  \item We demonstrate the identifiability of the model by estimating the true generating parameter of synthetic data, and we highlight the sensitivity of our method concerning its hyperparameters (\Cref{appendix: settings_identifiability}), also providing guidelines for tuning (\Cref{appendix:kernel_TS_LDDMM}).
  
  \item We highlight the \textit{interpretability} of TS-LDDMM for studying the inter-individual variability in a clinical dataset (\Cref{section:experiments}).
  
  \item We illustrate the quantitative interest of such representation on classification tasks on real shape-based datasets with regular and irregular sampling (\Cref{appendix: robustness,appendix: shape_classification}).
\end{itemize}

\begin{figure}[t]
  \centering
  \includegraphics[width=0.7\linewidth]{"./pictures/diffeo.jpeg"}
  \caption{A time series' graph $\msg=\{(t,s(t)): \eqsp t\in\msi\} $ can lose its structure after applying a general diffeomorphism $\phi.\msg$: a time value can be related to two values on the space axis.}
  \label{fig:diffeo}
\end{figure}

\begin{figure*}[t]
  \centering
  \includegraphics[width=\linewidth]{"./pictures/transport.jpeg"}
  
  \caption{LDDMM and TS-LDDMM are applied to ECG data.
  We observe that LDDMM, using a general Gaussian kernel, does not learn the time translation of the first spike but changes the space values, i.e., one spike disappears before emerging at a translated position. At the same time, TS-LDDMM handles the time change in the shape.
  This difference of \textit{deformations} implies differences in features \textit{representations}.   }
  \label{fig:transport}
  
\end{figure*}

  \vspace{-1ex}
\section{Notations}
We denote by integer ranges by $[k:l]=\{k,\ldots,l\}\subset \mathcal{P}(\Zset)$ and $ [l]=[1:l]$ with $k,l\in \Nset$, by $\rmC^m(\msi,\mse)$ the set of $m$-times continously differentiable function defined on an open set $\msu$ to a normed vector space $\mse$, by $||u||_\infty=\sup_{x\in \msu} |u(x)| $ for any bounded function $u:\msu \to \mse$, and by $\Nset_{>0}$ is the set of positive integers. 

\vspace{-1ex}
\section{Background on LDDMM}
\vspace{-1ex}
\label{section:LDDMM}

In this part, we expose how to learn the diffeomorphisms $(\phi_j)_{j\in[N]}$ using LDDMM, initially introduced in \cite{beg2005computing}. In a nutshell, for any $j\in [N]$, $\phi_j$ corresponds to a differential flow related to a learnable velocity field belonging to a well-chosen Reproducing Kernel Hilbert Space (RKHS).

In the next section, time series are going to be represented by diffeomorphism parameters $(\alpha_j)_{j\in[N]}$. That is why LDDMM is chosen since it offers a parametrization for diffeomorphisms that is sparse and interpretable, two features particularly relevant in the biomedical context.
 
The basic problem that we consider in this section is the following. Given a set of targets $\mathbf{y}=(y_i)_{i\in[T_2]}$ in $\Rset^{d'}$\footnote{Note that we denote by $d'\in\nset$ the ambient space}, a set of starting points $\mathbf{x}=(x_{i})_{i\in[T_1]}$ in $\Rset^{d'}$, we aim to find a diffeomorphism $\phi$ such that the finite set of points $\mathbf{y}$ is similar in a certain sense to the set of finite sets of transformed points $\phi \cdot \mathbf{x} =(\phi(x_i))_{i\in[T_1]} $. The function $\phi$ is occasionally referred to as a \textit{deformation}. In general, these sets $\mathbf{x},\mathbf{y}$ are meshes of continuous objects, e.g., surfaces, curves, images, and so on.

\vspace{-1ex}
\paragraph{Representing diffeomorpshims as deformations.}
Such \textit{deformations} $\phi$ are constructed via differential flow equations, for any $x_0\in \Rset^{d'} $ and $\tau\in[0,1]$:
\begin{equation}
  \label{eq:LDDMM_dynamic}
    \frac{\dd X(\tau)}{\dd \tau}= v_\tau(X(\tau)), \quad X(0)=x_0\eqsp ,
    \phi^v_\tau(x_0)=X(\tau), \quad \phi^v=\phi^v_1  \eqsp ,
\end{equation}
where the velocity field is $v:\tau\in [0,1]\mapsto v_\tau\in \msv $ and $\msv$ is a Hilbert space of continuously differentiable function on $\Rset^{d'}$. If $||\dd u ||_{\infty}+|| u ||_{\infty}\leq ||u ||_\msv $ for any $u\in \msv$ and $v\in \rml^2([0,1],\msv)=\{v\in \rmC^0([0,1],\msv): \int_0^1 ||v_\tau||^2_\msv \dd \tau<\infty \} $, by \citep[Theorem 5]{glaunes2005transport} $\phi^v$ exists and belongs to $\mcd(\Rset^{d'})$, where we denote by $\mcd(\mso) $ the set of diffeomorpshim defined on an open set $\mso$ to $\mso$. Therefore, for any choice of $v$, $\phi^v$ defines a valid deformation. This offers a general recipe to construct diffeomorphism given a functional space $\msv$.

With this in mind, the velocity field $v$ fitting the data can be estimated by minimizing $v \in \rml^2([0,1],\msv) \mapsto \mathscr{L}(\phi^{v}.\mathbf{x},\mathbf{y})$, where $\mathscr{L}$ is an appropriate loss function. However, two computational challenges arise. First, this optimization problem is ill-posed, and a penalty term is needed to obtain a unique solution. In addition, a parametric family $\msv_{\Theta} \subset \rml^2([0,1],\msv)$, parameterized by $\Theta$, is sought to efficiently solve this minimization problem.

\vspace{-1ex}
\paragraph{From deformations to geodesics.}
It has been proposed in \cite{miller2006geodesic} to interpret $\msv$ as a tangent space relative to the group of diffeomorphisms $\msh=\{ \phi^v:\eqsp v\in \rml^2([0,1],\msv)\}$. Following this geometric point of view, geodesics can be constructed on $\msh$ by using the following squared norm 
\begin{equation}
  \label{eq:geodesics_original}
  \mathscr{R}^2: g\in \msh\mapsto \inf_{ v\in \rml^2([0,1],\msv):\eqsp g=\phi^v} \int_0^1 || v_\tau||_\msv^2\dd \tau
\end{equation}
By deriving differential constraints related to the minimum of \eqref{eq:geodesics_original} and using Cauchy-Lipschitz conditions, geodesics can be defined only by giving the starting point and the initial velocity $v_0\in \msv$ \cite{miller2006geodesic}, as straight lines in Euclidean space. Denoting by $\tau \mapsto \rho_{v_0}(\tau)\in\msh$ the geodesic starting from the identity with inital velocity $v_0\in \msv$, the exponential map is defined as $\varphi^{\{v_0\}}\triangleq \rho_{v_0}(1)$. Using $\varphi^{\{v_0\}}$ instead of $\phi^v$, the previous matching problem becomes a \textit{geodesic shooting problem}:
\begin{equation}
  \label{eq:geodesics_shooting}
  \inf_{v_0 \in \msv} \mathscr{L}(\varphi^{\{v_0\}}.\mathbf{x},\mathbf{y}).
\end{equation}
Using $\varphi^{\{v_0\}}$ instead of $\phi^v$ for any $v\in \rml^2([0,1],\msv)$ regularizes the problem and induces a sparse representation for the learning diffeomorphisms. Moreover, by setting $\msv$ as an RKHS, the geodesic shooting problem has a unique solution and becomes tractable, as described in the next section.

\vspace{-1ex}
\paragraph{Discrete parametrization of diffeomorpshim.}

In this part, $\msv$ is chosen as an RKHS \cite{berlinet2011reproducing} generated by a smooth kernel $K$ (e.g., Gaussian). We follow \cite{durrleman2013sparse} and define a discrete parameterization of the velocity fields to perform geodesics shooting \eqref{eq:geodesics_shooting}. The initial velocity field $v_0$ is chosen as a finite linear combination of the RKHS basis vector fields, 
$\mathbf{n}_0$ control points $\msx_0=(x_{k,0})_{k\in[\mathbf{n}_0]}\in (\Rset^{d'})^{\mathbf{n}_0}$ and momentum vectors $\alpha_0=(\alpha_{k,0})_{k\in[\mathbf{n}_0]}\in (\Rset^{d'})^{\mathbf{n}_0} $ are defined such that for any $x\in \Rset^{d'}$, 
\begin{equation}
  \label{eq:def_v0}
  v_0\left(\alpha_0,\msx_0\right)(x)=\sum_{k=1}^{\mathbf{n}_0} K(x,x_{k,0})\alpha_{k,0} \eqsp .
\end{equation}
In our applications, the control points $(x_{k,0})_{k\in[\mathbf{n}_0]}$ can be understood as the discretized graph $(t_k,\mathbf{s}_0(t_k))_{k\in[\mathbf{n}_0]}$ of a starting time series $\mathbf{s}_0$. With this parametrization of $v_0$, \cite{miller2006geodesic} show that the velocity field $v$ of the solution of \eqref{eq:geodesics_shooting} keeps the same structure along time, such that for any $x\in\Rset^{d'}$ and $\tau\in[0,1]$, 
\begin{equation}
  \label{eq:specific_form}
  v_\tau(x)=\sum_{k=1}^{\mathbf{n}_0} K(x,x_k(\tau))\alpha_{k}(\tau) \eqsp ,
\end{equation}
\begin{equation} 
  \label{eq:integration}
    \left\{
      \begin{aligned}
      & \frac{\dd x_k(\tau)}{\dd \tau}=v_\tau(x_k(\tau)) \eqsp, \quad
      \frac{\dd \alpha_k(\tau)}{\dd \tau}=-\sum_{k=1}^{\mathbf{n}_0} \dd_{x_k(\tau)} K(x_k(\tau),x_l(\tau))\alpha_{l}(\tau)^\top \alpha_{k}(\tau) \eqsp  \\
      & \alpha_k(0)=\alpha_{k,0},\quad x_k(0)=x_{k,0} \eqsp , k\in[\mathbf{n}_0] 
      \end{aligned}
      \right .
\end{equation}
These equations are derived from the hamiltonian $H:(\alpha_k,x_k)_{k\in [\mathbf{n}_0]}\mapsto \sum_{k,l=1}^{\mathbf{n}_0} \alpha_{k}^\top K(x_k,x_l)\alpha_{l}  $, such that the velocity norm is preserved $||v_\tau||_{\msv}=||v_0||_\msv $ for any $\tau\in [0,1]$. By \eqref{eq:integration}, the velocity field related to a geodesic $v^*$ is fully parametrized by its initial control points and momentum $(x_{k,0},\alpha_{k,0})_{k\in[\mathbf{n}_0]}$. Thus, given a set of targets $\mathbf{y}=(y_i)_{i\in[T_2]}$ in $\Rset^{d'}$, a set of starting points $\mathbf{x}=(x_{i,0})_{i\in[T_1]}$ in $\Rset^{d'}$, a RKHS's kernel $K:\Rset^{d'}\times \Rset^{d'}\to \Rset^{d'\times d'}$, a distance on sets $\mathscr{L}$, a numerical integration scheme of ODE and a penalty factor $\lambda>0$, the basic geodesic shooting step minimizes the following function using a gradient descent method:
\begin{equation}
  \label{eq:relaxation}
  \mathcal{F}_{\mathbf{x},\mathbf{y}}: (\alpha_k)_{k\in [T_1]}\mapsto \mathscr{L}\left(\varphi^{\{v_0\}}.\mathbf{x},\mathbf{y}\right)+\lambda||v_0||_\msv^2 \eqsp,  
\end{equation}
where $v_0$ is defined by \eqref{eq:def_v0} and $\varphi^{\{v_0\}}.\mathbf{x}$ is the result of the numerical integration of \eqref{eq:integration} using control points $\mathbf{x}$ and initial momentums $(\alpha_k)_{k\in[T_1]} $. 

\vspace{-1ex}
\paragraph{Relation to Continuous Normalizing Flows.}
One particular popular choice to address the problem of learning a diffeomorphism or a velocity field is Normalizing Flows \cite{rezende2015variational,kobyzev2020normalizing} (NF) or their continuous counterpart \cite{chen2018neural,grathwohl2019scalable,salman2018deep} (CNF). However, we do not rely on this class of learning algorithms for several reasons. Indeed, existing and simple normalizing flows are not suitable for the type of data that we are interested in this paper \cite{feng2023multi,deng2020modeling}. In addition,  they are primarily designed to have tractable Jacobian functions, while we do not require such property in our applications. Finally, the use of a differential flow solution of an ODE \eqref{eq:LDDMM_dynamic} trick is also at the basis of CNF, which then consists of learning a velocity field to address in fitting the data through a loss aiming to address the problem at hand. Nevertheless, the main difference between CNF and LDDMM lies in the parametrization of the velocity field. LDDMM uses kernels to derive closed form formula and enhance interpretability while NF and CNF take advantage of deep neural networks to scale with large dataset in high dimensions.

\vspace{-1ex}
\section{Methodology}
\label{section:methodology}
\vspace{-1ex}

We consider in this paper observations which consist in a population of $N$ multivariate time series, for any $j\in[N]$, $s^j \in \rmC^1(\msi_j,\Rset^{d})$. However, we can only access a $n_j$-samples $\tsig^j=(\tsig_i^j=s^j(t^j_i))_{i\in[n_j]}$ collected at timestamps $(t^j_i)_{i\in[n_j]}$ for any $j \in [N]$. Note that \textbf{the number of samples $n_j$ is not necessarily the same across individuals} and the timestamps can be \textbf{irregularly sampled}. We assume the time series population is globally homogeneous regarding their "shapes" even if inter-individual variability exists. Intuitively speaking, the "shape" of a time series $s:\msi\to \Rset^d$ is encoded in its graphs $\msg(s)$ defined as the set $\{(t,s(t)):\eqsp t\in\msi \} $ and not only in its values $s(\msi)=\{s(t):\eqsp t\in\msi \} $ since the time axis is crucial. As a motivating use-case, $s^j$ can be the time series of a heartbeat extracted from an individual's electrocardiogram (ECG), see \Cref{fig:transport}. The homogeneity in a resulting dataset comes from the fact that humans have similar shapes of heartbeat \cite{ye2012heartbeat,madona2021pqrst}.

\vspace{-1ex}
\paragraph*{The deformation problem.}
In this paper, we aim to study the inter-individual variability in the dataset by finding a relevant representation of each time series. Inspired from the framework of shape analysis \cite{vaillant2004statistics}, addressing similar problems in morphology, we suggest to represent each time series' graph $\msg(s^j)$ as the transformation of a reference graph $\msg(\mathbf{s}_0)$, related to a time series $\mathbf{s}_0:\msi \to\Rset^d$, by a diffeomorphism $\phi_j$ on $\Rset^{d+1}$, for any $j\in[N]$,
\begin{equation}
 \label{eq:transformation}
 \phi_j.\msg(\mathbf{s}_0)=\{\phi_j\left(t,\mathbf{s}_0(t)\right), \eqsp t\in \msi \} \eqsp.
\end{equation}
$\bfs_0$ will be understood as the typical representative shape common to the collection of time series $(s^j)_{j\in[N]}$. As $\bfs_0$ is supposed to be fixed, then the representation of the time series $(s^j)_{j\in[N]}$ boils down to the one of the transformation $(\phi_j)_{j\in[N]}$. We aim to learn $\msg(\bfs_0)$ and $(\phi_j)_{j\in[N]} $. 

\vspace{-1ex}
\paragraph{Optimization related to \eqref{eq:transformation}.}
Defining the \textit{discretized graphs} of the time series $(s^j)_{j\in[N]}$ and a discretization of the reference graph $\msg(\mathbf{s}_0)$ as, for any $j\in[N]$,
\begin{equation}
  \label{eq:descretized_graph}
  \mathbf{y}_j=\msg(\tsig^j)=(t_i^j,\tsig^j_i)_{ i\in[n_j]}\in (\Rset^{d+1})^{n_j},\quad \tilde{\msg}_0=(t_i^0,\tsig^0_i)_{i\in[\mathbf{n}_0]}\in (\Rset^{d+1})^{\mathbf{n}_0} \eqsp ,
\end{equation}
with $\mathbf{n}_0=\operatorname{median}((n_j)_{j\in[N]})$, the representation problem given in \eqref{eq:transformation} boils down solving:
\begin{equation}
  \label{eq:general_optimization_problem}
  \argmin_{\tilde{\msg}_0,(\alpha_k^j)_{k\in [\mathbf{n}_0]}^{j\in[N]}} \sum_{j=1}^N \mathcal{F}_{\tilde{\msg}_0,\mathbf{y}_j}\left((\alpha_k^j)_{k\in [\mathbf{n}_0]}\right)\eqsp ,
\end{equation}
which is carried out by gradient descent on the control points $\tilde{\msg}_0$ and the momentums $\mathbf{\alpha}_j=(\alpha_k^j)_{k\in [\mathbf{n}_0]}$ for any $j\in[N]$, initialized by a dataset's time series graph of size $\mathbf{n}_0$ and by $0_{(d+1)\mathbf{n}_0}$ respectively. The optimization hyperparameter details are given in \Cref{appendix:optimizers_details}. The result of the minimization $\tilde{\msg}_0$ is then considered as the $\mathbf{n}_0$-samples of a common time series $\mathbf{s}_0$ and the momentums $\mathbf{\alpha}_j$ encoding $\phi_j$ yields a feature vector in $\Rset^{d \mathbf{n}_0} $ of $s^j$ for any $j\in[N]$. Finally, the vectors $(\mathbf{\alpha}_j)_{j\in[N]}$ can be analyzed with any statistical or machine learning tools such as Principal Components Analysis (PCA), Latent Discriminant Analysis (LDA), longitudinal data analysis and so on.

Nevertheless, \eqref{eq:general_optimization_problem} asks to define a kernel and a loss in order to perform geodesic shooting \eqref{eq:relaxation}, which is the purpose of the following subsection.

\vspace{-1ex}
\subsection{Application of LDDMM to time series analysis: TS-LDDMM}
\vspace{-1ex}
\label{section:time_series_specificity}
This section presents our theoretical contribution: we tailor the LDDMM framework to handle time series data. The reason is that applying a general diffeomorphism $\phi$ from $\Rset^{d+1}$ to a time series' graph $\msg(s)$ can result in a set $\phi.\msg(s)$ that does not correspond to the graph of any time series, as illustrated in the \Cref{fig:diffeo}. Thus, time series graphs have more structure than a simple 1D curve \cite{glaunes2008large} and deserve their unique analysis, which will prove fruitful as demonstrated in \Cref{section:experiments}.
       
To address this challenge, we need to identify an RKHS kernel $K:\Rset^{d+1}\times \Rset^{d+1}\to \Rset^{(d+1)^2}$ that generates deformations preserving the structure of the time series graph. This goal motivates us to clarify, in \Cref{theorem:representation}, the specific representation of diffeomorphisms we require before presenting a class of kernels that produce deformations with this representation.
       
Similarly, selecting a loss function on sets $\mathscr{L}$ that considers the temporal evolution in a time series' graph is crucial for meaningful comparisons with time series data. Consequently, we introduce the oriented Varifold distance. 

\vspace{-1ex}
\paragraph{A representation separating space and time.}
We prove that two time series graphs can always be linked by a time transformation composed with a space transformation. Moreover, a time series graph transformed by this kind of transformation is always a time series graph. We define $\Psi_\gamma\in \mcd(\Rset^{d+1}) : (t,x)\in\Rset^{d+1}\to (\gamma(t),x)$ for any $\gamma\in \mcd(\Rset)$ and $\Phi_f:  (t,x)\in\Rset^{d+1}\to (t,f(t,x)) $ for any $f\in \rmC^1(\Rset^{d+1},\Rset^d)$. We have the following representation theorem. All proofs are given in \Cref{appendix:proofs}. 

Denote by $\msg(s)\triangleq \{ (t,s(t)): \eqsp t\in \msi \} $ the graph of a time series $s: \msi \to \Rset^d$ and $ \phi.\msg(s)\triangleq\{ \phi(t,s(t)): \eqsp t\in \msi\} $ the action of  $\phi\in \mcd(\Rset^{d+1}) $ on $\msg(s)$.
\begin{theorem}
    \label{theorem:representation}
    Let $s:  \msj \to \Rset^d  $ and $\mathbf{s}_0: \msi\to \Rset^d $ be two continuously differentiable time seriess with $\msi,\msj$ two intervals of $\Rset$.
    There exist $f\in \rmC^1(\Rset^{d+1},\Rset^d)$ and $\gamma\in  \mcd(\Rset) $ such that $\gamma(\msi)=\msj $ and $\Phi_f\in \mcd(\Rset^{d+1})$,
    \begin{equation}
      \msg(s)= \Pi_{\gamma,f}.\msg(\mathbf{s}_0),\eqsp \Pi_{\gamma,f}=\Psi_\gamma\circ\Phi_f.
    \end{equation}
    Moreover, for any $\bar{f}\in \rmC^1(\Rset^{d+1},\Rset^d)$ and $\bar{\gamma}\in  \mcd(\Rset) $, there exists a continously differentiable time series $\bar{s}$ such that $\msg(\bar{s})= \Pi_{\bar{\gamma},\bar{f}}.\msg(\mathbf{s}_0)$
\end{theorem}

\begin{remark}
  Note that for any $\gamma \in \mcd(\Rset) $ and $s\in \rmC^0(\msi,\Rset^d)$,
  \begin{equation}
    \{(\gamma(t),s(t)),\eqsp t\in \msi \}=\{(t,s\circ \gamma^{-1}(t)):\eqsp t\in\gamma(\msi) \}\eqsp .
  \end{equation}
  As a result, $\Psi_\gamma $ can be understood as a temporal reparametrization and $\Phi_f$ encodes the transformation about the space.
\end{remark}

\vspace{-1ex}
\paragraph{Choice for the kernel associated with the RKHS $\msv$}
\label{paragraph:kernel_V}
As depicted on \Cref{fig:diffeo}-\ref{fig:transport}, we can not use any kernel $K$ to apply the previous methodology to learn deformations on time series' graphs. We describe and motivate our choice in this paragraph. Denote the one-dimensional Gaussian kernel by $K_\sigma^{(a)}(x,y)=\exp(-|x-y|^2/\sigma)$ for any $(x,y)\in (\Rset^a)^2$, $a\in \Nset$ and $\sigma>0$. To solve the geodesic shooting problem \eqref{eq:relaxation} on $\Rset^{d+1}$, we consider for $\msv$ the RKHS associated with the kernel defined for any $(t,x),(t',x')\in (\Rset^{d+1})^2$:
\begin{align}
  \label{eq:kernel_TAS}
  &K_{\msg}((t,x),(t',x'))=
  \begin{pmatrix}
    c_0K_{\text{time}} & 0 \\
    0 & c_1 K_{\text{space}} 
  \end{pmatrix} \eqsp ,\\
  & K_{\text{space}}=K_{\sigma_{T,1}}^{(1)}(t,t')K_{\sigma_x}^{(d)}(x,x') \Idd\eqsp,K_{\text{time}}=K_{\sigma_{T,0}}^{(1)}(t,t') \eqsp,
\end{align}
parametrized by the widths $\sigma_{T,0},\sigma_{T,1},\sigma_x>0$ and the constants $c_0,c_1>0$. This choice for $K_\msg$ is motivated by the representation \Cref{theorem:representation} and the following result. 
\begin{lemma}
  \label{lemma:choice_of_kernel_V}
  If we denote by $\msv$ the RKHS associated with the kernel $K_{\msg}$, then for any vector field $v$ generated by \eqref{eq:integration} with $v_0$ satisfying \eqref{eq:def_v0}, there exist $\gamma \in \msd(\Rset) $ and $f\in \rmC^1(\Rset^{d+1},\Rset^d)$ such that $\phi^v=\Psi_\gamma\circ\Phi_f $.
\end{lemma}
Instead of Gaussian kernels, other types of smooth kernels can be selected as long as the structure \eqref{eq:kernel_TAS} is respected.

\begin{remark}
  \label{remark:spaceandtime}
  With this choice of kernel, the features associated with the time transformation can be extracted from the momentums $(\alpha_{k,0})_{k\in[\mathbf{n}_0]}\in (\Rset^{d+1})^{\mathbf{n}_0}$ in \eqref{eq:def_v0} by taking the coordinates related to time. However, the features related to the space transformation are not only in the space coordinates since the related kernel $K_{\text{space}}$ depends on time as well.The kernel's representation has been carefully designed to integrate both space and time, while ensuring that time remains independent of space. Initially, we considered separating the spatial and temporal components. However, post-hoc analysis of such a representation proved to be challenging. The separated spatial and temporal representations are correlated, and understanding this correlation is essential for interpreting the data. As a result, concatenating the two representations becomes necessary, though there is no straightforward method for doing so, as they are not commensurable. Consequently, we opted for a representation that inherently integrates both space and time.
\end{remark}
In \Cref{appendix:kernel_TS_LDDMM}, we give guidelines for selecting the hyperparameters $(\sigma_{T,0},\sigma_{T,1},\sigma_x,c_0,c_1)$.
  
\vspace{-1ex}
\paragraph{Loss}
This section specifies the distance function $\scrl$ introduced in the loss function defined in \eqref{eq:relaxation}. 

In practice, we can only access discretized graphs of time series, $(t_i^j,\tsig^j_i)_{i\in[n_j]}$ for any $j\in[N]$, that are potentially of different sizes $n_j$ and sampled at different timestamps $(t_i^j)_{i\in[n_j]}$ for any $j\in[N]$. Usual metrics, such as the Euclidean distance, are not appealing as they make the underlying assumptions of equal size sets and the existence of a pairing between points. Distances between measures on sets (taking the empirical distribution), such as Maximum Mean Discaprency (MMD) \cite{dziugaite2015training,borgwardt2006integrating}, alleviate those issues; however, MMD only accounts for positional information and lacks information about the time evolution between sampled points. A classical data fidelity metric from shape analysis corresponding to the distance between \textit{oriented varifolds} associated with curves alleviates this last issue \cite{kaltenmark2017general}. Intuitively, an oriented varifold is a measure that accounts for positional and tangential information about the underlying curves at sample points. More details and information about \textit{oriented varifolds} can be found in \Cref{appendix:varifold}. 

More precisely, given two sets $\msg_0=(g_i^0)_{i\in[T_0]},\msg_1=(g_i^1)_{i\in[T_1]}\in (\Rset^{d+1})^{T_1}$ and a kernel\footnote{$\mathbb{S}^d=\{x\in\Rset^{d+1}:\eqsp |x|=1\}$} $k:(\Rset^{d+1} \times \mathbb{S}^d)^2\to \Rset$ verifying \citep[Proposition 2 \& 4]{kaltenmark2017general}, for any $\xi\in\{0,1\}$ and $i\in[T_\xi-1]$, denoting the center and length of the $i^{th}$ segment $[g_i^\xi,g_{i+1}^\xi]$ by $c_i^\xi = (g_i^\xi + g_{i+1}^\xi)/2$, $l_i^\xi = \| g_{i+1}^\xi-g_{i}^\xi\|$, and 
$\overrightarrow{v_i}^\xi = (g_{i+1}^\xi-g_{i}^\xi)/l_i^\xi$, the varifold distance between $\msg_0$ and $\msg_1$  is defined as,
\begin{align}
  &d_{\msw^*}^2(\msg_0,\msg_1) = \sum_{i,j = 1}^{T_0-1}l^0_i k((c^0_i,\overrightarrow{v_i}^0),(c^0_j,\overrightarrow{v_j}^0))l^0_j
  - 2 \sum_{i=1}^{T_0-1}\sum_{j=1}^{T_1-1}l^0_i k((c^0_i,\overrightarrow{v_i}^0),(c^1_j,\overrightarrow{v_j}^1))l^1_j \\
  &+ \sum_{i,j = 1}^{T_1-1}l^1_i k((c^1_i,\overrightarrow{v_i}^1),(c^1_j,\overrightarrow{v_j}^1))l^1_j 
\end{align}

In practice, we set the kernel $k$ as the product of two anisotropic Gaussian kernels, $k_{\pos}$ and $k_{\dir}$, 
such that for any $(x,\overrightarrow{u}),(y,\overrightarrow{v}) \in (\Rset^{d+1} \times \mathbb{S}^d)^2$
\begin{equation}
  k((x,\overrightarrow{u}),(y,\overrightarrow{v})) = k_{\pos}(x,y)k_{\dir}(\overrightarrow{u},\overrightarrow{v}) \eqsp.
\end{equation}
Note that the loss kernel $k$ has nothing to do with the velocity field kernel denoted by $K_\msg$ or $K$ specified in \Cref{paragraph:kernel_V}. Finally, we define the data fidelity loss function, $\scrl$, as a sum of $ d_{\msw^*}^2$ using different kernel's width parameters $\sigma$ to incorporate multiscale information. $\scrl$ is indeed differentiable with respect to its first variable. The specific kernels $k_{\pos},k_{\dir}$ that we use in our experiments are given \Cref{appendix:kernel_implementation}. For further readings on curves and surface representation as varifolds, readers can refer to \cite{kaltenmark2017general,charon2013varifold}. 

A pedagogical \href{https://tslddmmapp.streamlit.app}{\underline{online application}} is available to inspect the effect of hyperprameters on geodesic shooting \eqref{eq:integration} and registration \eqref{eq:relaxation}.
 
\vspace{-1ex}
\section{Experiments}
\label{section:experiments}
[L'intro est ce necessaire ?]


First, we show on synthetic data that the proposed representation is identifiable provided that the hyperparameters and the reference graph are wisely selected, i.e.,
 the parameter $v_0^*$ generating a deformation $\varphi^{\{v_0^*\}}$ of a time series graph $\msg$ can be estimated from the data $\msg,\varphi^{\{v_0^*\}}.\msg$ by solving the geodesic shooting problem \eqref{eq:relaxation}.
 Secondly, we illustrate the qualitative interest of TS-LDDMM in studying inter-individual variability on a clinical dataset.
  Thirdly, we demonstrate the quantitative performance of our representation by performing classification on shape-based datasets.
  The method is implemented on Python using the library JAX\footnote{https://github.com/google/jax}. The code was compiled on a server with NVIDIA RTX A2000 12GB GPU, Intel(R) Xeon(R) Gold 5220R CPU @ 2.20GHz, and 250 GB of RAM. The code will be available on Github.
\subsection{Synthetic experiments}
\begin{figure}[t]
    \centering
    \includegraphics[width=0.5\linewidth]{pictures/samples.pdf}
    \vspace{-2.5em}
    \caption{Plots of $\varphi^{\{v_0(\mathbf{\alpha}^*,\msx)\}}.\msx$ for different values of $\mathbf{\alpha}^*$ according to its sampling parameter $t_a,s_a,m_s $, taking $\msx=\msg(s_0)$ with $s_0:k\in [300]\to \sin(2\pi k/300) $.}
    \label{fig:exemple_synthetic}
    \vspace{-1em}
\end{figure}

\begin{table}
    \caption{Values of $\scrl(\varphi^{\{v_0(\mathbf{\alpha}^*,\msx)\}}.\msx,\varphi^{\{\hat{v}_0\}}.\msx)$ as $\mathbf{\alpha}^*$ is sampled according to Gen(10,10,50) and $\hat{v}_0$ is estimated using $K_\msg$ with varying parameters $\sigma_{T,1},\sigma_x$.}
      \centering
         \begin{tabular}{lrrrrrrr}
         \toprule
         $\sigma_{T,0} \backslash \sigma_x$  & 1 & 10 & 50 & 100 & 200 & 300 \\
         \midrule
         0.1 & 2e+0 & 3e-4  & 1e-5&4e-6&7e-4&4e-3 \\
        1 & 4e-2 & 1e-4  & 1e-5&4e-6&7e-4 &4e-3  \\
         100 & 4e-2 & 2e-4  & 1e-5&4e-6&7e-4&4e-3  \\
         \bottomrule
         \end{tabular}
      \label{table:synthetic2}
      \vspace{-1em}
  \end{table}
 


% \begin{subfigure}{0.7\linewidth}
%   \centering
%   \resizebox{\columnwidth}{!}{%
%      \begin{tabular}{lrrrr}
%      \toprule
%      $\sigma_{T,0} \backslash \sigma_x$  & 0.1 & 1 & 100 \\
%      \midrule
%      1 & 2e+0 & 4e-2 & 4e-2 \\
%      10 & 3e-4 & 1e-4  & 2e-4 \\
%      50 & 1e-5 & 1e-5  & 1e-5 \\
%      100 & 4e-6 & 4e-6 & 4e-6 \\
%      200 & 7e-4 & 7e-4 & 7e-4 \\
%      300 & 4e-3 & 4e-3 & 4e-3 \\
%      \bottomrule
%      \end{tabular}
%      %
%      }
%  \caption{Values of $\scrl(y^*,\varphi^{\{\hat{v}_0\}}.\msx)$ where $y^*$ is sampled according to Gen(10,10,50) and $\hat{v}_0$ is estimated using $K_\msg$ with varying parameters $\sigma_{T,1},\sigma_x$.}
% \end{subfigure}
First, we show the model identifiability when the kernel $K_G$ is well specified: the estimated parameter is a good approximation of the generating parameter when the generation and the estimation procedure use the same hyperparameters for the RKHS kernel $K_\msg$.
All the hyperparameter values for generation and estimation are given in \Cref{appendix:numerics_synthetic}.
We fix the initial control points as $\msx=\left(x_k=(k,\sin(2\pi k/300))\right)_{k\in[300]} $.
Given $m_{s}\in \Nset_{>0}$ and $t_{a},s_{a}>0$, we randomly generate initial momentums $\mathbf{\alpha}^*=(\alpha_k^*)_{k\in[\mathbf{n}_0]}$ with the following sampling, called Gen($m_s,t_a,s_a$):
 For any $k\in[\mathbf{n}_0]$, $\mathbf{\alpha}_k'$ is sampled according to a Gaussian normal distribution $\mathcal{N}(0_{d+1},I_{d+1})$.
Then, $(\alpha_k')_{k\in[\mathbf{n}_0]}$ is regularized by a rolling average of size $m_{s}$, we get $\bar{\mathbf{\alpha}}'=(\bar{\alpha}_k')_{k\in[\mathbf{n}_0]}$.
Finally, we normalize $\bar{\mathbf{\alpha}}'$ to derive $\mathbf{\alpha}^*$ such that $|([\alpha_k^*]_t)_{k\in[\mathbf{n}_0]}|=t_{\text{amp}}$ and $|([\alpha_k^*]_s)_{k\in[\mathbf{n}_0]}|=s_{\text{amp}}$ for any $k\in[\mathbf{n}_0]$, denoting by $[\alpha_k^*]_t,[\alpha_k^*]_s$ the time and space coordinates of $\alpha_k^*$ respectively.
Note that the regularizing step $(\mathbf{\alpha}_k')_{k\in[\mathbf{n}_0]}\to \bar{\mathbf{\alpha}}' $ is necessary to obtain realistic deformations which take into account the regularity induced by the RKHS $\msv$.
%  uniformly on the set $\mss_M=\{ (\alpha_k)_{k\in[\mathbf{n}_0]}\in (\Rset^{d+1})^{\mathbf{n}_0}:\eqsp |\alpha_k|=M\}  $ with $M>0,\mathbf{n}_0\in \Nset_{>0}$,
%  then we regularize $\mathbf{\alpha}$ by convolving
%   with ... to get the generation parameter $\mathbf{\alpha}^*$.
Then, using $v_0(\mathbf{\alpha}^*,\msx)$ as defined in \eqref{eq:def_v0} with initial momentums $\mathbf{\alpha}^*$ and control points $\msx$, we apply the induced deformation $\varphi^{\{v_0\}} $ by \eqref{eq:integration} to $\msx$ and obtain $\varphi^{\{v_0\}}.\msx$.
Finally, we solve \eqref{eq:relaxation} to recover an estimation $\hat{\mathbf{\alpha}}$ of $\mathbf{\alpha}^*$ and report the average relative error (ARE) $|v_0(\hat{\mathbf{\alpha}},\msx)-v_0(\mathbf{\alpha}^*,\msx)|_\msv/|v_0(\mathbf{\alpha}^*,\msx)|_\msv$ on 50 repetitions.
This procedure is performed for any $m_{s},t_{a},s_{a}\in \{10,50,100\}\times \{5,10,15,20\}^2 $.
 Mean, standard deviation, and maximum of the ARE on all these hyperparameters choices are respectively $\mathbf{0.10, 0.03, 0.17}$.
 Therefore, the estimation procedure \eqref{eq:relaxation} offers a good approximation of the true parameter when the kernel $K_\msg$ is well specified.
 We observe that the estimation is difficult when $t_a\ll s_a$ because the time series can be very noisy as illustrated in \Cref{fig:exemple_synthetic}: this impacts the Varifold loss which is sensitive to tangents.
  
 Secondly, we demonstrate a weak identifiability when the kernel $K_\msg$ is misspecified: we can reconstruct the graph time series' after deformations even if the hyperparameters of $K_\msg$
  are different during the generation and the estimation.
   The hyperparameters of $K_\msg$ during generation are $(c_0,c_1,\sigma_{T,0},\sigma_{T,1},\sigma_x)=(1,0.1,100,1,1)$ and we fix $\sigma_{T,1},c_0,c_1=(1,1,0.1) $ for $K_\msg$ during estimation.
   We aim to understand the impact of $\sigma_{T,1},\sigma_x$ on the reconstruction since they are encoding the smoothness of the transformation according to time and space.  

    For any choice of the hyperparameters $\sigma_{T,1},\sigma_x\in \{1,10,50,100,200,300 \}\times \{0.1,1,100\}$ related to $K_\msg$ in the estimation,
     we average $\scrl(\varphi^{\{v_0(\mathbf{\alpha}^*,\msx)\}}.\msx,\varphi^{\{\hat{v}_0\}}.\msx)$ on 50 repetitions when $\mathbf{\alpha}^*$ is sampled according to Gen$(10,10,50)$ and $\hat{v}_0=v_0(\hat{\alpha},\msx)$ denoting by $\hat{\alpha}$ the result of the minimization \eqref{eq:relaxation}.
  We observe in \Cref{table:synthetic2} that the reconstruction is almost perfect except in the case when $\sigma_{t,0}=1$ during estimation, while $ \sigma_{t,0}=100$ during generation.
   Compared to $\sigma_{T,0}$, $\sigma_x$ has nearly no impact on the reconstruction.
   In \Cref{appendix:kernel_implementation}-\ref{appendix:kernel_TS_LDDMM}, we propose guidelines to drive future hyperparameters tuning and further discussions related to $\sigma_{T,1},c_0,c_1$. 
\subsection{Qualitative analysis of respiratory behavior in mice}
%We consider a dataset composed of mouse respiratory time series before and after a drug injection.
% A complete presentation of this dataset is given in \Cref{appendix:mouse_dataset}.
%  The mouse are divided in two groups depending on their genotypes : \textit{colq} and \textit{wt}.
%  We aim to study the difference of respiratory shapes according to the genotype.
%   That is why, TS-LDDMM \eqref{eq:general_optimization_problem} is applied to derive the features representations $(\mathbf{\alpha}_j)_{j\in[N]}\in (\Rset^2)^N$ related to $N$=\sam{fill} respiratory time series coming from 14 mouse (7 \textit{colq} and \textit{wt}) before drug injection.
%   Then, a Principal Components Analysis (PCA) is performed on $(\mathbf{\alpha}_j)_{j\in[N]}$
\begin{figure*}[t]
  \centering
  \includegraphics[width=0.95\linewidth]{"./pictures/exp_1_bis.pdf"}
  \vspace{-1.5em}
  \caption{Analysis of the three principal components (PC) related to the respiratory cycles of the mouse before exposure.
  In Figure A), the densities of each genotype according to each PC are displayed. In Figure B), the deformations of the reference graph $S_0$ along each PC are given. In Figure D), the graph of reference $S^j$, also called barycenter, related to each mouse, is displayed according to their coordinates on PC1 and PC3. In Figure C) et E), illustrations of respiratory cycles related to mice coming from the \textbf{wt} and \textbf{colq} group are displayed.  }
  \label{fig:exp_1_PCA}
  %\vspace{-1em}
\end{figure*}

\begin{figure*}[t]
  \centering
  \includegraphics[width=0.95\linewidth]{"./pictures/exp_2.pdf"}
  \vspace{-1em}
  \caption{Analysis of the first Principal Component (PC1) related to the respiratory cycles of the mouse 
  before and after exposure. In Figure A), the densities of each genotype according are displayed. In Figure B), the deformations of the reference graph $S_0$ PC1 is given. In Figure C), respiratory cycles displayed with respect to time and according to their coordinates on PC1 and PC2}
  \label{fig:exp_2_PCA}
  \vspace{-1.5em}
\end{figure*}

This experiment highlights the \textit{interpretability} of TS-LDDMM for studying the inter-individual variability in a clinical dataset.
We consider a time series dataset recording the evolution of the respiratory airflow of mice exposed 
to an irritant molecule altering respiratory functions \cite{nervo2019respiratory}. The dataset is divided into two groups, one 
composed of 7 control mice (\textbf{wt}) and the other of 7 mice (\textbf{colq}) deficient in an enzyme 
involved in the control of respiration. For each mouse, the respiratory airflow was recorded for 
15 to 20 minutes before exposure to the irritant molecule and then for 35 to 40 minutes. A complete 
description of the dataset is given in the \Cref{appendix:mouse_dataset}.
By comparing the shape of individual respiratory cycles (inspiration + expiration, see \Cref{fig:exp_1_PCA}-C)), 
we show that TS-LDDMM features can encode genotype distinctive breathing behaviors and their evolution 
after exposure to the irritant molecule. 

We first compare breathing behaviors before exposure.
Solving \eqref{eq:general_optimization_problem}, we derive the reference respiratory cycle's graph $S_0$ and the TS-LDDMM features representations 
$(\alpha_j)_{j\in[N_1]}$ related to $N_1=700$ respiratory cycles extracted according 
to the procedure \cite{germain2023unsupervised}.
Then, we perform a kernel PCA on the initial velocity field $\left(v_0(\alpha_j,S_0)\right)_{j\in[N_1]}\in \msv^{N_1}$ defined in \eqref{eq:def_v0}.
In \Cref{fig:exp_1_PCA}, we focus on the analysis of the three Principal Components (PC).

As observable from \Cref{fig:exp_1_PCA}-B), principal components refer to different types of deformations. 
By interpreting \Cref{fig:exp_1_PCA}-B): Only PC1 accounts for time warping, PC2 expresses the trade-off between inspiration and expiration duration, and PC3 corresponds to a change in signal amplitude.
 Compared to \textbf{wt} mice, the distribution of \textbf{colq} mice 
TS-LDMMM feature representation along the PC1 axis has a heavy tail and the associated deformation (+3 $\sigma_{\text{PC}}$) 
shows an inspiration with two peaks. As illustrated in \Cref{fig:exp_1_PCA}-A), such respiratory cycles are preponderant
 with \textbf{colq} mice and may be caused by motor impairment due to their enzyme deficiency, \cite{germain2023unsupervised}.
  In addition, 
 the \textbf{colq} mice were smaller than the \textbf{wt} mice due to a delay in growth caused by their lack of an enzyme. 
 This difference can be seen on PC3 since the volumes of air (area under the curve) inspired and exhaled are
  smaller for the smaller mice. In correlation, the distribution of \textbf{wt} mice TS-LDDMM feature representations 
  along the PC3 axis have a heavy tail corresponding to large air volume as depicted by the deformation 
  (+3 $\sigma_{\text{PC}}$) in \Cref{fig:exp_1_PCA}-B). Finally, \Cref{fig:exp_1_PCA}-D) shows that PC1 and PC3 capture the main differences between
   the two groups as their respective reference graphs $S^j$ are located in different parts of the space. 
%In Figure \Cref{fig:exp_1_PCA}, we display coordinates densities per genotype and PC.

We perform a second experiment to analyze the evolution of breathing behaviors when mice are exposed to the irritant 
molecule. We follow the same procedure as before. However, we take $N_2=1400$ with 25\% (resp. 75\%) 
before (resp. after) exposure. In \Cref{fig:exp_2_PCA}, we focus on the first principal component PC since
it encodes the effect of the irritant molecule as depicted in \Cref{fig:exp_2_PCA}-C) (the exposure occurs at 20 minutes).
\Cref{fig:exp_2_PCA}-B) shows that the deformation (+3 $\sigma_{\text{PC}}$) leads to longer respiratory cycles that include pauses,
as observed in \cite{germain2023unsupervised}. As well, \Cref{fig:exp_2_PCA}-A) shows that TS-LDDMM features distributions are less spread 
out for \textbf{colq} mice compared to \textbf{wt} mice. Indeed, the irritant molecule inhibits the action of the 
deficient enzyme, \textbf{wt} mice strongly react to the irritant molecule, whereas \textbf{colq} mice are better 
adapted due to their deficiency.
%Secondly, we compare breathing behaviors before and after exposure to observe the impact of the irritant molecule.
% We follow the same procedure as for before exposure, but we take $N_2=1400$ respiratory cycles extracted according 
% to the procedure CITE....
% In \Cref{fig:exp_2_PCA}, we focus on the first Principal Component (PC) since it encodes the effect of the irritant molecule as demonstrated on Figure \ref{fig:exp_2_PCA}-C) (exposure at time 20).
% We observe on \Cref{fig:exp_2_PCA}-B) that after exposure the mouse have longuer breath such that the expiration is longuer than inspiration.
%  Moreover, we deduce from Figure \ref{fig:exp_2_PCA}-A) that \textbf{colq} are more constant in their breath compared to \textbf{wt} after exposure.
%
% In Figure REF, we 
% also display the reference respiratory cycle S0 and its deformations in the principal component directions.
%  Additionally, we learn each mouse's reference respiratory cycle and represent them in the first and third PC coordinates system in Figure REF. 



\subsection{Quantitative performances of the TS-LDDMM representation in classification}
Combined with a Support Vector Classifier (SVC) \cite{hsu2003practical}, TS-LDDMM representation can be used for 
classification tasks using the kernel associated with the initial velocity space $\msv$.
We compare TS-LDDMM-SVC classification performances with another SVC using representation 
learned with T-loss \cite{franceschi2019unsupervised}, an unsupervised deep learning feature 
representation method for time series. We also include fully supervised methods in deep learning 
-ResNet, CNN \cite{ismail2019deep}- and machine learning: Catch22 \cite{lubba2019catch22}, Rocket
\cite{dempster2020rocket}, Dynamic Time Wrapping k-Nearest Neighbors (DTW-kNN) 
\cite{muller2007dynamic}. Methods are compared using f1-score on several shape-based UCR/UEA datasets 
\cite{dau2019ucr,bagnall2018uea} introduced in \Cref{appendix:classification_dataset}. All implementation details are given 
in \Cref{appendix:classification_implementation}.
\Cref{table:classification} presents the reuslts. TS-LDDMM-SVC consistently outperforms the other unsupervised methods. It is ranked 1,3,4,3 for all methods combined, 
demonstrating its competitiveness as an unsupervised method on time series dataset homogeneous regarding shape.
%TS-LDDMM representation can be used in a Support Vector Classifier (SVC) \cite{hsu2003practical} using the kernel of the space of initial velocity $\msv$.
%We compare its capacity of representation to an SVC using T-Loss representation \cite{franceschi2019unsupervised}, which is an unsupervised representation deep learning method for time series,
%and to others fully supervised methods using deep learning -ResNet, CNN \cite{ismail2019deep}- or not: Catch22 \cite{lubba2019catch22}, Rocket \cite{dempster2020rocket}, Dynamic Time Wrapping \cite{muller2007dynamic} k-Nearest Neighbors (DTW-kNN).
%These methods are compared using f1-score on several datasets introduced in \Cref{appendix:classification_dataset}. All details of implementations are given in \Cref{appendix:classification_implementation}.
%TS-LDDMM-SVC is always ranked 2 or 3 behind a fully supervised method, demonstrating its competitiveness as an unsupervised method.
%The more homogeneous the shapes in the dataset are, the better TS-LDDMM-SVC performs.


\begin{table}[t]
  \vspace{-0.5em}
  \caption{Classification results in f1-score (U: unsupervised, S: supervised, DL: deep learning, ML: machine learning). \textbf{x} best unsupervised method, \underline{x} best supervised method.}
  \resizebox{\linewidth}{!}{%
  \begin{tabular}{lllrrrr}
    \toprule
     &  &  & ArrowHead & ECG200 & GunPoint & NATOPS \\
    \midrule
    \multirow[m]{3}{*}{U} & \multirow[t]{3}{*}{} & TS-LDDMM-SVC & \textbf{0.84} & \textbf{0.82} & \textbf{0.94} & \textbf{0.93} \\
     &  & T-loss-SVC & 0.57 & 0.76 & 0.82 & 0.88 \\
     &  & DTW-kNN & 0.70 & 0.75 & 0.91 & 0.88 \\
    \cline{1-7} 
    \multirow[m]{5}{*}{S} & \multirow[m]{2}{*}{DL} & CNN & 0.70 & 0.79 & 0.85 & \underline{0.96} \\
     &  & ResNet & 0.77 & 0.87 & 0.97 & 0.95 \\
    \cline{2-7}
     & \multirow[m]{2}{*}{ML} & Catch22 & 0.73 & 0.81 & 0.96 & 0.89 \\
     &  & Rocket & \underline{0.81} & \underline{0.91} & \underline{1.00} & 0.88 \\
    \bottomrule
    \end{tabular}
    \label{table:classification}
  %
}
\vspace{-2em}
  \end{table}
\vspace{-1ex}

\section{Related Works}
\vspace{-1ex}

Shape analysis focuses on statistical analysis of mathematical objects invariant under some deformations like rotations, dilations, or time parameterization. The main idea is to represent these objects in a complete Riemannian manifold $(\mathcal{M},\mathbf{g})$ with a metric $\mathbf{g}$ adapted to the geometry of the problem \cite{miller2006geodesic}. Then, any set of points in $\mathcal{M}$ can be represented as points in the tangent space of their Frechet mean $\mathbf{m}_0$ \cite{pal2017riemannian,le2001locating} by considering their logarithms. The goal is to find a well-suited Riemannian structure according to the nature of the studied object.
 
LDDMM framework is a relevant shape analysis tool to represent curves as depicted in \cite{glaunes2008large}. However, graphs of time series are a well-structured type of curve due to the inclusion of the temporal dimension that requires specific care (\Cref{fig:diffeo}). In a similar vein, Qiu \textit{et al} \cite{qiu2009time} proposes a method for tracking anatomical shape changes in serial images using LDDMM. They include temporal evolution, but not for the same purpose: the aim is to perform longitudinal modeling of brain images.

Leaving the LDDMM representation, the results of \cite{srivastava2010shape,heo2024logistic} address the representation of curves with the Square-Root Velocity (SRV) representation. However, the SRV representation is applied after reparametrization of the temporal dimension of the unit length segment. Consequently, the graph structure of the time series is not respected, and the original time evolution of the time series is not encoded in the final representation. Very recently, in a functional data analysis (FDA) framework, a paper \cite{wu2024shape} (Shape-FPCA) improved by representing the original time evolution. However, the space and time representations remain correlated, complicating post-hoc analysis, as discussed in  \Cref{remark:spaceandtime}. Additionally, this method is tailored for \textit{continuous objects} and applies only to time series of the \textit{same length}, making the estimation more sensitive to noise. This issue can be addressed through interpolation, but this approach is not always reliable in sparse and irregular sampling scenarios. Most FDA approaches, as seen in \cite{shang2014survey,yu2017principal,warmenhoven2021pca}, address this challenge using interpolation or basis function expansion. In summary, FDA methods typically separate space and time representations for continuous objects, whereas TS-LDDMM algorithm maintain a discrete-to-discrete analysis, inherently integrating both space and time representations.

Balancing between discrete and continuous elements is a challenging task. In the deep learning literature \cite{chen2018neural, kidger2020neural, tzen2019neural, jia2019neural, liu2019neural, ansari2023neural}, Neural Ordinary Differential Equations (Neural ODEs) \cite{chen2018neural} learn continuous latent representations using a vector field parameterized by a neural network, serving as a continuous analog to Residual Networks \cite{zagoruyko2016wide}. This approach was further enhanced by Neural Controlled Differential Equations (Neural CDEs) \cite{kidger2020neural} for handling irregular time series, functioning as continuous-time analogs of RNNs \cite{schuster1997bidirectional}. Extending Neural ODEs, Neural Stochastic Differential Equations (Neural SDEs) introduce regularization effects \cite{liu2019neural}, although optimization remains challenging. Leveraging techniques from continuous-discrete filtering theory, Ansari et al. \cite{ansari2023neural} applied successfully Neural SDEs to irregular time series. Oh \textit{et al.} \cite{oh2024stable} improved these results by incorporating the concept of controlled paths into the drift term, similar to how Neural CDEs outperform Neural ODEs. With TS-LDDMM, the representation is also derived from an ODE, but the velocity field is parameterized with kernels and optimized to have a minimal norm, which enhances interpretability.

All these state-of-the-art methods previously mentionned \cite{glaunes2008large,oh2024stable,wu2024shape,heo2024logistic} are compared to TS-LDDMM in \Cref{appendix: robustness} and \Cref{appendix: shape_classification}.

Compared to the Metamorphosis framework \cite{blanz2003face}, LDDMM framework has weaker assumptions.
The 3DMM framework requires that each mesh be re-parametrized into a consistent form where the number of vertices, triangulation, and the anatomical meaning of each vertex are consistent across all meshes, as stated in the introduction of \cite{booth20163d}.
In our context, we do not need such pre-processing; the time series graph can have different sizes.
\vspace{-1ex}
\section{Limitations and conclusion}
\vspace{-1ex}
\label{sec:limitations}
  
This paper proposes a feature representation method, TS-LDDMM, designed for 
shape comparison on homogeneous time series datasets. We show on a real dataset 
its ability to study, with high interpretability, the inter-individual shape 
variability. As an unsupervised approach, it is user-friendly and enables knowledge 
transfer for different supervised tasks such as classification.

Although TS-LDDMM is already competitive for classification, its performances can be leveraged on more heterogeneous datasets using a hierarchical clustering extension, which is relegated for future work. 

TS-LDDMM employs kernel computations, which require specific libraries (e.g., KeOps \cite{charlier2021kernel}) to be efficient and scalable. However, in our experiments, the time complexity of TS-LDDMM is comparable to that of competitors.
 It is clear that TS-LDDMM needs to be extended to handle very large datasets with high-dimensional time series (such as videos).
 
Additionally, TS-LDDMM requires tuning several hyperparameters, though this is a common requirement among competitors \cite{glaunes2008large, oh2024stable, wu2024shape, heo2024logistic}. In future work, adaptive methods are expected to be developed to provide a more user-friendly interface.