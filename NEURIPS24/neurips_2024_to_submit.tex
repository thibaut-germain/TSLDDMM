\documentclass{article}


% if you need to pass options to natbib, use, e.g.:
%     \PassOptionsToPackage{numbers, compress}{natbib}
% before loading neurips_2024


% ready for submission
\usepackage[final]{neurips_2024}


% to compile a preprint version, e.g., for submission to arXiv, add add the
% [preprint] option:
%     \usepackage[preprint]{neurips_2024}


% to compile a camera-ready version, add the [final] option, e.g.:
%     \usepackage[final]{neurips_2024}


% to avoid loading the natbib package, add option nonatbib:
%    \usepackage[nonatbib]{neurips_2024}


\usepackage[utf8]{inputenc} % allow utf-8 input
\usepackage[T1]{fontenc}    % use 8-bit T1 fonts
\usepackage{hyperref}       % hyperlinks
\usepackage{url}            % simple URL typesetting
\usepackage{booktabs}       % professional-quality tables
\usepackage{amsfonts}       % blackboard math symbols
\usepackage{nicefrac}       % compact symbols for 1/2, etc.
\usepackage{microtype}      % microtypography
\usepackage{xcolor}         % colors
\usepackage{amsmath}
\usepackage{amssymb}
\usepackage{mathtools}
\usepackage{amsthm}
\usepackage{caption}
\usepackage{subcaption}
\usepackage{wrapfig}
\usepackage[capitalize,noabbrev]{cleveref}

\title{Shapes analysis for time series}


% The \author macro works with any number of authors. There are two commands
% used to separate the names and addresses of multiple authors: \And and \AND.
%
% Using \And between authors leaves it to LaTeX to determine where to break the
% lines. Using \AND forces a line break at that point. So, if LaTeX puts 3 of 4
% authors names on the first line, and the last on the second line, try using
% \AND instead of \And before the third author name.


\author{%
  Thibaut Germain$^1$\thanks{Corresponding author. Contact at \texttt{thibaut.germain@ens-paris-saclay.fr}}\\
  Centre Borelli, ENS Paris-Saclay\\
  4 av. des sciences, 91190 \\
  \And
  Samuel Gruffaz$^1$\\
  Centre Borelli, ENS Paris-Saclay\\
  4 av. des sciences, 91190 \\
  \AND
  Charles Truong$^1$\\
  Centre Borelli, ENS Paris-Saclay\\
  4 av. des sciences, 91190 \\
  \And
  Laurent Oudre$^1$\\
  Centre Borelli, ENS Paris-Saclay\\
  4 av. des sciences, 91190 \\
  \And
  Alain Durmus \\
  CMAP, CNRS, Ecole polytechnique\\
  Institut Polytechnique de Paris\\
  91120 Palaiseau, France \\
}


% ready for submission
\usepackage[utf8]{inputenc} % allow utf-8 input
\usepackage[T1]{fontenc}    % use 8-bit T1 fonts

\usepackage{wrapfig,lipsum,booktabs}

%\usepackage{comment}

\usepackage{paralist}
\usepackage{booktabs}       % professional-quality tables
\usepackage{nicefrac}       % compact symbols for 1/2, etc.
\usepackage{multirow}
 \usepackage{xargs}

%\usepackage[tbtags]{amsmath}
%\usepackage{amsthm}
%\usepackage{bm}
\allowdisplaybreaks
\usepackage{amssymb,mathrsfs}
\usepackage{amsfonts}
\usepackage{mathtools}
\usepackage{upgreek}
\usepackage{graphicx}
\usepackage{wrapfig}
%\usepackage[dvipsnames]{xcolor}
\usepackage{soul}
\usepackage{pifont}
\usepackage{bbm}
%\usepackage[colorlinks = true, citecolor = black]{hyperref}
%\usepackage{algpseudocode,algorithm,algorithmicx}
\usepackage{stmaryrd}
\usepackage{array}
\usepackage{enumitem}
\def\UrlBreaks{\do\/\do-}
\usepackage{tikz}
\usepackage{pgfplots}
\usepackage{aliascnt}
\usepackage{todonotes}
\usepackage{cleveref}
\let\etoolboxforlistloop\forlistloop % save the good meaning of \forlistloop
\usepackage{autonum}
\let\forlistloop\etoolboxforlistloop % restore the good meaning of \forlistloop
\usepackage{bm}

\usepackage{accents}

\newcommand{\cmark}{\textcolor{green!80!black}{\ding{51}}}
\newcommand{\xmark}{\textcolor{red}{\ding{55}}}

\newcommand\barbelow[1]{%
  \underaccent{\bar}{#1}}

  %% Maxime' comments 
\definecolor{lightred}{rgb}{1, 0.8, 0.8}
\DeclareRobustCommand{\maxime}[1]{
{
    \begingroup
    \sethlcolor{lightred}
    \hl{(Maxime:) #1}
    \endgroup} 
}
\DeclareMathAlphabet{\mathpzc}{OT1}{pzc}{m}{it}
\theoremstyle{plain}
\makeatletter
\newtheorem{theorem}{Theorem}
\crefname{theorem}{theorem}{Theorems}
\Crefname{Theorem}{Theorem}{Theorems}


%\newtheorem*{lemma_nonumber*}{Lemma}


%\newaliascnt{lemma}{theorem}
\newtheorem{lemma}{Lemma}
%\aliascntresetthe{lemma}
%\crefname{lemma}{lemma}{lemmas}
%\Crefname{Lemma}{Lemma}{Lemmas}



\newaliascnt{corollary}{theorem}
\newtheorem{corollary}[corollary]{Corollary}
\aliascntresetthe{corollary}
\crefname{corollary}{corollary}{corollaries}
\Crefname{Corollary}{Corollary}{Corollaries}

\newaliascnt{proposition}{theorem}
\newtheorem{proposition}[proposition]{Proposition}
\aliascntresetthe{proposition}
\crefname{proposition}{proposition}{propositions}
\Crefname{Proposition}{Proposition}{Propositions}

\newaliascnt{definition}{theorem}
\newtheorem{definition}[definition]{Definition}
\aliascntresetthe{definition}
\crefname{definition}{definition}{definitions}
\Crefname{Definition}{Definition}{Definitions}

\newaliascnt{remark}{theorem}
\newtheorem{remark}[remark]{Remark}
\aliascntresetthe{remark}
\crefname{remark}{remark}{remarks}
\Crefname{Remark}{Remark}{Remarks}


\newtheorem{example}[theorem]{Example}
\crefname{example}{example}{examples}
\Crefname{Example}{Example}{Examples}


\crefname{figure}{figure}{figures}
\Crefname{Figure}{Figure}{Figures}

\newtheorem{assumption}{\textbf{H}\hspace{-3pt}}
\crefformat{assumption}{{\textbf{H}}#2#1#3}
\newtheorem{assumptionsup}{\textbf{A}\hspace{-3pt}}
\crefformat{assumptionsup}{{\textbf{A}}#2#1#3}
\newtheorem{assumptionnona}{\textbf{N}\hspace{-3pt}}
\crefformat{assumptionnona}{{\textbf{N}}#2#1#3}

\newtheorem{assumptionnonsup}{\textbf{M}\hspace{-3pt}}
\crefformat{assumptionnonsup}{{\textbf{M}}#2#1#3}


%\usepackage{floatrow}
%\newfloatcommand{capbtabbox}{table}[][\FBwidth]

\def\rmM{\mathrm{M}}
\def\msf{\mathsf{G}}
\def\borel{\mathcal{B}}
\def\Qker{\mathrm{Q}}
\def\Rker{\mathrm{R}}
\def\Sker{\mathrm{S}}
\def\Pens{\mathscr{P}}
\def\Ltt{\mathtt{L}}
\def\Mtt{\mathtt{M}}
\def\Jac{\mathrm{Jac}}
\def\dom{\mathrm{dom}}
\def\msd{\mathsf{D}}
\def\msw{\mathsf{W}}
\newcommand{\rref}[1]{\textup{\Cref{#1}}}
\def\rmA{\mathrm{A}}
\def\rmB{\mathrm{B}}
\def\msu{\mathsf{U}}
\newcommand{\maxence}[1]{\todo[inline,color=green!40]{{\bf Maxence:~}#1}}
\newcommand{\valentin}[1]{\todo[inline,color=green!40]{{\bf Valentin:~}#1}}
%\newcommand{\alain}[1]{\todo[inline,color=green!40]{{\bf Alain:~}#1}}

% MACROS
% \newcommand{\w}{\mathbf{w}}
% \newcommand{\x}{\mathbf{x}}
% \newcommand{\y}{\mathbf{y}}
% \newcommand{\J}{\mathbf{J}}
% \newcommand{\f}{\mathbf{f}}
%\newcommand{\R}{\mathbb{R}}
\newcommand{\Image}{\text{Im}}
\newcommand{\Ker}{\text{Ker}}
\newcommand{\TM}{\Pi^\star\msm}
\newcommand{\Th}{\mathrm{S}_{h/2}}
\newcommand{\Rh}{\mathrm{R}_{h}}
\newcommand{\Gh}{\mathrm{G}_{h}}
\newcommand{\ug}{\underline{\mathrm{g}}}
\newcommand{\oGh}{\overline{\mathrm{G}_{h}}}
\newcommand{\uGh}{\underline{\mathrm{G}_{h}}}
\newcommand{\ugh}{\underline{\mathrm{g}_{h}}}
\newcommand{\ughz}{\underline{\mathrm{g}_{h,z}}}
\newcommand{\Fh}{\mathrm{F}_{h}}
\newcommand{\tildeFh}{\tilde{\mathrm{F}}_{h}}
\newcommand{\tildeGh}{\tilde{\mathrm{G}}_{h}}
\newcommand{\In}{\text{I}_n}
\newcommand{\phiRh}{\mathrm{R}_{h}^{\Phi}}
\newcommand{\tildeRh}{\tilde{\mathrm{R}}_{h}}
\newcommand{\Tminush}{\Pi_{-h/2}}
\newcommand*{\myotop}{\perp\mkern-20.7mu\bigcirc}
\newcommand{\TMx}[1]{\Pi^\star_{#1}\msm}


\newcommand{\tcmb}[1]{\textcolor{blue}{#1}}
\newcommand{\tcmo}[1]{\textcolor{orange}{#1}}
\newcommand{\tcmv}[1]{\textcolor{darkgreen}{#1}}


\def\M{\msm}
\def\vol{\text{vol}}

%%% Alain


%%% mathsf
\def\complementary{\mathrm{c}}
\def\msi{\mathsf{I}}
\def\msj{\mathsf{J}}
\def\msa{\mathsf{A}}
%\def\msd{\mathsf{X}}
\def\msdd{\mathsf{D}}
\def\msk{\mathsf{K}}
\def\msK{\mathsf{K}}
\def\msks{\mathsf{K}^{\star}}
\def\mss{\mathsf{S}}
\def\msn{\mathsf{N}}
\def\msat{\tilde{\mathsf{A}}}
\def\tgamma{\tilde{\gamma}}
\def\msb{\mathsf{B}} 
\def\msc{\mathsf{C}}
\def\mse{\mathsf{E}}
\def\msf{\mathsf{F}}
\def\mso{\mathsf{O}}
\def\msg{\mathsf{G}}
\def\msh{\mathsf{H}}
\def\bmsh{\bar{\msh}}
\def\msm{\mathsf{M}}
\def\msu{\mathsf{U}}
\def\msv{\mathsf{V}}
\def\msr{\mathsf{R}}
\newcommand{\msff}[2]{\mathsf{F}_{#1}^{#2}}
\def\msp{\mathsf{P}}
\def\msq{\mathsf{Q}}
\def\msx{\mathsf{X}}
\def\msz{\mathsf{Z}}
\def\msy{\mathsf{Y}}



%% mathcal
\def\mca{\mathcal{A}}
\def\bmca{\bar{\mathcal{A}}}
\def\mcd{\mathcal{D}}
\def\mcl{\mathcal{L}}
\def\mcs{\mathcal{S}}
\def\mct{\mathcal{T}}
\def\tmct{\tilde{\mathcal{T}}}
\def\tmcs{\tilde{\mathcal{S}}}
\def\bmcl{\bar{\mathcal{L}}}
\def\bmcs{\bar{\mathcal{S}}}
\def\bmct{\bar{\mathcal{T}}}
\def\mcat{\tilde{\mathcal{A}}}
\def\mcab{\bar{\mathcal{A}}}
\def\mcbb{\mathcal{B}}  %%% \mcb est déjà pris
\newcommand{\mcb}[1]{\mathcal{B}(#1)}
\def\mcc{\mathcal{C}}
\def\mcy{\mathcal{Y}}
\def\mcx{\mathcal{X}}
\def\mce{\mathcal{E}}
\def\mcf{\mathcal{F}}
\def\mcg{\mathcal{G}}
\def\mch{\mathcal{H}}
\def\mcm{\mathcal{M}}
\def\mcu{\mathcal{U}}
\def\mcv{\mathcal{V}}
\def\mcr{\mathcal{R}}
\def\tmcr{\tilde{\mathcal{R}}}
\newcommand{\mcff}[2]{\mathcal{F}_{#1}^{#2}}
\def\mcfb{\bar{\mathcal{F}}}
\def\bmcf{\bar{\mathcal{F}}}
\def\mcft{\tilde{\mathcal{F}}}
\def\tmcf{\tilde{\mathcal{F}}}
\def\mcp{\mathcal{P}}
\def\mcq{\mathcal{Q}}
\def\mcz{\mathcal{Z}}

%% mathbb

\def\rset{\mathbb{R}}
\def\mE{\mathbb{E}}
\def\rsets{\mathbb{R}^*}
\def\cset{\mathbb{C}}
\def\zset{\mathbb{Z}}
\def\nset{\mathbb{N}}
\def\nsets{\mathbb{N}^*}
\def\qset{\mathbb{Q}}
\def\Rset{\mathbb{R}}
\def\Cset{\mathbb{C}}
\def\Zset{\mathbb{Z}}
\def\Nset{\mathbb{N}}
\def\Tset{\mathbb{T}}


%%%% mathrm 
\def\rmm{\mathrm{m}}
\def\Rsetm{\mathrm{m}}
\def\rmd{\mathrm{d}}
\def\Rsetd{\mathrm{d}}
\def\rmf{\mathrm{f}}
\def\Rsetl{\mathrm{L}}
\def\RsetH{\mathrm{H}}
\def\rmH{\mathrm{H}}
\def\rml{\mathrm{L}}
\def\Rsete{\mathrm{e}}
\def\rme{\mathrm{e}}
\def\rmn{\mathrm{n}}
\def\Rsetn{\mathrm{n}}
\def\RsetC{\mathrm{C}}
\def\RsetCb{\mathrm{C}_{\operatorname{b}}}
\def\rmCb{\mathrm{C}_{\operatorname{b}}}
\def\Rsetc{\mathrm{c}}
\def\rmc{\mathrm{c}}
\def\rmC{\mathrm{C}}
\def\rma{\mathrm{a}}
\def\Rseta{\mathrm{a}}
\def\rmU{\mathrm{U}}
\def\rmY{\mathrm{Y}}
\def\rmb{\mathrm{b}}

%%%% mathfrak
\def\frX{\mathfrak{X}}
\def\frg{\mathfrak{g}}


% Operands
\newcommand{\absolute}[1]{\left\vert #1 \right\vert}
\newcommand{\abs}[1]{\left\vert #1 \right\vert}
\newcommand{\absLigne}[1]{\vert #1 \vert}
\newcommand{\tvnorm}[1]{\| #1 \|_{\mathrm{TV}}}
\newcommand{\tvnormEq}[1]{\left \| #1 \right \|_{\mathrm{TV}}}
\newcommand{\Vnorm}[2][1=V]{\| #2 \|_{#1}}
\newcommand{\normpi}[2][2=\Rsetl^2(\pi)]{\left\Vert  #1 \right\Vert_{#2}}
\newcommand{\normH}[2][2=2]{\left\Vert  #1 \right\Vert}
\newcommand{\normHLigne}[2][2=2]{\Vert  #1 \Vert}
\newcommand{\normHLine}[2][2=2]{\Vert  #1 \Vert}
\newcommand{\normmu}[2][2=2]{\left\Vert  #1 \right\Vert_{#2}}
\newcommand{\normopmu}[2][2=2]{\left\vvvert  #1 \right\vvvert_{#2}}
\newcommand{\normoppi}[2][2=\rml^2(\pi)]{\left\vvvert  #1 \right\vvvert_{#2}}
\newcommand{\normopH}[2][2=2]{\left\vvvert  #1 \right\vvvert}
\newcommand{\normop}[2][2=2]{\left\vvvert  #1 \right\vvvert}
\newcommandx{\psr}[3][3=]{\left\langle#1,#2 \right\rangle_{#3}}
\newcommandx{\normr}[2][2=]{ \left\Vert#1 \right\Vert_{#2}}
\newcommandx{\psrLigne}[3][3=]{\langle#1,#2 \rangle_{#3}}
\newcommandx{\normrLigne}[2][2=]{ \Vert#1 \Vert_{#2}}
\newcommand{\pspi}[2]{\left\langle#1,#2 \right\rangle_2}
\newcommand{\psH}[2]{\left\langle#1,#2 \right\rangle}
\newcommand{\psmu}[2]{\left\langle#1,#2 \right\rangle_2}
\newcommand{\psmuLigne}[2]{\langle#1,#2 \rangle_2}
\newcommand{\normpiLine}[2][2=2]{\Vert  #1 \Vert_{#2}}
\newcommand{\normpiLigne}[2][2=2]{\Vert  #1 \Vert_{#2}}
\newcommand{\normmuLine}[2][2=2]{\Vert  #1 \Vert_{#2}}
\newcommand{\normmuLigne}[2][2=2]{\Vert  #1 \Vert_{#2}}
\newcommand{\normopmuLine}[2][2=2]{\vvvert  #1 \vvvert_{#2}}
\newcommand{\normopmuL}[2][2=2]{\normopmuLine{#1}[#2]}
\newcommand{\normopHLine}[2][2=2]{\vvvert  #1 \vvvert}
\newcommand{\normopLine}[2][2=2]{\vvvert  #1 \vvvert}
\newcommand{\pspiLine}[2]{\langle#1,#2 \rangle_2}
\newcommand{\psmuLine}[2]{\langle#1,#2 \rangle_2}
\newcommand{\VnormEq}[2][1=V]{\left\| #2 \right\|_{#1}}
\newcommandx{\norm}[2][1=]{\ifthenelse{\equal{#1}{}}{\left\Vert #2 \right\Vert}{\left\Vert #2 \right\Vert^{#1}}}
\newcommand{\normLigne}[1]{\| #1 \|}
\newcommand{\crochet}[1]{\left\langle#1 \right\rangle}
\newcommand{\parenthese}[1]{\left(#1 \right)}
\newcommand{\parentheseLigne}[1]{(#1 )}
\newcommand{\parentheseDeux}[1]{\left[ #1 \right]}
\newcommand{\parentheseDeuxLigne}[1]{[ #1 ]}
\newcommand{\defEns}[1]{\left\lbrace #1 \right\rbrace }
\newcommand{\defEnsLigne}[1]{\lbrace #1 \rbrace }
\newcommand{\defEnsPoint}[1]{\left\lbrace #1 \right. }
\newcommand{\defEnsPointDeux}[1]{\left. #1 \right  \rbrace }
\newcommand{\defEnsL}[1]{\left\lbrace #1 \right. }
\newcommand{\defEnsR}[1]{\left. #1 \right  \rbrace }

% Proba
\newcommand{\proba}[1]{\mathbb{P}\left( #1 \right)}
\newcommand{\PP}{\mathbb{P}}
\newcommand{\probaLigne}[1]{\mathbb{P}( #1 )}
\newcommand\probaMarkovTilde[2][2=]
{\ifthenelse{\equal{#2}{}}{{\widetilde{\mathbb{P}}_{#1}}}{\widetilde{\mathbb{P}}_{#1}\left[ #2\right]}}
\newcommand{\probaMarkov}[2]{\mathbb{P}_{#1}\left( #2\right)}
\newcommand{\PE}{\bbE} %j'ai l'impression qu'il manquait ça
\newcommand{\expe}[1]{\PE \left[ #1 \right]}
\newcommand{\expeExpo}[2]{\PE^{#1} \left[ #2 \right]}
\newcommand{\expeLigne}[1]{\PE [ #1 ]}
\newcommand{\expeLine}[1]{\PE [ #1 ]}
\newcommand{\expeMarkov}[2]{\PE_{#1} \left[ #2 \right]}
\newcommand{\expeMarkovLigne}[2]{\PE_{#1} [ #2 ]}
\newcommand{\expeMarkovExpo}[3]{\PE_{#1}^{#2} \left[ #3 \right]}
\newcommand{\probaMarkovTildeDeux}[2]{\widetilde{\mathbb{P}}_{#1} \left[ #2 \right]}
\newcommand{\expeMarkovTilde}[2]{\widetilde{\PE}_{#1} \left[ #2 \right]}


\newcommand{\plusinfty}{+\infty}
\def\ie{\textit{i.e.}}
\def\cadlag{càdlàg}
\def\eqsp{\;}
\def\eqand{\quad \text{  and  }\quad }
\renewcommand{\iint}[2]{\{ #1,\ldots,#2\}}
\newcommand{\coint}[1]{\left[#1\right)}
\newcommand{\ocint}[1]{\left(#1\right]}
\newcommand{\ooint}[1]{\left(#1\right)}
\newcommand{\ccint}[1]{\left[#1\right]}
\newcommand{\cointLigne}[1]{[#1)}
\newcommand{\ocintLigne}[1]{(#1]}
\newcommand{\oointLigne}[1]{(#1)}
\newcommand{\ccintLigne}[1]{[#1]}
\newcommand{\cointLine}[1]{[#1)}
\newcommand{\ocintLine}[1]{(#1]}
\newcommand{\oointLine}[1]{(#1)}
\newcommand{\ccintLine}[1]{[#1]}


\newcommand{\boule}[2]{\operatorname{B}(#1,#2)}
\newcommand{\ball}[2]{\operatorname{B}(#1,#2)}
\newcommand{\cball}[2]{\overline{\operatorname{B}}(#1,#2)}
\newcommand{\ballc}[2]{\operatorname{B}^\rmc(#1,#2)}
\newcommand{\boulefermee}[2]{\overline{\mathrm{B}}(#1,#2)}
\def\TV{\mathrm{TV}}
\newcommand\sequence[3][2=,3=]
{\ifthenelse{\equal{#3}{}}{\ensuremath{\{ #1_{#2}\}}}{\ensuremath{\{ #1_{#2}, \eqsp #2 \in #3 \}}}}
\newcommand\sequenceD[3][2=,3=]
{\ifthenelse{\equal{#3}{}}{\ensuremath{\{ #1_{#2}\}}}{\ensuremath{( #1)_{ #2 \in #3} }}}

\newcommand{\sequencen}[2][2=n\in\N]{\ensuremath{\{ #1_n, \eqsp #2 \}}}
\newcommand\sequenceDouble[4][3=,4=]
{\ifthenelse{\equal{#3}{}}{\ensuremath{\{ (#1_{#3},#2_{#3}) \}}}{\ensuremath{\{  (#1_{#3},#2_{#3}), \eqsp #3 \in #4 \}}}}
\newcommand{\sequencenDouble}[3][3=n\in\N]{\ensuremath{\{ (#1_{n},#2_{n}), \eqsp #3 \}}}


\newcommand{\wrt}{w.r.t.}
\newcommand{\Withoutlog}{w.l.o.g.}
\def\iid{i.i.d.}
\def\ifof{if and only if}
\def\eg{e.g.}
\def\Id{\mathrm{Id}}
\def\Idd{\mathrm{I}_d}
\newcommand{\ensemble}[2]{\left\{#1\,:\eqsp #2\right\}}
\newcommand{\ensembleLigne}[2]{\{#1\,:\eqsp #2\}}
%\newcommand{\set}[2]{\ensemble{#1}{#2}}
\def\rmD{\mathrm{D}}%%rmd déjà pris
\def\Rsetd{\mathrm{D}}


\def\Tr{\mathrm{Tr}}
\def\generator{\calL}
\newcommand{\core}{\rmc}
\def\lambdac{\lambda_{\mathrm{c}}}
\def\bfe{\mathbf{e}}

\newcommand{\alain}[1]{\todo[color=red!20]{{\textbf AL:} #1}}
 \newcommand{\alaini}[1]{\todo[color=red!20,inline]{{\textbf AL:} #1}}
  \newcommand{\pablo}[1]{\todo[color=black!20]{{\textbf PJ:} #1}}
   \newcommand{\pabloi}[1]{\todo[color=black!20,inline]{{\textbf PJ:} #1}}
 \def\a{a}
\def\domain{\mathrm{D}}
\def\card{\operatorname{card}}
\def\bigone{\operatorname{1}}
\def\bigO{\mathcal{O}}
\def\Cp{C_{\operatorname{P}}}
\def\core{\mathsf{C}}
\def\entropyA{\mathscr{H}}
\def\range{\operatorname{Ran}}
\def\kernel{\operatorname{Ker}}
\def\spec{\operatorname{Spec}}

%%%%%%%%%%%%
% function
\def\ff{f}
\def\gg{g}

\newcommand\overlineb[1]{{#1}}
\def\poly{\mathrm{poly}}
\def\veps{\varepsilon}
\def\trace{\operatorname{Tr}}
\def\sign{\operatorname{sgn}}
\def\diag{\operatorname{diag}}
\def\diam{\operatorname{diam}}
\newcommand\tcb[1]{\textcolor{cyan}{#1}}
\newcommand\tcbb[1]{\textcolor{blue}{#1}}
\newcommand\tcr[1]{\textcolor{red}{#1}}
\def\rateConv{\alpha}
\def\vareps{\varepsilon}
\def\sphere{\mathbb{S}}
\newcommand{\1}{\mathbbm{1}}


\newcommand{\vphi}{\varphi}
\newcommand{\even}{\mathrm{e}}

\def\restriction#1#2{\mathchoice
              {\setbox1\hbox{${\displaystyle #1}_{\scriptstyle #2}$}
              \restrictionaux{#1}{#2}}
              {\setbox1\hbox{${\textstyle #1}_{\scriptstyle #2}$}
              \restrictionaux{#1}{#2}}
              {\setbox1\hbox{${\scriptstyle #1}_{\scriptscriptstyle #2}$}
              \restrictionaux{#1}{#2}}
              {\setbox1\hbox{${\scriptscriptstyle #1}_{\scriptscriptstyle #2}$}
              \restrictionaux{#1}{#2}}}
\def\restrictionaux#1#2{{#1\,\smash{\vrule height .8\ht1 depth .85\dp1}}_{\,#2}}



%\DeclareMathOperator{\Var}{Var}

%%%%%%%%%%%%%%%%%%%%%%%%%%%%%%%%%%%%%%%%%%%%%%%%%%%%5
%%%%%% Pablo

\def\rmX{\mathrm{X}}
\def\rmVar{\mathrm{Var}}

\def\Ret{\mathrm{Ret}}

\def\bphi{\bar{\phi}}
\def\bgamma{\bar{\gamma}}

\def\loiGauss{\mathrm{N}}
\def\tx{\tilde{x}}
\def\tz{\tilde{z}}



\newcommand\norme[1]{\left\Vert #1 \right\Vert}


\def\parallelTransport{\Pi}
\def\distM{\mathrm{d}_{\msm}} %we prefer \Theta for the manifold
\def\distT{\rho_{\Theta}}
\def\metricM{\mathfrak{g}}
\def\vol{\mathrm{vol}}
\def\volM{\mathrm{vol}_{\msm}}
\def\grad{\mathrm{grad}\,}
\def\Hess{\mathrm{Hess}\,}
\def\teta{\tilde{\eta}}
\def\rmD{\mathrm{D}}

\def\noise{e}
\def\hnoise{\hat{e}}
\def\Exp{\mathrm{Exp}}
\def\Log{\mathrm{Log}}
\def\planT{\rmT}
\def\Cut{\mathrm{Cut}}
\def\ID{\mathrm{ID}}
\def\dupgamma{\dot{\upgamma}}


\def\grassmann{\mathrm{Gr}}
\def\stiefel{\mathrm{St}}
\def\transpose{\top}
\def\scrl{\mathscr{L}}
\def\scrlinf{\scrl_{\infty}}
\def\scrlU{\scrl^{(1)}}
\def\scrlD{\scrl^{(2)}}

\def\metric{\metricM}

\def\cu{\overline{c}}
\def\cl{\underline{c}}

\def\sigmaZ{\sigma_0^2}
\def\sigmaU{\sigma_1^2}
\def\bfb{\mathbf{b}}
\def\ttheta{\tilde{\theta}}
\def\Gammabf{\boldsymbol{\Gamma}}

\newcommand{\beq}{\begin{equation}}
\newcommand{\eeq}{\end{equation}}

%\def\bf1{\boldsymbol{1}}
\def\proj{\operatorname{proj}}
\def\Leb{\mathrm{Leb}}
\def\Vdist{V_2}
\def\VHuber{V_1}
\def\thetas{{\theta^{\star}}}
\def\thetaspi{{\theta^{\star}_{\pi}}}
\def\supp{\mathrm{supp}}
\def\constanteC{C}

\def\ConstH{C_H}
\def\rmcl{\mathrm{Cl}}
\def\vt{\mathpzc{V}}

%\renewcommand\hmmax{0}
%\renewcommand\bmmax{0}
\def\bnu{\bar{\nu}}
\def\bupeta{\bar{\upeta}}
\def\tb{\tilde{b}}
\def\cupeta{\check{\upeta}}
\def\gVar{g} %fonctº lisse sur la variété
\def\gPlan{\mathrm{g}} %fonctº lisse sur l'esp Tangent
\def\gInter{\mathpzc{g}} %fonctº lisse sur intervalle réel
\def\dz{\operatorname{dz}}
\def\argmin{\operatorname{argmin}}

\newcommand{\floor}[1]{\left\lfloor #1 \right\rfloor}
\newcommand{\ceil}[1]{\left\lceil #1 \right\rceil}
\def\btheta{\bar{\theta}}
\def\tsigma{\tilde{\sigma}}
\def\tf{\tilde{f}}
\def\MKer{Q_{\upeta}}
\def\bu{\bar{u}}

\def\planTsT{\planT_{\thetas}\Theta}
\def\bouletan{\bar{\mathbb{B}}}
\def\SPD{\mathrm{Sym}}


\newcommand*{\dd}{\mathop{}\!\mathrm{d}}
\def\ee{~}


%%%%%%%%%%%%%%%%%%%%%%%%%%
%%%  hmc notation
\def\orbit{\mathcal{O}}
\def\F{U}


\def\Psiverlet{\Psi}
\newcommandx{\gperthmc}[2][1=,2=]{\ifthenelse{\equal{#1}{}}{\Xi}{\ifthenelse{\equal{#2}{}}{\Xi_{h,#1}}{\Xi_{#2,#1}}}}
\newcommandx{\Phiverlet}[2][1=,2=]{\ifthenelse{\equal{#1}{}}{\Phi}{\Phi_{#1}^{\circ (#2)}}}
\newcommandx{\gpertub}[2][1=,2=]{\ifthenelse{\equal{#1}{}}{g}{g_{#1}^{#2}}}
\newcommandx{\Phiverletq}[2][1=,2=]{\ifthenelse{\equal{#1}{}}{\widetilde{\Phi}}{\widetilde{\Phi}_{#1}^{\circ (#2)}}}
\newcommandx{\Phiverletqi}[2][1=,2=]{\ifthenelse{\equal{#1}{}}{\bar{\Psi}}{\bar{\Psi}_{#1}^{(#2)}}}
\newcommandx{\Pkerhmc}[2][1=,2=]{\ifthenelse{\equal{#1}{}}{\mathrm{P}}{\mathrm{P}_{#1, #2}}}
\newcommandx{\tPkerhmc}[2][1=,2=]{\ifthenelse{\equal{#1}{}}{\tilde{\mathrm{P}}}{\tilde{\mathrm{P}}_{#1, #2}}}
\newcommandx{\PkerhmcD}[2][1=,2=]{\ifthenelse{\equal{#1}{}}{\mathrm{K}}{\mathrm{K}_{#1, #2}}}
\def\rmp{\mathrm{p}}
\def\rmq{\mathrm{q}}
\def\Kmax{K_{\mathrm{m}}}
\def\mcs{\mathcal{S}}
\def\gauss{\mathrm{N}}
\def\Kker{\mathrm{K}}
\def\Tker{\Pi}
\def\Mmatrix{M}
\def\argmax{\text{argmax}}
\newcommandx{\thetahat}[1][1=]{\ifthenelse{\equal{#1}{}}{\hat{\theta}}{\hat{\theta}(#1)}}
\newcommandx{\Kerpi}[2][1=,2=]{\ifthenelse{\equal{#1}{}}{\Pi}{\Pi_{#1}^{#2}}}
\newcommandx{\poinu}[2][1=,2=]{\ifthenelse{\equal{#1}{}}{\nu}{\nu_{#1}^{#2}}}
\def\lyapD{\mathpzc{V}}
\def\LyapD{\mathpzc{V}}
\def\VlyapD{\mathpzc{V}}
\def\lyapDs{\mathpzc{V}^*}
\def\LyapDs{\mathpzc{V}^*}
\def\VlyapDs{\mathpzc{V}^*}

\def\tsig{\tilde{s}}

%%%
% mathbf
\def\bfs{\mathbf{s}}



%%%%
% accro
\def\pos{\mathrm{pos}}
\def\dir{\mathrm{dir}}

\newcommand\sam[1]{\textcolor{blue}{#1}}
\newcommand\thi[1]{\textcolor{orange}{#1}}


\begin{document}
\footnotetext[1]{Complete affiliation: Université Paris Saclay, Université Paris Cité, ENS Paris Saclay, CNRS, SSA, INSERM, Centre
Borelli, F-91190, Gif-sur-Yvette, France.}
\maketitle

\begin{abstract}
Analyzing inter-individual variability of physiological functions is particularly appealing in medical and biological contexts to describe or quantify health conditions. Such analysis can be done by comparing individuals to a reference one with time series as biomedical data.
This paper introduces an unsupervised representation learning (URL) algorithm for time series tailored to inter-individual studies. The idea is to represent time series as deformations of a reference time series. The deformations are diffeomorphisms parameterized and learned by our method called TS-LDDMM. Once the deformations and the reference time series are learned, the vector representations of individual time series are given by the parametrization of their corresponding deformation. At the crossroads between URL for time series and shape analysis, the proposed algorithm handles irregularly sampled multivariate time series of variable lengths and provides shape-based representations of temporal data.
In this work, we establish a representation theorem for the graph of a time series and derive its consequences on the LDDMM framework. We showcase the advantages of our representation compared to existing methods using synthetic data and real-world examples motivated by biomedical applications.
\end{abstract}






% citep
% \section{TODO}
% %De ce que je comprends , les gens font surtout des variational auto encoder, ou du masking
% %Mentionner que la méthode proposé peut être utilisé avec des réseaux
% %\sam{Approche NTK, fédéré personalisation, projet avec Axel, Lien entre continuous normalizing flow}
% Plan intro: 
% \begin{itemize}
%     \item Representation of patterns in time series has several applications in biology, but halso for classification, deep learning finds features automaticly.
%      Taking or not taking into account time can be done with DTW. However, either the representation is the result of a black box and need post analysis to be understand,
%       which is difficult to apply statiscal tools on it, or the representation is desgned by hand to select relevant features, but the capacity of analysis is limited by the knowledge of the designer.
%      \item In shape analysis, representation of shapes for statistical analysis in a long time issue, which has been solved mainly by the LDDMM framework. Representing shapes as diffeomorpshims of a common shape. 
%      \item However, this tool has not been yet applied on time series, even if there is already some tools developped for oriented curves. 
%      \item In this work, we propose to apply LDDMM to time series and to show its interest to derive an interpretable and unsupervised representation of patterns in time series.
%      We proposed some tools developped in the shape analysis community to the researcher interested in machine learning and time series ones.
%      \item First, we introduce our assumptions regarding the dataset and the main problem of diffeomorpshim learning. Then, we show how to solve it with LDDMM .
%       We expose how to apply it to time series by selecting wisely a RKHS kernel, motivating it with a representation theorem.
%       experiments on synthetic data are presented to show that the parameter of representation are identifiable when the time seriess are not too sharp and the kernels parameter well tuned. 
%       Then, we sh-ow how to apply the method on an unsupervised case of moouse respeitaory dataset and a supervised one on ... .

% \end{itemize}

% One option to tackle this issue is to derive ...
% feature representation of time series which depends on the problem at hand.
% which is parti... -> which is of prime interest in ...
\vspace{-1ex}
\section{Introduction}
\vspace{-1ex}
%https://arxiv.org/abs/1811.00075
%Appliqué sur les "UWaveGestureLibrary","ArticularyWordRecognition","Cricket" pour l'archive UEA
%"ArrowHead", "BME", "ECG200","FacesUCR","GunPoint" dans le cas univarié

%Idée de parler d'inter individualité, spécialiser davantage
%k shape, shapelets , ye2009time


Our goal is to analyze the inter-individual variability within a time series dataset, an approach of significant interest in physiological contexts \cite{guscelli2019importance, wang2016research, bar2012studying, germain2023unsupervised}.
 Specifically, we aim to develop an unsupervised feature representation method that encodes the specificities of individual time series in comparison to a reference time series.
In physiology, examining the various "shapes" in a time series related to biological phenomena and their variations due to individual differences or pathological conditions is common.
 However, the term "shape" lacks a precise definition and is more intuitively understood as the silhouette of a pattern in a time series. In this paper, we refer to the shape of a time series as the graph of this signal.

 Although community structures with representatives can be learned in an unsupervised manner \cite{trirat2024universal, meng2023unsupervised} using contrastive loss \cite{franceschi2019unsupervised, tonekaboni2021unsupervised, meng2023unsupervised} or similarity measures \cite{asgari2023clustering, germain2023unsupervised, paparrizos2015k, ye2009time},
  the study of inter-individual variability of shapes within a cluster \cite{niennattrakul2007inaccuracies, shirato2023identifying} remains an open problem in unsupervised representation learning (URL), particularly for \textit{irregularly sampled} time series with \textit{variable lengths}.
   
Our work explicitly focuses on learning shape-based representation of time series.
First, we propose to view the shape of a time series not merely as its curve $\{s_t:\eqsp t\in\msi\}$, but as its graph $\msg(s)=\{(t,s(t)):\eqsp t\in \msi\}$.
   Then, building on the shape analysis literature \cite{beg2005computing,vaillant2004statistics}, we adopt the Large Deformation Diffeomorphic Metric Mapping (LDDMM) framework \cite{beg2005computing,vaillant2004statistics} to analyze these graphs.
    The core idea is to represent each element $\msg(s^j)$ of a dataset $(s^j)_{j\in[N]}$ as the transformation of a reference graph $\msg(\mathbf{s}_0)$ by a diffeomorphism $\phi_j$, i.e. $\msg(s^j) \sim \phi_j . \msg(\mathbf{s}_0)$ .
    The diffeomorphism $\phi_j$ is learned by integrating an ordinary differential equation parameterized by a Reproducing Kernel Hilbert Space (RKHS).
     The parameters $(\alpha_j)_{j\in[N]}$ encoding the diffemorphisms $(\phi_j)_{j\in[N]}$ yield the representation features of the graphs $(\msg(s^j))_{j\in[N]}$.
     Finally, these shape-encoding features can be used as inputs to any statistical or machine-learning model.

However, a graph time series transformation by a general diffeomorphism is not always a graph time series, see e.g. \Cref{fig:diffeo}, thus a graph time series is more than a simple curve \cite{glaunes2008large}.
 Our contributions arise from this observation: we specify the class of diffeomorphisms to consider and show how to learn them.
  This change is fruitful in representing transformations of time series graphs as illustrated in \Cref{fig:transport}.
   %In this regard, this work is not an application of LDDMM to 1D curves \cite{glaunes2008large}.

     %In particular, the method can handle \textit{irregularly sampled} time series with \textit{variable sizes}.

    %  Our contributions can be summarized as follows:
    %  \begin{itemize}
    %    \item We propose an unsupervised method (TS-LDDMM) to analyze inter-individual variability of shapes in a time series dataset. In particular, the method can handle \textit{irregularly sampled} time series with \textit{variable sizes}.
    %    \item We motivate our extension of LDDMM to time series by introducing a theoretical framework with a representation theorem for time series graph (\Cref{theorem:representation}) and kernels related to their structure (\Cref{lemma:choice_of_kernel_V}).
    %    \item We demonstrate the identifiability of the model by estimating the true generating parameter of synthetic data, and we highlight the sensitivity of our method with respect to its hyperparameters, also providing guidelines for tuning.
    %    We highlight the \textit{interpretability} of TS-LDDMM for studying the inter-individual variability in a clinical dataset.
    %     We illustrate the quantitative interest of the representation on classification tasks on real shape-based datasets.
        
    %  \end{itemize}
    Our contributions can be summarized as follows:
    \begin{itemize}
       \item We propose an unsupervised method (TS-LDDMM) to analyze the inter-individual variability of shapes in a time series dataset (\Cref{section:methodology}). In particular, the method can handle multivariate time series \textit{irregularly sampled} and with \textit{variable sizes}.
       \item We motivate our extension of LDDMM to time series by introducing a theoretical framework with a representation theorem for time series graph (\Cref{theorem:representation}) and kernels related to their structure (\Cref{lemma:choice_of_kernel_V}).
       \item We demonstrate the identifiability of the model by estimating the true generating parameter of synthetic data, and we highlight the sensitivity of our method concerning its hyperparameters (\Cref{appendix: settings_identifiability}), also providing guidelines for tuning (\Cref{appendix:kernel_TS_LDDMM}).
       \item We highlight the \textit{interpretability} of TS-LDDMM for studying the inter-individual variability in a clinical dataset (\Cref{section:experiments}).
        \item We illustrate the quantitative interest of such representation on classification tasks on real shape-based datasets with regular and irregular sampling (\Cref{appendix: robustness}-\ref{appendix: settings_classification}).
 \end{itemize}

    %To the best of our knowledge, these theoretical and practical works have never been done before.
      % On synthetic data, we show the sensibility of the results according to the kernel parameters and give guidelines to tune it.
      % In an unsupervised framework, we vectorize respitory patterns of a mouse dataset and analyze the variations depending on the drug taken by the mouse.
      % Finally, we compare to the state-of-the-art methods in classification an ... classifier taking as inputs the representation given by LDDMM.
      
  %    \paragraph*{Related papers}
  %    %https://arxiv.org/pdf/2106.00750.pdf look to the related papers
  %  In the shape analysis community, a famous paper \cite{srivastava2010shape} addresses the problem of representation of curves $C:\msi \to \Rset^d$ by using the Square-Root Velocity (SRV) representation.
  %      This structure is particularly relevant to compute geodesics and distances between curves when the space is quotiented by time reparametrization.
  %       However, in our case the time axis deserves a particular attention, since the curves are derived from time-series. 
  %       %While \cite{srivastava2010shape} tackles the computations of geodesics and distances on quotient curves space, our aim is to get a vector representation of each pattern. 
  %       Very recently, in \cite{heo2024logistic}, this representation was used to compute PCA and to perform classification on quotient curves space, we follow the same ideas, but we focus on time series and we use LDDMM instead of SRV.
 % optimization framework with a well chosen neural architecture in order to learn automatically the features, which has been very 
% To understand these complex data, we are looking to simpler representations toperform classical statiscal analysis or to perform efficient downstream tasks of machine learning, the time series representation res time series to a real valued vector representation. in order to apply classical statiscal or machine learning tools depending on the needs such as Principal Component Analysis (PCA), logistic regression or deep learning methods for classification or anomaly detection.

% \thi{
%   Our goal is to analyze the inter-individual variability within a time series dataset, an approach of prime interest in physiological contexts \cite{guscelli2019importance,wang2016research,bar2012studying,germain2023unsupervised}. More specifically, we aim to find an unsupervised features representation method that encodes individual time series specificities compared to a reference one.
% }

% \thi{
%   Studying shape differences between time series related to biological mechanisms is a common practice in physiology to characterize healthy and pathological functioning CITE. For instance, the shapes of heartbeats in electrocardiograms are discriminant for some cardiovascular pathologies CITE. Several approaches have been proposed for such comparison. Some employ shape-based similarity measures between time series \cite{asgari2023clustering,germain2023unsupervised,paparrizos2015k,ye2009time},  others embed time series as vectors of predefined features CITE, and, with the rise of deep neural networks, unsupervised learning representation of time series~\cite{trirat2024universal,meng2023unsupervised} has shown to be a valuable approach CITE notably with contrastive learning~\cite{franceschi2019unsupervised,tonekaboni2021unsupervised,meng2023unsupervised}. However, shape-based representation of time series within cohorts~\cite{niennattrakul2007inaccuracies,shirato2023identifying} remains an open problem in URL.
% }

% \thi{
%   Our work focuses explicitly on learning shape-based representation of time series. First, we propose not to see the shape of a time series through its curve $\{s_t:\eqsp t\in\msi\}$, but rather through its graph $\msg(s)=\{(t,s(t)):\eqsp t\in \msi\}$.
%    Then, building on the shape analysis literature \cite{beg2005computing,vaillant2004statistics}, we follow the Large Deformation Diffeomorphic Metric Mapping (LDDMM) framework \cite{beg2005computing,vaillant2004statistics} to analyze these graphs. The idea is to represent each element $\msg(s^j)$ of a dataset $(s^j)_{j\in[N]}$ as the transformation of a reference graph $\msg(\mathbf{s}_0)$ by a diffeomorphism $\phi_j$ (ie $\msg(s^j) \sim \phi_j . \msg(\mathbf{s}_0)$ ).
%     The diffeomorphism $\phi_j$ is learned by integrating an ordinary differential equation parameterized by a Reproducing Kernel Hilbert Space (RKHS).
%      The parameters $(\alpha_j)_{j\in[N]}$ encoding the diffemorphisms $(\phi_j)_{j\in[N]}$ yield the representation features of the graphs $(\msg(s^j))_{j\in[N]}$. Finally, these shape-encoding features can feed any statistical or machine-learning model.
%   }

% \thi{
%   However, a graph time series transformation by a general diffeomorphism is not always a graph time series, see e.g. \Cref{fig:diffeo}, thus a graph time series is more than a simple curve \cite{glaunes2008large}. Our contributions arise from this observation: we specify the class of diffeomorphisms to consider and show how to learn them. This change is fruitful in representing transformations of time series graphs as illustrated in \Cref{fig:transport}.
% }

% \thi{
%   Our contributions can be summarized as follows:
%     \begin{itemize}
%        \item We propose an unsupervised method (TS-LDDMM) to analyze the inter-individual variability of shapes in a time series dataset. In particular, the method can handle multivariate time series \textit{irregularly sampled} and with \textit{variable sizes}.
%        \item We motivate our extension of LDDMM to time series by introducing a theoretical framework with a representation theorem for time series graph (\Cref{theorem:representation}) and kernels related to their structure (\Cref{lemma:choice_of_kernel_V}).
%        \item We demonstrate the identifiability of the model by estimating the true generating parameter of synthetic data, and we highlight the sensitivity of our method concerning its hyperparameters, also providing guidelines for tuning.
%        \item We highlight the \textit{interpretability} of TS-LDDMM for studying the inter-individual variability in a clinical dataset.
%         \item We illustrate the quantitative interest of such representation on classification tasks on real shape-based datasets.
%  \end{itemize}
% }

\begin{figure}[t]
  \centering
  \includegraphics[width=0.7\linewidth]{"./pictures/diffeo.jpeg"}
  
  \caption{A time series' graph $\msg=\{(t,s(t)): \eqsp t\in\msi\} $ can lose its structure after applying a general diffeomorphism $\phi.\msg$: a time value can be related to two values on the space axis.}
  \label{fig:diffeo}
  
\end{figure}

\begin{figure*}[t]
  \centering
  \includegraphics[width=\linewidth]{"./pictures/transport.jpeg"}
  
  \caption{LDDMM and TS-LDDMM are applied to ECG data.
  We observe that LDDMM, using a general Gaussian kernel, does not learn the time translation of the first spike but changes the space values, i.e., one spike disappears before emerging at a translated position. At the same time, TS-LDDMM handles the time change in the shape.
  This difference of \textit{deformations} implies differences in features \textit{representations}.   }
  \label{fig:transport}
  
\end{figure*}

  \vspace{-1ex}
\section{Notations}
We denote by integer ranges by $[k:l]=\{k,\ldots,l\}\subset \mathcal{P}(\Zset)$ and $ [l]=[1:l]$ with $k,l\in \Nset$,
by $\rmC^m(\msi,\mse)$ the set of $m$-times continously differentiable function defined on an open set $\msu$ to a normed vector space $\mse$,
 by $||u||_\infty=\inf_{x\in \msu} |u(x)| $ for any bounded function $u:\msu \to \mse$,
and by $\Nset_{>0}$ is the set of positive integers. 
%We are in a sub class of curve quotiented by temporal reparametrisation since a time series' graph can be understood as the equivalent class of t ->(t,s(t))

%Reproducing Kernel Hilbert Space (RKHS)
\vspace{-1ex}
\section{Background on LDDMM}
\vspace{-1ex}
\label{section:LDDMM}

In this part, we expose how to
 learn the diffeomorphisms $(\phi_j)_{j\in[N]}$ using LDDMM, initially introduced in \cite{beg2005computing}.
 In a nutshell, for any $j\in [N]$, $\phi_j$ corresponds to a differential flow related to a learnable velocity field belonging to a well-chosen Reproducing Kernel Hilbert Space (RKHS).

 In the next section, time series are going to be represented by diffeomorphism parameters $(\alpha_j)_{j\in[N]}$.
 That is why LDDMM is chosen since it offers a parametrization for diffeomorphisms that is sparse and interpretable, two features particularly relevant in the biomedical context.
 

The basic problem that we consider in this section is the following. Given a set of targets $\mathbf{y}=(y_i)_{i\in[T_2]}$ in $\Rset^{d'}$\footnote{Note that we denote by $d'\in\nset$ the ambient space}, a set of starting points $\mathbf{x}=(x_{i})_{i\in[T_1]}$ in $\Rset^{d'}$, we aim to find a diffeomorphism $\phi$ such that the finite set of points $\mathbf{y}$ is similar in a certain sense to the set of finite sets of transformed points $\phi \cdot \mathbf{x} =(\phi(x_i))_{i\in[T_1]} $.
 The function $\phi$ is occasionally referred to as a \textit{deformation}. In general, these sets $\mathbf{x},\mathbf{y}$ are meshes of continuous objects, e.g., surfaces, curves, images, and so on.

 \vspace{-1ex}
% LDDMM is designed to analyzed an existing dataset, while NF and CNF are made to generalize a dataset for data augmentation.
\paragraph{Representing diffeomorpshims as deformations.}
Such \textit{deformations} $\phi$ are constructed via differential flow equations, for any $x_0\in \Rset^{d'} $ and $\tau\in[0,1]$:
\begin{equation}
  \label{eq:LDDMM_dynamic}
    \frac{\dd X(\tau)}{\dd \tau}= v_\tau(X(\tau)), \quad X(0)=x_0\eqsp ,
    \phi^v_\tau(x_0)=X(\tau), \quad \phi^v=\phi^v_1  \eqsp ,
\end{equation}
where the velocity field is $v:\tau\in [0,1]\mapsto v_\tau\in \msv $
and $\msv$ is a Hilbert space of continuously differentiable function
on $\Rset^{d'}$.  If
$||\dd u ||_{\infty}+|| u ||_{\infty}\leq ||u ||_\msv $ for any
$u\in \msv$ and
$v\in \rml^2([0,1],\msv)=\{v\in \rmC^0([0,1],\msv): \int_0^1 ||
v_\tau||^2_\msv \dd \tau<\infty \} $, by \citep[Theorem 5]{glaunes2005transport}
$\phi^v$ exists and belongs to $\mcd(\Rset^{d'})$, where we denote by $\mcd(\mso) $ the set of diffeomorpshim defined on an open set $\mso$ to $\mso$.
 Therefore, for any choice of $v$, $\phi^v$ defines a valid deformation. 
This offers a general recipe to construct diffeomorphism given a functional space $\msv$.

With this in mind, the velocity field $v$ fitting the data can be
estimated by minimizing 
$v \in \rml^2([0,1],\msv) \mapsto \mathscr{L}(\phi^{v}.\mathbf{x},\mathbf{y})$, where $\mathscr{L}$ is an appropriate loss function.
 However, two computational challenges arise.
  First, this optimization problem is ill-posed, and a penalty term is needed to obtain a unique solution.
   In addition, a parametric family $\msv_{\Theta} \subset \rml^2([0,1],\msv)$, parameterized by $\Theta$, is sought to efficiently solve this minimization problem. 
\paragraph*{From deformations to geodesics.}
%\paragraph*{Regularizing the problem to derive uniqueness.}
It has been proposed in \cite{miller2006geodesic} to interpret $\msv$ as a tangent space relative to the group of diffeomorphisms $\msh=\{ \phi^v:\eqsp v\in \rml^2([0,1],\msv)\}$.
Following this geometric point of view, geodesics can be constructed on $\msh$ by using the following squared norm 
 \begin{equation}
  \label{eq:geodesics_original}
    \mathscr{R}^2: g\in \msh\mapsto \inf_{ v\in \rml^2([0,1],\msv):\eqsp g=\phi^v} \int_0^1 || v_\tau||_\msv\dd \tau
 \end{equation}
By deriving differential constraints related to the minimum of \eqref{eq:geodesics_original} and using Cauchy-Lipschitz conditions, geodesics can be defined only by giving the starting point and the initial velocity $v_0\in \msv$ \cite{miller2006geodesic}, as straight lines in Euclidean space.
Denoting by $\tau \mapsto \rho_{v_0}(\tau)\in\msh$ the geodesic starting from the identity with inital velocity $v_0\in \msv$, the exponential map is defined as $\varphi^{\{v_0\}}\triangleq \rho_{v_0}(1)$. Using $\varphi^{\{v_0\}}$ instead of $\phi^v$, the previous matching problem becomes a \textit{geodesic shooting problem}:
 \begin{equation}
  \label{eq:geodesics_shooting}
  \inf_{v_0 \in \msv} \mathscr{L}(\varphi^{\{v_0\}}.\mathbf{x},\mathbf{y}).
 \end{equation}
 Using $\varphi^{\{v_0\}}$ instead of $\phi^v$ for any $v\in \rml^2([0,1],\msv)$ regularizes the problem and induces a sparse representation for the learning diffeomorphisms.
 Moreover, by setting $\msv$ as an RKHS, the geodesic shooting problem has a unique solution and becomes tractable, as described in the next section.




% More precisely, on the group of diffeomorphisms $\msh=\{ \phi^v:\eqsp v\in \rml^2([0,1],\msv)\}$,
%  the following squared norm can be defined
%  \begin{equation}
%   \label{eq:geodesics_original}
%     \mathscr{R}^2: g\in \msh\mapsto \inf_{ v\in \rml^2([0,1],\msv):\eqsp g=\phi^v} \int_0^1 || v_\tau||_\msv^2\dd \tau
%  \end{equation}
%      as the minimal "energy" needed to perform the deformation $g$.
% By \citep[Theorem 6]{glaunes2005transport}, there exists $v^*\in \rml^2([0,1],\msv)$ such that the previous infimum is a minimum in $v^*$ such that $(\phi^{v^*}_\tau)_{\tau\in[0,1]}$ can be understood as the geodesic between the identity function and $g$.
% However, given a diffeomorphism $g$, computing $\mathscr{R}(g)$ is intractable in most cases. 
% To circumvent this issue, another characterization of geodesics can be considered.
%  As in Riemannian geometry, 
%  instead of defining geodesics from their starting and end points, it is possible to define them from their starting point (here, the identity function) and an initial velocity $v_0\in\msv$.
%  \sam{To CHANGE}
%  %In other words, an Exponential map on deformations is wanted:
% given $v_0\in\msv$, it has been suggested to generate diffeomorphisms as $\varphi^{\{v_0\}}=\phi^v$ with
% %  \begin{equation}
% %   \label{eq:geodesics_shooting}
% %   v=\underset{w\in \rml^2([0,1],\msv):\eqsp w_0=v_0}{\argmin} \int_0^1 || w_\tau||_\msv^2\dd \tau\eqsp ,
% %  \end{equation}
% % since with this definition it holds $\mathscr{R}^2(\phi^v)= \int_0^1 || v_\tau||_\msv^2\dd \tau $.
%  By setting $\msv$ as an RKHS, the geodesic shooting problem \eqref{eq:geodesics_shooting} has a unique solution and becomes tractable, as described in the next section.
% Moreover, we have $\mathscr{R}(\varphi^{\{v_0\}})=||v_0||_\msv$ such that we can work with $v_0\in \msv $ instead of $v\in \rml^2([0,1],\msv)$.
%  Therefore, compared to CNF, only $v_0$ is parametrized by using the geodesic structure and not each $v_\tau$ for any $\tau\in[0,1]$.
%  Moreover, we have $\mathscr{R}(\phi^{\{v_0\}})=||v_0||_\msv$ such that we can work with $v_0\in \msv $ instead of $v\in L^2([0,1],\msv)$.
%  %Therefore, compared to CNF, only $v_0$ is parametrized by using the geodesic structure and not each $v_\tau$ for any $\tau\in[0,1]$.
%  This is presented with more details in the next paragraph.




\vspace{-1ex}
\paragraph{Discrete parametrization of diffeomorpshim.}

In this part, $\msv$ is chosen as an RKHS \cite{berlinet2011reproducing} generated by a smooth kernel $K$ (e.g., Gaussian). 
We follow \cite{durrleman2013sparse} and define a 
 discrete parameterization of the velocity fields to perform geodesics shooting \eqref{eq:geodesics_shooting}.  
  The initial velocity field $v_0$ is chosen as a finite linear combination of the RKHS basis vector fields, 
$\mathbf{n}_0$ control points $\msx_0=(x_{k,0})_{k\in[\mathbf{n}_0]}\in (\Rset^{d'})^{\mathbf{n}_0}$ and momentum vectors $\alpha_0=(\alpha_{k,0})_{k\in[\mathbf{n}_0]}\in (\Rset^{d'})^{\mathbf{n}_0} $ are defined such that for any $x\in \Rset^{d'}$, 
  \begin{equation}
    \label{eq:def_v0}
    v_0\left(\alpha_0,\msx_0\right)(x)=\sum_{k=1}^{\mathbf{n}_0} K(x,x_{k,0})\alpha_{k,0} \eqsp .
  \end{equation}
   In our applications, the control points $(x_{k,0})_{k\in[\mathbf{n}_0]}$ can be understood as the discretized graph $(t_k,\mathbf{s}_0(t_k))_{k\in[\mathbf{n}_0]}$ of a starting time series $\mathbf{s}_0$. 
  With this parametrization of $v_0$, \cite{miller2006geodesic} show that the velocity field $v$ of the solution of \eqref{eq:geodesics_shooting} keeps the same
  structure along time, such that for any $x\in\Rset^{d'}$ and $\tau\in[0,1]$, 
  \begin{equation}
    \label{eq:specific_form}
    v_\tau(x)=\sum_{k=1}^{\mathbf{n}_0} K(x,x_k(\tau))\alpha_{k}(\tau) \eqsp ,
  \end{equation}
  \begin{equation} 
    \label{eq:integration}
      \left\{
        \begin{aligned}
        & \frac{\dd x_k(\tau)}{\dd \tau}=v_\tau(x_k(\tau)) \eqsp, \quad
        \frac{\dd \alpha_k(\tau)}{\dd \tau}=-\sum_{k=1}^{\mathbf{n}_0} \dd_{x_k(\tau)} K(x_k(\tau),x_l(\tau))\alpha_{l}(\tau)^\top \alpha_{k}(\tau) \eqsp  \\
        & \alpha_k(0)=\alpha_{k,0},\quad x_k(0)=x_{k,0} \eqsp , k\in[\mathbf{n}_0] 
        \end{aligned}
        \right .
  \end{equation}
  These equations are derived from the hamiltonian $H:(\alpha_k,x_k)_{k\in [\mathbf{n}_0]}\mapsto \sum_{k,l=1}^{\mathbf{n}_0} \alpha_{k}^\top K(x_k,x_l)\alpha_{l}  $, such that
  the velocity norm is preserved $||v_\tau||_{\msv}=||v_0||_\msv $ for any $\tau\in [0,1]$.
   By \eqref{eq:integration}, the velocity field related to a geodesic $v^*$ is fully parametrized by its initial control points and momentum $(x_{k,0},\alpha_{k,0})_{k\in[\mathbf{n}_0]}$.
   Thus, given a set of targets $\mathbf{y}=(y_i)_{i\in[T_2]}$ in $\Rset^{d'}$, a set of starting points $\mathbf{x}=(x_{i,0})_{i\in[T_1]}$ in $\Rset^{d'}$, a RKHS's kernel $K:\Rset^{d'}\times \Rset^{d'}\to \Rset^{d'\times d'}$, a distance on sets $\mathscr{L}$, 
 a numerical integration scheme of ODE and a penalty factor $\lambda>0$, the basic geodesic shooting step minimizes the following function using a gradient descent method:
   \begin{equation}
    \label{eq:relaxation}
    \mathcal{F}_{\mathbf{x},\mathbf{y}}: (\alpha_k)_{k\in [T_1]}\mapsto \mathscr{L}\left(\varphi^{\{v_0\}}.\mathbf{x},\mathbf{y}\right)+\lambda||v_0||_\msv^2 \eqsp,  
   \end{equation}
   where $v_0$ is defined by \eqref{eq:def_v0} and $\varphi^{\{v_0\}}.\mathbf{x}$ is the result of the numerical integration of \eqref{eq:integration} using control points $\mathbf{x}$ and initial momentums $(\alpha_k)_{k\in[T_1]} $. 



   %Before to specify the choice of the RKHS kernel $K$ and the loss on sets $\mathscr{L}$ in \Cref{section:time_series_specificity}, we first discuss the connection of LDDMM with normalizing flows. % to highlight the interest of the LDDMM approach.


  
   
%Conversely to NF or CNF, by \eqref{eq:integration}, different momentums parameters raise different final deformations $\phi^{\{v_0\}}$, this injectivity properties offers interpretability of the representation and gives sens to PCA or LDA post-processing as exposed in \Cref{section:experiments}.
  
   %If a diffeomorpshim $\Phi$ is given and we want to represent it as a deformation $\phi^v$ with $v$ of minimal ernergy, then
  %$(\alpha_{k,0})_k$ is the only parameters to learn since $(x_{k,0})_k$ is fixed.
  


%   \begin{algorithm}[tb]
%     \caption{Geodesic shooting with LDDMM}
%     \label{alg:example}
%  \begin{algorithmic}
%     \STATE {\bfseries Input:} a RKHS's kernel $K$, a loss on sets $L$, $\mathbf{x}=(x_{i,0})_{i\in[T_1]}$, $\mathbf{y}=(y_i)_{i\in[T_2]}$ two sets of $\Rset^{d'}$, initial momentums 
%     $(\alpha_{k,0})_{k\in [T_1]} \in (\Rset^{d'})^{T_1}$ and a number of gradient iterations $n_{\text{iter}}$.

%     \STATE $f: (\alpha_k)_k\mapsto L\left(\phi^{\{v_0\}}.\mathbf{x},\mathbf{y}\right)+\lambda||v_0||_\msv^2  $ with $v_0$ and $\phi^{\{v_0\}}$
%      respectively defined by \eqref{eq:def_v0} and \eqref{eq:integration} using control points $\mathbf{x}$ and initial momentums $(\alpha_k)_k $. 
%      \sam{Peut être à définir dans un autre algo ?}
%     \STATE $\alpha_{k,0}^*\gets \alpha_{k,0}, \eqsp k\in[T_1]$
%     \FOR{$i=1$ {\bfseries to} $n_{\text{iter}}$}
%     \STATE $\alpha_{k,0}^*\gets \text{AdamStep}(f,\alpha_{k,0}^*$) , where $\text{AdamStep}$ is given in appendix (mettre ref)
%     \ENDFOR
%     \STATE {\bfseries Return:} $(\alpha_{k,0}^*)_{k\in [T_1]}$
    
%  \end{algorithmic}
%  \end{algorithm}
\vspace{-1ex}
\paragraph{Relation to Continuous Normalizing Flows.}

One particular popular choice to address the problem of learning a diffeomorphism or a velocity field is Normalizing Flows \cite{rezende2015variational,kobyzev2020normalizing} (NF) or their continuous counterpart \cite{chen2018neural,grathwohl2019scalable,salman2018deep} (CNF).
However, we do not rely on this class of learning algorithms for several reasons. Indeed, existing and simple normalizing flows are not suitable for the type of data that we are interested in this paper \cite{feng2023multi,deng2020modeling}. 
  In addition,  they are primarily designed to have tractable Jacobian functions, while we do not require such property in our applications. 
Finally, the use of a differential flow solution of an ODE
\eqref{eq:LDDMM_dynamic} trick is also at the basis of CNF, which
then consists of learning a velocity field to address in fitting the data through a loss aiming to address the problem
at hand. Nevertheless, the main difference between CNF and LDDMM lies in the
parametrization of the velocity field. LDDMM uses kernels to
derive closed form formula and enhance interpretability while
NF and CNF take advantage of deep neural networks to scale with
large dataset in high dimensions.
\vspace{-1ex}
\section{Methodology}
\label{section:methodology}
%\paragraph{Assumptions}
%Denoting by $C^0(\msi,\msj) $ the space of continous functions between two sets $\msi,\msj$,
\vspace{-1ex}
\sam{Lien avec functional data analysis ? Insister sur le fait que concaténer temps et espace c'est moisi}
We consider in this paper observations which consist in a population of $N$ multivariate time series, for any $j\in[N]$, $s^j \in \rmC^1(\msi_j,\Rset^{d})$. 
However, we can only access a $n_j$-samples $\tsig^j=(\tsig_i^j=s^j(t^j_i))_{i\in[n_j]}$ collected at timestamps $(t^j_i)_{i\in[n_j]}$ for any $j \in [N]$.
 Note that \textbf{the number of samples $n_j$ is not necessarily the same across individuals} and the timestamps can be \textbf{irregularly sampled}.
 We assume the time series population is globally homogeneous regarding their "shapes" even if inter-individual variability exists.
 Intuitively speaking, the "shape" of a time series $s:\msi\to \Rset^d$ is encoded in its graphs $\msg(s)$ defined as the set $\{(t,s(t)):\eqsp t\in\msi \} $ and not only in its values $s(\msi)=\{s(t):\eqsp t\in\msi \} $ since the time axis is crucial. % on the shape as illustrated in Figure\alain{ (mettre ref)}.
 As a motivating use-case, $s^j$ can be the time series of a heartbeat extracted from an individual's electrocardiogram (ECG), see \Cref{fig:transport}.
 The homogeneity in a resulting dataset comes from the fact that humans have similar shapes of heartbeat \cite{ye2012heartbeat,madona2021pqrst}.
 % the shape is also referred as morphology of the time series. 
  %  [More specifically, the heterogeneity in such dataset is often due to local or glocal dilatation of the time series regarding time, the heartbeat is quicker or slower depending on the individual.
  %  In maths, denoting by $\mcd(\msi,\msj) $ the set of diffeomorpshim defined on the sets $\msi$ to $\msj$,
  %   there exist a common pattern $\mathbf{s}_0: \msi \to \Rset^d$ and individual time reparametrisations $ \psi_j\in\mcd(\msi,\msi_j) $ such that,
  %   for any $j\in[N]$, $s^j=\mathbf{s}_0\circ \psi_j^{-1} $.]
%  \paragraph{Goals}%\sam{potentiellement changer l'ordre de time and space}
\vspace{-1ex}
\paragraph*{The deformation problem.}
In this paper, we aim to study the inter-individual variability in the dataset by finding a relevant representation of each time series.
Inspired from the framework of shape analysis \cite{vaillant2004statistics}, addressing similar problems in morphology,
 we suggest to represent each time series' graph $\msg(s^j)$ as the transformation of a reference graph $\msg(\mathbf{s}_0)$, related to a time series $\mathbf{s}_0:\msi \to\Rset^d$, by a diffeomorphism $\phi_j$ on $\Rset^{d+1}$, for any $j\in[N]$,
\begin{equation}
 \label{eq:transformation}
 \phi_j.\msg(\mathbf{s}_0)=\{\phi_j\left(t,\mathbf{s}_0(t)\right), \eqsp t\in \msi \} \eqsp.
\end{equation}
$\bfs_0$ will be understood as the typical representative shape common to the collection of time series $(s^j)_{j\in[N]}$.
As $\bfs_0$ is supposed to be fixed, then the representation of the time series $(s^j)_{j\in[N]}$ boils down to the one of the transformation $(\phi_j)_{j\in[N]}$.
We aim to learn $\msg(\bfs_0)$ and $(\phi_j)_{j\in[N]} $. 

%First, we introduce the Large Deformation Diffeomorphic Metric Mapping (LDDMM) framework.
% Then, we explain how to learn a discretization of the graph $\msg(\bfs_0)$ and the diffeomorphisms $(\phi_j)_{j\in[N]} $ by using LDDMM and a gradient descent minimization.
  %Finally, we tackle the specificity of graph time series by deriving a representation theorem on diffeomorphisms, which enables us to select the kernel needed in LDDMM, and thus, we propose TS-LDDMM. 
 
 
     %\sam{Parler de l'unicté?}
     \vspace{-1ex}
     \paragraph{Optimization related to \eqref{eq:transformation}.}
     Defining the \textit{discretized graphs} of the time series $(s^j)_{j\in[N]}$ and a discretization of the reference graph $\msg(\mathbf{s}_0)$ as, for any $j\in[N]$,
 \begin{equation}
 \label{eq:descretized_graph}
 \mathbf{y}_j=\msg(\tsig^j)=(t_i^j,\tsig^j_i)_{ i\in[n_j]}\in (\Rset^{d+1})^{n_j},\quad \tilde{\msg}_0=(t_i^0,\tsig^0_i)_{i\in[\mathbf{n}_0]}\in (\Rset^{d+1})^{\mathbf{n}_0} \eqsp ,
 \end{equation}
 with $\mathbf{n}_0=\operatorname{median}((n_j)_{j\in[N]})$, the representation problem given in \eqref{eq:transformation} boils down solving:
 \begin{equation}
 \label{eq:general_optimization_problem}
 \argmin_{\tilde{\msg}_0,(\alpha_k^j)_{k\in [\mathbf{n}_0]}^{j\in[N]}} \sum_{j=1}^N \mathcal{F}_{\tilde{\msg}_0,\mathbf{y}_j}\left((\alpha_k^j)_{k\in [\mathbf{n}_0]}\right)\eqsp ,
 \end{equation}
 which is carried out by gradient descent on the control points $\tilde{\msg}_0$ and the momentums $\mathbf{\alpha}_j=(\alpha_k^j)_{k\in [\mathbf{n}_0]}$ for any $j\in[N]$, initialized by a dataset's time series graph of size $\mathbf{n}_0$ and by $0_{(d+1)\mathbf{n}_0}$ respectively.
 The optimization hyperparameter details are given in \Cref{appendix:optimizers_details}.
 The result of the minimization $\tilde{\msg}_0$ is then considered as the $\mathbf{n}_0$-samples of a common time series $\mathbf{s}_0$ and the momentums $\mathbf{\alpha}_j$ encoding $\phi_j$ yields a feature vector in $\Rset^{d \mathbf{n}_0} $ of $s^j$ for any $j\in[N]$.
 Finally, the vectors $(\mathbf{\alpha}_j)_{j\in[N]}$ can be analyzed with any statistical or machine learning tools such as Principal Components Analysis (PCA), Latent Discriminant Analysis (LDA), longitudinal data analysis and so on.
%As $\bfs_0$ is supposed to be fixed, then the representation of the time series $(s_j)_{j\in[N]}$ boils down to the one of the transformation $(\phi_j)_{j\in[N]}$.
 %In \Cref{section:optimization} the choice of the common time series $\mathbf{s}_0$ will be optimized to have $\phi_j$ of minimal norm. %T is given in the following part.

%  To learn the diffeomorphisms $(\phi_j)_{j\in[N]}$, we propose to use in this paper Large Deformation Diffeomorphic Metric Mapping (LDDMM).
%   In a nutshell, for any $j\in [N]$, $\phi_j$ corresponds to a differential flow related to a learnable velocity field belonging to a well-chosen Reproducing Kernel Hilbert Space (RKHS).
% We now describe more precisely LDDMM and in a next step the procedure to learn $\mathbf{s}_0$.

% \sam{to improve ?}
% Compared to the litterature of Unsupervised Representation Learning (URL) \cite{meng2023unsupervised}, the homogeneity assumption can be quite restrictive, however this is quite common in Shape Analysis \cite{vaillant2004statistics} (SA).
%     While in URL the goal is to find features accounting for the clustering structure in the heterogene data and to transfer this representation for different ML tasks, in SA the goal is to analyze the inter-individual variability with precision to derive clinical conclusions.
%     The generalization of the method presented in this paper to heterogene dataset is possible, but relagated for future works.
  %These velocity fields are then learned by minimizing the norm of this RKHS.

%In all the following, we denote by $\mcd(\mso) $ the set of diffeomorpshim defined on an open set $\mso$ to $\mso$.
%\sam{Transformer , en : dans les ensembles}

   %\sam{Parler de l'unicté?}


   %Before to specify the choice of the RKHS kernel $K$ and the loss on sets $\mathscr{L}$ in \Cref{section:time_series_specificity}, we first discuss the connection of LDDMM with normalizing flows. % to highlight the interest of the LDDMM approach.


  
   
%Conversely to NF or CNF, by \eqref{eq:integration}, different momentums parameters raise different final deformations $\phi^{\{v_0\}}$, this injectivity properties offers interpretability of the representation and gives sens to PCA or LDA post-processing as exposed in \Cref{section:experiments}.
  
   %If a diffeomorpshim $\Phi$ is given and we want to represent it as a deformation $\phi^v$ with $v$ of minimal ernergy, then
  %$(\alpha_{k,0})_k$ is the only parameters to learn since $(x_{k,0})_k$ is fixed.
  


%   \begin{algorithm}[tb]
%     \caption{Geodesic shooting with LDDMM}
%     \label{alg:example}
%  \begin{algorithmic}
%     \STATE {\bfseries Input:} a RKHS's kernel $K$, a loss on sets $L$, $\mathbf{x}=(x_{i,0})_{i\in[T_1]}$, $\mathbf{y}=(y_i)_{i\in[T_2]}$ two sets of $\Rset^{d'}$, initial momentums 
%     $(\alpha_{k,0})_{k\in [T_1]} \in (\Rset^{d'})^{T_1}$ and a number of gradient iterations $n_{\text{iter}}$.

%     \STATE $f: (\alpha_k)_k\mapsto L\left(\phi^{\{v_0\}}.\mathbf{x},\mathbf{y}\right)+\lambda||v_0||_\msv^2  $ with $v_0$ and $\phi^{\{v_0\}}$
%      respectively defined by \eqref{eq:def_v0} and \eqref{eq:integration} using control points $\mathbf{x}$ and initial momentums $(\alpha_k)_k $. 
%      \sam{Peut être à définir dans un autre algo ?}
%     \STATE $\alpha_{k,0}^*\gets \alpha_{k,0}, \eqsp k\in[T_1]$
%     \FOR{$i=1$ {\bfseries to} $n_{\text{iter}}$}
%     \STATE $\alpha_{k,0}^*\gets \text{AdamStep}(f,\alpha_{k,0}^*$) , where $\text{AdamStep}$ is given in appendix (mettre ref)
%     \ENDFOR
%     \STATE {\bfseries Return:} $(\alpha_{k,0}^*)_{k\in [T_1]}$
    
%  \end{algorithmic}
%  \end{algorithm}

Nevertheless, \eqref{eq:general_optimization_problem} asks to define a kernel and a loss in order to perform geodesic shooting \eqref{eq:relaxation}, which is the purpose of the following subsection.
\vspace{-1ex}
   \subsection{Application of LDDMM to time series analysis: TS-LDDMM}
   \vspace{-1ex}
%\tcr{Mettre l'applicaiton clair pour les times series}

        \label{section:time_series_specificity}
        This section presents our theoretical contribution: we tailor the LDDMM framework to handle time series data.
        The reason is that applying a general diffeomorphism $\phi$ from $\Rset^{d+1}$ to a time series' graph $\msg(s)$ can result in a set $\phi.\msg(s)$ that does not correspond to the graph of any time series, as illustrated in the \Cref{fig:diffeo}.
        Thus, time series graphs have more structure than a simple 1D curve \cite{glaunes2008large} and deserve their unique analysis, which will prove fruitful as demonstrated in \Cref{section:experiments}.
       
       To address this challenge, we need to identify an RKHS kernel $K:\Rset^{d+1}\times \Rset^{d+1}\to \Rset^{(d+1)^2}$ that generates deformations preserving the structure of the time series graph. This goal motivates us to clarify, in \Cref{theorem:representation}, the specific representation of diffeomorphisms we require before presenting a class of kernels that produce deformations with this representation.
       
       Similarly, selecting a loss function on sets $\mathscr{L}$ that considers the temporal evolution in a time series' graph is crucial for meaningful comparisons with time series data. Consequently, we introduce the oriented Varifold distance. 
\vspace{-1ex}
       \paragraph{A representation separating space and time.}
       We prove that two time series graphs can always be linked by a time transformation composed with a space transformation. Moreover, a time series graph transformed by this kind of transformation is always a time series graph.
        We define $\Psi_\gamma\in \mcd(\Rset^{d+1}) : (t,x)\in\Rset^{d+1}\to (\gamma(t),x)$ for any $\gamma\in \mcd(\Rset)$ and $\Phi_f:  (t,x)\in\Rset^{d+1}\to (t,f(t,x)) $ for any $f\in \rmC^1(\Rset^{d+1},\Rset^d)$. 
        We have the following representation theorem.
        All proofs are given in \Cref{appendix:proofs}. 


  Denote by $\msg(s)\triangleq \{ (t,s(t)): \eqsp t\in \msi \} $ the graph of a time series $s: \msi \to \Rset^d$ and $ \phi.\msg(s)\triangleq\{ \phi(t,s(t)): \eqsp t\in \msi\} $ the action of  $\phi\in \mcd(\Rset^{d+1}) $ on $\msg(s)$.
    \begin{theorem}
        \label{theorem:representation}
    Let $s:  \msj \to \Rset^d  $ and $\mathbf{s}_0: \msi\to \Rset^d $ be two continuously differentiable time seriess with $\msi,\msj$ two intervals of $\Rset$.
     There exist $f\in \rmC^1(\Rset^{d+1},\Rset^d)$ and $\gamma\in  \mcd(\Rset) $ such that $\gamma(\msi)=\msj $ and $\Phi_f\in \mcd(\Rset^{d+1})$,
     \begin{equation}
        \msg(s)= \Pi_{\gamma,f}.\msg(\mathbf{s}_0),\eqsp \Pi_{\gamma,f}=\Psi_\gamma\circ\Phi_f.
     \end{equation}
     Moreover, for any $\bar{f}\in \rmC^1(\Rset^{d+1},\Rset^d)$ and $\bar{\gamma}\in  \mcd(\Rset) $, there exists a continously differentiable time series $\bar{s}$ such that 
     $\msg(\bar{s})= \Pi_{\bar{\gamma},\bar{f}}.\msg(\mathbf{s}_0)$
    \end{theorem}
    %\sam{preuve todo, classement des variétés à bord, donne un homeomorphisme sur la restriction}
    % \begin{proof}
    %   Let $s:  \msj \to \Rset^d  $ and $\mathbf{s}_0: \msi\to \Rset^d $ be two continuously differentiable time seriess with $\msi=(a,b),\msj=(\alpha,\beta)$ two intervals of $\Rset$.
    %   % $\msg(\mathbf{s}_0)$ and $\msg(s)$ are two smooth connected 1-dimensional manifold of $\Rset^{d+1}$, thus, by \cite[Appendix
    %   % Classfying 1-Manifolds]{milnor1965topology}, there exists $\Pi\in \mcd(\Rset^{d+1})$ such that $\msg(s)=\Pi.\msg(\mathbf{s}_0)$.
    %   %  Therefore, denoting by $\Pi_1: \Rset^{d+1}\mapsto \Rset$, we define $\gamma(t)\triangleq \Pi_1(t,\mathbf{s}_0(t))$ for any $t\in \msi$, then for any $t\in [b,+\infty)$,
    %   %   \begin{equation}
    %   %     \gamma(t)=\\Pi_1(b,\mathbf{s}_0(b))+(t-b)\frac{\dd\gamma}{\dd^- t}(b) \eqsp,
    %   %   \end{equation}
    %   %   and for any $t\in (-\infty,a]$,
    %   %   \begin{equation}
    %   %      \gamma(t)=\Pi_1(a,\mathbf{s}_0(a))+(t-a)\frac{\dd\gamma}{\dd^+ t}(a)\eqsp ,
    %   %   \end{equation}
    %   %    where $ \frac{\dd\gamma}{\dd^- t},\frac{\dd\gamma}{\dd^+ t}$ are respectively the left and right derivative.
        
    %   By setting $\gamma: t\in \Rset \mapsto (\beta-\alpha)(t-a)/(b-a)+\alpha\in \Rset $, we have $ \gamma(\msi)=\msj$ and $\gamma \in \mcd(\Rset) $.
    %    By defining $f:(t,x)\in\Rset^{d+1}\mapsto x-\mathbf{s}_0(t)+s\circ \gamma(t) $, the map $\Phi_f\in \mcd(\Rset^{d+1})$,
    %     indeed, its inverse is $\Phi_f^{-1}:(t,x)\in\Rset^{d+1}\mapsto (t,x+\mathbf{s}_0(t)-s(t)) $ and is continuously differentiable.
    %      Moreover, we have $\Pi_{\gamma,f}.\msg(\mathbf{s}_0)=\{(\gamma(t),s\circ \gamma(t)):\eqsp t\in\msi \}=\msg(s) $.

        

    %     Let $\bar{f}\in \rmC^0(\Rset^{d+1},\Rset^d)$, $\bar{\gamma}\in  \mcd(\Rset) $ and $\mathbf{s}_0\in \rmC^0(\msi,\Rset^d)$ with $\msi$ an interval of $\Rset$.
    %     We have :
    %     \begin{align}
    %       \Pi_{\gamma,f}.\msg(\mathbf{s}_0)&=\{(\gamma(t),f(t,\mathbf{s}_0(t))),\eqsp t\in \msi \} \\
    %       &\label{eq:proof1_last_eq}=\{(t,f\left(\gamma^{-1}(t),\mathbf{s}_0(\gamma^{-1}(t))\right),\eqsp t\in \gamma(\msi) \} \eqsp .
    %     \end{align}
    %     By defining $\bar{s}:t\in \gamma(\msi)\to f\left(\gamma^{-1}(t),\mathbf{s}_0(\gamma^{-1}(t))\right) $, we have $\bar{s}\in \rmC^0(\gamma(\msi), \Rset^d) $ by composition of continuous functions
    %     and $ \msg(\bar{s})= \Pi_{\gamma,f}.\msg(\mathbf{s}_0)$ by \eqref{eq:proof1_last_eq}, which concludes the proof.
    % \end{proof}
    \begin{remark}
      Note that for any $\gamma \in \mcd(\Rset) $ and $s\in \rmC^0(\msi,\Rset^d)$,
   \begin{equation}
    \{(\gamma(t),s(t)),\eqsp t\in \msi \}=\{(t,s\circ \gamma^{-1}(t)):\eqsp t\in\gamma(\msi) \}\eqsp .
   \end{equation}
   As a result, $\Psi_\gamma $ can be understood as a temporal reparametrization and $\Phi_f$ encodes the transformation about the space.
 \end{remark}

 \vspace{-1ex}
    \paragraph{Choice for the kernel associated with the RKHS $\msv$}
    \label{paragraph:kernel_V}
    As depicted on \Cref{fig:diffeo}-\ref{fig:transport}, we can not use any kernel $K$ to apply the previous methodology to learn deformations on time series' graphs.
    We describe and motivate our choice in this paragraph.
     Denote the one-dimensional Gaussian kernel by $K_\sigma^{(a)}(x,y)=\exp(-|x-y|^2/\sigma)$ for any $(x,y)\in (\Rset^a)^2$, $a\in \Nset$ and $\sigma>0$.
    To solve the geodesic shooting problem \eqref{eq:relaxation} on $\Rset^{d+1}$, we consider for $\msv$ the RKHS associated with the kernel defined for any $(t,x),(t',x')\in (\Rset^{d+1})^2$:
    \begin{align}
      \label{eq:kernel_TAS}
      &K_{\msg}((t,x),(t',x'))=\begin{pmatrix}
        c_0K_{\text{time}} & 0 \\
        0 & c_1 K_{\text{space}} 
        \end{pmatrix} \eqsp ,\\
       & K_{\text{space}}=K_{\sigma_{T,1}}^{(1)}(t,t')K_{\sigma_x}^{(d)}(x,x') \Idd\eqsp,K_{\text{time}}=K_{\sigma_{T,0}}^{(1)}(t,t') \eqsp,
    \end{align}
    parametrized by the widths $\sigma_{T,0},\sigma_{T,1},\sigma_x>0$ and the constants $c_0,c_1>0$.
This choice for $K_\msg$ is motivated by the representation \Cref{theorem:representation} and the following result. 
    \begin{lemma}
      \label{lemma:choice_of_kernel_V}
      If we denote by $\msv$ the RKHS associated with the kernel $K_{\msg}$, then for any vector field $v$ generated by \eqref{eq:integration} with $v_0$ satisfying \eqref{eq:def_v0},
       there exist $\gamma \in \msd(\Rset) $ and $f\in \rmC^1(\Rset^{d+1},\Rset^d)$ such that $\phi^v=\Psi_\gamma\circ\Phi_f $.
    \end{lemma}
    Instead of Gaussian kernels, other types of smooth kernels can be selected as long as the structure \eqref{eq:kernel_TAS} is respected.
    % \begin{proof}
    %   Let $v$ be a vector field generated by \eqref{eq:integration} with $v_0$ satisfying \eqref{eq:def_v0}.
    %  We remark that the first coordinate of the velocity field $v_\tau$ denoted by $v_\tau^{\text{time}}$ only depends on the time variable $t$ for any $\tau\in[0,1]$.
    %  Thus, when computing the first coordinate of the deformation $\phi^v$, denoted by $\gamma$, we integrate \eqref{eq:LDDMM_dynamic} with $v_\tau$ replaced by $v_\tau^{\text{time}}$,
    %   thus $\gamma$ is independant of the variable $x$. Moreover, $\gamma\in \mcd(\Rset)$ since a Gaussian kernel induced an Hilbert space $\msv$ satisfying $|f|_V\leq |f|_\infty+ |\dd f|_\infty  $ for any $f\in \msv$ by \citep[Theorem 9]{glaunes2005transport}.
    %   For the same reason, we have $\phi^v\in \mcd(\Rset^{d+1})$, and thus its last coordinates denoted by $f$ belongs to $\rmC^1(\Rset^{d+1},\Rset^d)$, and by construction $\phi^v=\Psi_\gamma\circ\Phi_f $.
    % \end{proof
    \begin{remark}
      With this choice of kernel, the features associated with the time transformation can be extracted from the momentums $(\alpha_{k,0})_{k\in[\mathbf{n}_0]}\in (\Rset^{d+1})^{\mathbf{n}_0}$ in \eqref{eq:def_v0} by taking the coordinates related to time.
      However, the features related to the space transformation are not only in the space coordinates since the related kernel $K_{\text{space}}$ depends on time as well.
      \end{remark}
      In \Cref{appendix:kernel_TS_LDDMM}, we give guidelines for selecting the hyperparameters $(\sigma_{T,0},\sigma_{T,1},\sigma_x,c_0,c_1)$.
  
  
    

    %  and then we can perform a statistical analysis on the representations $(\psi_j,\phi^j)_{j\in[N]}$,
    % such as applying PCA decomposition \cite{vaillant2004statistics}, carring out longitudinal analysis \cite{guigui2022parallel} and so on (\sam{mettre ref}).
    % % Cette hypothèse intéressante car ça permet de gérer des signaux de taille variables, tout les signaux se repréentent comme ça 
    % % Espace homogène, injectivité de t ,s(t), permet de garder la structure temporel en compte
    % \sam{Can we show unicity by assuming that $\phi,\psi$ minimize a norm ?}
 % KEOPS tp pour lddmm




%and their empiric samples version $\tilde{G}_j=(\{(t_i,s^j_i), \eqsp i\in [n_i]\})_j $

%  We translate the notation from the continous to discrete by adding a tilde :
%   $\tilde{G}^j=\triangleq \{ (t_i,\tilde{s}_i^j),\eqsp i\in[n_i] \}$.

% \paragraph{A problem of minimization}
% In the next section, we will present that diffeomorpshim on space an time can be parametrized as $\phi^{v^S},\phi^{v^T}$
%  with parameters $(v^S,v^T) \in \Theta_S\times \Theta_T$. Then, pattern will be represented by the parameters $v^S,v^T$ solutions of the following minimization :
% \begin{align}
%   \label{eq:minimization}
%   &\argmin_{v^S \in \Theta_S,\eqsp v^T\in \Theta_T } \mathbf{D}(v^S,v^T)+\lambda \mathbf{N}(v^S,v^T)\eqsp , \\
%   & \mathbf{D}(v^S,v^T)=L(\phi^{v^S}.G^\msj(\mathbf{s_0}\circ (\psi^{v^T})^{-1}),G^\msj(s)) \eqsp , 
% \end{align} 
% where $L$ is a loss on graphs and $\mathbf{N}$ is a term of regularization.

 


% $s_i: t\in[0,T_0]\mapsto s(t)\in \Rset^d $ is only known through its samples $(S_i^j(t_i)=s_i)_{i\in[N]} $ at timestamps $(t_i)_{i\in[T]}$.

%\sam{améliorer transition}
\vspace{-1ex}
\paragraph{Loss}
\sam{Simplifier cette partie ?}
This section specifies the distance function $\scrl$ introduced in the loss function defined in \eqref{eq:relaxation}. 

In practice, we can only access discretized graphs of time series, $(t_i^j,\tsig^j_i)_{i\in[n_j]}$ for any $j\in[N]$, that are potentially of 
different sizes $n_j$ and sampled at different timestamps $(t_i^j)_{i\in[n_j]}$ for any $j\in[N]$.
 Usual metrics, such as the Euclidean distance, are not appealing as they 
make the underlying assumptions of equal size sets and the existence of a pairing between points.
 Distances between measures on 
sets (taking the empirical distribution), such as Maximum Mean Discaprency (MMD) \cite{dziugaite2015training,borgwardt2006integrating}, alleviate those issues; however, MMD only accounts for positional information 
and lacks information about the time evolution between sampled points.
 A classical data fidelity metric from shape analysis 
corresponding to the distance between \textit{oriented varifolds} associated with curves alleviates this last issue \cite{kaltenmark2017general}.  
Intuitively, an oriented varifold is a measure that accounts for positional and tangential information about the underlying 
curves at sample points. More details and information about \textit{oriented varifolds} can be found in \Cref{appendix:varifold}. 

More precisely, given two sets $\msg_0=(g_i^0)_{i\in[T_0]},\msg_1=(g_i^1)_{i\in[T_1]}\in (\Rset^{d+1})^{T_1}$ and a kernel\footnote{$\mathbb{S}^d=\{x\in\Rset^{d+1}:\eqsp |x|=1\}$} $k:(\Rset^{d+1} \times \mathbb{S}^d)^2\to \Rset$
verifying \citep[Proposition 2 \& 4]{kaltenmark2017general}, for any $\xi\in\{0,1\}$ and $i\in[T_\xi-1]$, denoting the center and length of the $i^{th}$ segment $[g_i^\xi,g_{i+1}^\xi]$ by
$c_i^\xi = (g_i^\xi + g_{i+1}^\xi)/2$, $l_i^\xi = \| g_{i+1}^\xi-g_{i}^\xi\|$, and 
$\overrightarrow{v_i}^\xi = (g_{i+1}^\xi-g_{i}^\xi)/l_i^\xi$, the varifold distance between $\msg_0$ and $\msg_1$  is defined as,
\begin{align}
  &d_{\msw^*}^2(\msg_0,\msg_1) = \sum_{i,j = 1}^{T_0-1}l^0_i k((c^0_i,\overrightarrow{v_i}^0),(c^0_j,\overrightarrow{v_j}^0))l^0_j
  - 2 \sum_{i=1}^{T_0-1}\sum_{j=1}^{T_1-1}l^0_i k((c^0_i,\overrightarrow{v_i}^0),(c^1_j,\overrightarrow{v_j}^1))l^1_j \\
  &+ \sum_{i,j = 1}^{T_1-1}l^1_i k((c^1_i,\overrightarrow{v_i}^1),(c^1_j,\overrightarrow{v_j}^1))l^1_j 
\end{align}

In practice, we set the kernel $k$ as the product of two anisotropic Gaussian kernels, $k_{\pos}$ and $k_{\dir}$, 
such that for any $(x,\overrightarrow{u}),(y,\overrightarrow{v}) \in (\Rset^{d+1} \times \mathbb{S}^d)^2$
\begin{equation}
 k((x,\overrightarrow{u}),(y,\overrightarrow{v})) = k_{\pos}(x,y)k_{\dir}(\overrightarrow{u},\overrightarrow{v}) \eqsp.
 \end{equation}
 Note that the loss kernel $k$ has nothing to do with the velocity field kernel denoted by $K_\msg$ or $K$ specified in \Cref{paragraph:kernel_V}.
Finally, we define the data fidelity loss function, $\scrl$, as a sum of $ d_{\msw^*}^2$ using different kernel's width parameters $\sigma$ to incorporate multiscale information. $\scrl$ is indeed differentiable with respect to its first variable.
The specific kernels $k_{\pos},k_{\dir}$ that we use in our experiments are given \Cref{appendix:kernel_implementation}.
For further readings on curves and surface representation as varifolds, readers can refer to \cite{kaltenmark2017general,charon2013varifold}. 

 



%\sam{TO improve by Thibaut
%\thi{parler du fait qu'on a pas pairing, grace au difféo on conserve la structure}
%To relax the geodesic problem \Cref{eq:geodesics_original}, the end point of the geodesics is compared to the target through a loss $\mathscr{L}$ as proposed in \eqref{eq:relaxation}.
%Since we are dealing with discrete time series $\tsig^j$ of different samples size $n_j$, we need a loss on set measures and not only on vectors.
% A natural choice would be the Maximum Mean Discrepency (MMD) \cite{dziugaite2015training,borgwardt2006integrating}, for any $X=(x_i)_{i\in[T_1]},Y=(y_i)_{i\in[T_2]}\subset Rset^{d'}$, given a kernel $k:\Rset^{d'}\times \Rset^{d'} \to \Rset$,
% \begin{align}
%  &\operatorname{MMD}^k(\mu_X,\mu_Y)=\frac{1}{T_1^2}\sum_{i,j=1}^{T_1} k(x_i,x_j)+\frac{1}{T_2^2}\sum_{i,j=1}^{T_2} k(y_i,y_j) \\
%  &-\frac{2}{T_1T_2}\sum_{i=1}^{T_1}\sum_{j=1}^{T_2} k(x_i,y_j) \eqsp , 
% \end{align}
% where $\mu_X=\sum_{i=1}^{T_1} \delta_{x_i}/T_1$, $\mu_Y=\sum_{j=1}^{T_2} \delta_{y_j}/T_2$.
%However, this distance is agnostic of the time evolutions in time series' graph $(t_i^j,\tsig^j_i)_{i\in[n_j]}$, there is no orientation of the set taken into account.
% That is why, we prefer to choose the Varifold distance $d_V(\cdot,\cdot)$ \cite{kaltenmark2017general} which was specifically designed for oriented curves.
%  The Varifold distance extends the MMD by adding tangent vectors to the input:
%   instead of considering a set $X=(x_i)_{i\in[T_1]}$, we extend the input as $\overrightarrow{X}=(x_i,v_i)_{i\in[T_1]}$ where $v_i$ is the tangent vector at $x_i$ such that
%    \begin{equation}
%      d_V(X,Y)=\operatorname{MMD}^{K_p}(\mu_{\overrightarrow{X}},\mu_{\overrightarrow{X}}) \eqsp ,
%    \end{equation}
%    where in practice the tangent vectors are taken as discrete derivative and $K_p$ is a kernel on $\Rset^{2d'}$ such that for any $(x,v),(y,w)\in (\Rset^{2d})^2$, 
%    we have, 
%    \begin{equation}
%      K_p((x,v),(y,w))=k(x,v)k_T(v,w) \eqsp ,
%    \end{equation}
%   where $k_T $ is a kernel on the tangent vector space.



%\sam{Parler de méthode adaptatif ici}
  






%Wavelet-Based Multiscale Flow For Realistic Image Deformation in the Large Diffeomorphic Deformation Model Framework
% 3.1 +3.2 + specialité de la décomposition (t,f(t,x))
% \section{Specificity to time series}
% We are looking to time seriess 
% Each time series $s: t\in[0,T]\mapsto s(t) $ is only known through its samples $(s(t_i)=s_i)_{i\in[N]} $ at timestamps $(t_i)_{i\in[N]}$.
%is described by its graph $G(s)=\{(y,s(t)), \} $



% \section{optimzation
% % We are not doing Pre-ESPAce car l'on s'intéresse à la reparamétrisation en temps, au lieu de calculer la transfo en temps après avoir
% % push en espace, on le fait avant pour que tout soit défini depuis s_0, méthode itérative car ça marche mieux, moins de calcul de gradient
% %Natural smoothing with TTS
% \label{section:optimization}
%     \begin{center}
%         \begin{tikzpicture}
%             \node at (0,0) {$G_f^I$};
%             \draw [-stealth](0.3,0) -- (1.3,0);
%             \node at (1.6,0) {$G_f^J$};
%             \draw [-stealth](1.6,-0.3) -- (1.6,-1.3);
%             \node at (1.6,-1.6) {$G_g^J$};
%             \draw [-stealth](1.3,-1.6) -- (0.3,-1.6);
%             \node at (0,-1.6) {$G_g^I$};
%             \draw [-stealth](0,-1.3) -- (0,-0.3);
        
%             \node at (0.8,0.3) {$T$};
%             \node at (0.8,-1.9) {$T^{-1}$};
%             \node at (1.9,-0.8) {$S_J$};
%             \node at (-0.4,-0.8) {$S_I^{-1}$};
%         \end{tikzpicture}
%         \end{center}
    
%         where $S_I^{-1} = T^{-1} \circ S_J^{-1} \circ T$


\vspace{-1ex}
\section{Experiments}
\label{section:experiments}
[L'intro est ce necessaire ?]


First, we show on synthetic data that the proposed representation is identifiable provided that the hyperparameters and the reference graph are wisely selected, i.e.,
 the parameter $v_0^*$ generating a deformation $\varphi^{\{v_0^*\}}$ of a time series graph $\msg$ can be estimated from the data $\msg,\varphi^{\{v_0^*\}}.\msg$ by solving the geodesic shooting problem \eqref{eq:relaxation}.
 Secondly, we illustrate the qualitative interest of TS-LDDMM in studying inter-individual variability on a clinical dataset.
  Thirdly, we demonstrate the quantitative performance of our representation by performing classification on shape-based datasets.
  The method is implemented on Python using the library JAX\footnote{https://github.com/google/jax}. The code was compiled on a server with NVIDIA RTX A2000 12GB GPU, Intel(R) Xeon(R) Gold 5220R CPU @ 2.20GHz, and 250 GB of RAM. The code will be available on Github.
\subsection{Synthetic experiments}
\begin{figure}[t]
    \centering
    \includegraphics[width=0.5\linewidth]{pictures/samples.pdf}
    \vspace{-2.5em}
    \caption{Plots of $\varphi^{\{v_0(\mathbf{\alpha}^*,\msx)\}}.\msx$ for different values of $\mathbf{\alpha}^*$ according to its sampling parameter $t_a,s_a,m_s $, taking $\msx=\msg(s_0)$ with $s_0:k\in [300]\to \sin(2\pi k/300) $.}
    \label{fig:exemple_synthetic}
    \vspace{-1em}
\end{figure}

\begin{table}
    \caption{Values of $\scrl(\varphi^{\{v_0(\mathbf{\alpha}^*,\msx)\}}.\msx,\varphi^{\{\hat{v}_0\}}.\msx)$ as $\mathbf{\alpha}^*$ is sampled according to Gen(10,10,50) and $\hat{v}_0$ is estimated using $K_\msg$ with varying parameters $\sigma_{T,1},\sigma_x$.}
      \centering
         \begin{tabular}{lrrrrrrr}
         \toprule
         $\sigma_{T,0} \backslash \sigma_x$  & 1 & 10 & 50 & 100 & 200 & 300 \\
         \midrule
         0.1 & 2e+0 & 3e-4  & 1e-5&4e-6&7e-4&4e-3 \\
        1 & 4e-2 & 1e-4  & 1e-5&4e-6&7e-4 &4e-3  \\
         100 & 4e-2 & 2e-4  & 1e-5&4e-6&7e-4&4e-3  \\
         \bottomrule
         \end{tabular}
      \label{table:synthetic2}
      \vspace{-1em}
  \end{table}
 


% \begin{subfigure}{0.7\linewidth}
%   \centering
%   \resizebox{\columnwidth}{!}{%
%      \begin{tabular}{lrrrr}
%      \toprule
%      $\sigma_{T,0} \backslash \sigma_x$  & 0.1 & 1 & 100 \\
%      \midrule
%      1 & 2e+0 & 4e-2 & 4e-2 \\
%      10 & 3e-4 & 1e-4  & 2e-4 \\
%      50 & 1e-5 & 1e-5  & 1e-5 \\
%      100 & 4e-6 & 4e-6 & 4e-6 \\
%      200 & 7e-4 & 7e-4 & 7e-4 \\
%      300 & 4e-3 & 4e-3 & 4e-3 \\
%      \bottomrule
%      \end{tabular}
%      %
%      }
%  \caption{Values of $\scrl(y^*,\varphi^{\{\hat{v}_0\}}.\msx)$ where $y^*$ is sampled according to Gen(10,10,50) and $\hat{v}_0$ is estimated using $K_\msg$ with varying parameters $\sigma_{T,1},\sigma_x$.}
% \end{subfigure}
First, we show the model identifiability when the kernel $K_G$ is well specified: the estimated parameter is a good approximation of the generating parameter when the generation and the estimation procedure use the same hyperparameters for the RKHS kernel $K_\msg$.
All the hyperparameter values for generation and estimation are given in \Cref{appendix:numerics_synthetic}.
We fix the initial control points as $\msx=\left(x_k=(k,\sin(2\pi k/300))\right)_{k\in[300]} $.
Given $m_{s}\in \Nset_{>0}$ and $t_{a},s_{a}>0$, we randomly generate initial momentums $\mathbf{\alpha}^*=(\alpha_k^*)_{k\in[\mathbf{n}_0]}$ with the following sampling, called Gen($m_s,t_a,s_a$):
 For any $k\in[\mathbf{n}_0]$, $\mathbf{\alpha}_k'$ is sampled according to a Gaussian normal distribution $\mathcal{N}(0_{d+1},I_{d+1})$.
Then, $(\alpha_k')_{k\in[\mathbf{n}_0]}$ is regularized by a rolling average of size $m_{s}$, we get $\bar{\mathbf{\alpha}}'=(\bar{\alpha}_k')_{k\in[\mathbf{n}_0]}$.
Finally, we normalize $\bar{\mathbf{\alpha}}'$ to derive $\mathbf{\alpha}^*$ such that $|([\alpha_k^*]_t)_{k\in[\mathbf{n}_0]}|=t_{\text{amp}}$ and $|([\alpha_k^*]_s)_{k\in[\mathbf{n}_0]}|=s_{\text{amp}}$ for any $k\in[\mathbf{n}_0]$, denoting by $[\alpha_k^*]_t,[\alpha_k^*]_s$ the time and space coordinates of $\alpha_k^*$ respectively.
Note that the regularizing step $(\mathbf{\alpha}_k')_{k\in[\mathbf{n}_0]}\to \bar{\mathbf{\alpha}}' $ is necessary to obtain realistic deformations which take into account the regularity induced by the RKHS $\msv$.
%  uniformly on the set $\mss_M=\{ (\alpha_k)_{k\in[\mathbf{n}_0]}\in (\Rset^{d+1})^{\mathbf{n}_0}:\eqsp |\alpha_k|=M\}  $ with $M>0,\mathbf{n}_0\in \Nset_{>0}$,
%  then we regularize $\mathbf{\alpha}$ by convolving
%   with ... to get the generation parameter $\mathbf{\alpha}^*$.
Then, using $v_0(\mathbf{\alpha}^*,\msx)$ as defined in \eqref{eq:def_v0} with initial momentums $\mathbf{\alpha}^*$ and control points $\msx$, we apply the induced deformation $\varphi^{\{v_0\}} $ by \eqref{eq:integration} to $\msx$ and obtain $\varphi^{\{v_0\}}.\msx$.
Finally, we solve \eqref{eq:relaxation} to recover an estimation $\hat{\mathbf{\alpha}}$ of $\mathbf{\alpha}^*$ and report the average relative error (ARE) $|v_0(\hat{\mathbf{\alpha}},\msx)-v_0(\mathbf{\alpha}^*,\msx)|_\msv/|v_0(\mathbf{\alpha}^*,\msx)|_\msv$ on 50 repetitions.
This procedure is performed for any $m_{s},t_{a},s_{a}\in \{10,50,100\}\times \{5,10,15,20\}^2 $.
 Mean, standard deviation, and maximum of the ARE on all these hyperparameters choices are respectively $\mathbf{0.10, 0.03, 0.17}$.
 Therefore, the estimation procedure \eqref{eq:relaxation} offers a good approximation of the true parameter when the kernel $K_\msg$ is well specified.
 We observe that the estimation is difficult when $t_a\ll s_a$ because the time series can be very noisy as illustrated in \Cref{fig:exemple_synthetic}: this impacts the Varifold loss which is sensitive to tangents.
  
 Secondly, we demonstrate a weak identifiability when the kernel $K_\msg$ is misspecified: we can reconstruct the graph time series' after deformations even if the hyperparameters of $K_\msg$
  are different during the generation and the estimation.
   The hyperparameters of $K_\msg$ during generation are $(c_0,c_1,\sigma_{T,0},\sigma_{T,1},\sigma_x)=(1,0.1,100,1,1)$ and we fix $\sigma_{T,1},c_0,c_1=(1,1,0.1) $ for $K_\msg$ during estimation.
   We aim to understand the impact of $\sigma_{T,1},\sigma_x$ on the reconstruction since they are encoding the smoothness of the transformation according to time and space.  

    For any choice of the hyperparameters $\sigma_{T,1},\sigma_x\in \{1,10,50,100,200,300 \}\times \{0.1,1,100\}$ related to $K_\msg$ in the estimation,
     we average $\scrl(\varphi^{\{v_0(\mathbf{\alpha}^*,\msx)\}}.\msx,\varphi^{\{\hat{v}_0\}}.\msx)$ on 50 repetitions when $\mathbf{\alpha}^*$ is sampled according to Gen$(10,10,50)$ and $\hat{v}_0=v_0(\hat{\alpha},\msx)$ denoting by $\hat{\alpha}$ the result of the minimization \eqref{eq:relaxation}.
  We observe in \Cref{table:synthetic2} that the reconstruction is almost perfect except in the case when $\sigma_{t,0}=1$ during estimation, while $ \sigma_{t,0}=100$ during generation.
   Compared to $\sigma_{T,0}$, $\sigma_x$ has nearly no impact on the reconstruction.
   In \Cref{appendix:kernel_implementation}-\ref{appendix:kernel_TS_LDDMM}, we propose guidelines to drive future hyperparameters tuning and further discussions related to $\sigma_{T,1},c_0,c_1$. 
\subsection{Qualitative analysis of respiratory behavior in mice}
%We consider a dataset composed of mouse respiratory time series before and after a drug injection.
% A complete presentation of this dataset is given in \Cref{appendix:mouse_dataset}.
%  The mouse are divided in two groups depending on their genotypes : \textit{colq} and \textit{wt}.
%  We aim to study the difference of respiratory shapes according to the genotype.
%   That is why, TS-LDDMM \eqref{eq:general_optimization_problem} is applied to derive the features representations $(\mathbf{\alpha}_j)_{j\in[N]}\in (\Rset^2)^N$ related to $N$=\sam{fill} respiratory time series coming from 14 mouse (7 \textit{colq} and \textit{wt}) before drug injection.
%   Then, a Principal Components Analysis (PCA) is performed on $(\mathbf{\alpha}_j)_{j\in[N]}$
\begin{figure*}[t]
  \centering
  \includegraphics[width=0.95\linewidth]{"./pictures/exp_1_bis.pdf"}
  \vspace{-1.5em}
  \caption{Analysis of the three principal components (PC) related to the respiratory cycles of the mouse before exposure.
  In Figure A), the densities of each genotype according to each PC are displayed. In Figure B), the deformations of the reference graph $S_0$ along each PC are given. In Figure D), the graph of reference $S^j$, also called barycenter, related to each mouse, is displayed according to their coordinates on PC1 and PC3. In Figure C) et E), illustrations of respiratory cycles related to mice coming from the \textbf{wt} and \textbf{colq} group are displayed.  }
  \label{fig:exp_1_PCA}
  %\vspace{-1em}
\end{figure*}

\begin{figure*}[t]
  \centering
  \includegraphics[width=0.95\linewidth]{"./pictures/exp_2.pdf"}
  \vspace{-1em}
  \caption{Analysis of the first Principal Component (PC1) related to the respiratory cycles of the mouse 
  before and after exposure. In Figure A), the densities of each genotype according are displayed. In Figure B), the deformations of the reference graph $S_0$ PC1 is given. In Figure C), respiratory cycles displayed with respect to time and according to their coordinates on PC1 and PC2}
  \label{fig:exp_2_PCA}
  \vspace{-1.5em}
\end{figure*}

This experiment highlights the \textit{interpretability} of TS-LDDMM for studying the inter-individual variability in a clinical dataset.
We consider a time series dataset recording the evolution of the respiratory airflow of mice exposed 
to an irritant molecule altering respiratory functions \cite{nervo2019respiratory}. The dataset is divided into two groups, one 
composed of 7 control mice (\textbf{wt}) and the other of 7 mice (\textbf{colq}) deficient in an enzyme 
involved in the control of respiration. For each mouse, the respiratory airflow was recorded for 
15 to 20 minutes before exposure to the irritant molecule and then for 35 to 40 minutes. A complete 
description of the dataset is given in the \Cref{appendix:mouse_dataset}.
By comparing the shape of individual respiratory cycles (inspiration + expiration, see \Cref{fig:exp_1_PCA}-C)), 
we show that TS-LDDMM features can encode genotype distinctive breathing behaviors and their evolution 
after exposure to the irritant molecule. 

We first compare breathing behaviors before exposure.
Solving \eqref{eq:general_optimization_problem}, we derive the reference respiratory cycle's graph $S_0$ and the TS-LDDMM features representations 
$(\alpha_j)_{j\in[N_1]}$ related to $N_1=700$ respiratory cycles extracted according 
to the procedure \cite{germain2023unsupervised}.
Then, we perform a kernel PCA on the initial velocity field $\left(v_0(\alpha_j,S_0)\right)_{j\in[N_1]}\in \msv^{N_1}$ defined in \eqref{eq:def_v0}.
In \Cref{fig:exp_1_PCA}, we focus on the analysis of the three Principal Components (PC).

As observable from \Cref{fig:exp_1_PCA}-B), principal components refer to different types of deformations. 
By interpreting \Cref{fig:exp_1_PCA}-B): Only PC1 accounts for time warping, PC2 expresses the trade-off between inspiration and expiration duration, and PC3 corresponds to a change in signal amplitude.
 Compared to \textbf{wt} mice, the distribution of \textbf{colq} mice 
TS-LDMMM feature representation along the PC1 axis has a heavy tail and the associated deformation (+3 $\sigma_{\text{PC}}$) 
shows an inspiration with two peaks. As illustrated in \Cref{fig:exp_1_PCA}-A), such respiratory cycles are preponderant
 with \textbf{colq} mice and may be caused by motor impairment due to their enzyme deficiency, \cite{germain2023unsupervised}.
  In addition, 
 the \textbf{colq} mice were smaller than the \textbf{wt} mice due to a delay in growth caused by their lack of an enzyme. 
 This difference can be seen on PC3 since the volumes of air (area under the curve) inspired and exhaled are
  smaller for the smaller mice. In correlation, the distribution of \textbf{wt} mice TS-LDDMM feature representations 
  along the PC3 axis have a heavy tail corresponding to large air volume as depicted by the deformation 
  (+3 $\sigma_{\text{PC}}$) in \Cref{fig:exp_1_PCA}-B). Finally, \Cref{fig:exp_1_PCA}-D) shows that PC1 and PC3 capture the main differences between
   the two groups as their respective reference graphs $S^j$ are located in different parts of the space. 
%In Figure \Cref{fig:exp_1_PCA}, we display coordinates densities per genotype and PC.

We perform a second experiment to analyze the evolution of breathing behaviors when mice are exposed to the irritant 
molecule. We follow the same procedure as before. However, we take $N_2=1400$ with 25\% (resp. 75\%) 
before (resp. after) exposure. In \Cref{fig:exp_2_PCA}, we focus on the first principal component PC since
it encodes the effect of the irritant molecule as depicted in \Cref{fig:exp_2_PCA}-C) (the exposure occurs at 20 minutes).
\Cref{fig:exp_2_PCA}-B) shows that the deformation (+3 $\sigma_{\text{PC}}$) leads to longer respiratory cycles that include pauses,
as observed in \cite{germain2023unsupervised}. As well, \Cref{fig:exp_2_PCA}-A) shows that TS-LDDMM features distributions are less spread 
out for \textbf{colq} mice compared to \textbf{wt} mice. Indeed, the irritant molecule inhibits the action of the 
deficient enzyme, \textbf{wt} mice strongly react to the irritant molecule, whereas \textbf{colq} mice are better 
adapted due to their deficiency.
%Secondly, we compare breathing behaviors before and after exposure to observe the impact of the irritant molecule.
% We follow the same procedure as for before exposure, but we take $N_2=1400$ respiratory cycles extracted according 
% to the procedure CITE....
% In \Cref{fig:exp_2_PCA}, we focus on the first Principal Component (PC) since it encodes the effect of the irritant molecule as demonstrated on Figure \ref{fig:exp_2_PCA}-C) (exposure at time 20).
% We observe on \Cref{fig:exp_2_PCA}-B) that after exposure the mouse have longuer breath such that the expiration is longuer than inspiration.
%  Moreover, we deduce from Figure \ref{fig:exp_2_PCA}-A) that \textbf{colq} are more constant in their breath compared to \textbf{wt} after exposure.
%
% In Figure REF, we 
% also display the reference respiratory cycle S0 and its deformations in the principal component directions.
%  Additionally, we learn each mouse's reference respiratory cycle and represent them in the first and third PC coordinates system in Figure REF. 



\subsection{Quantitative performances of the TS-LDDMM representation in classification}
Combined with a Support Vector Classifier (SVC) \cite{hsu2003practical}, TS-LDDMM representation can be used for 
classification tasks using the kernel associated with the initial velocity space $\msv$.
We compare TS-LDDMM-SVC classification performances with another SVC using representation 
learned with T-loss \cite{franceschi2019unsupervised}, an unsupervised deep learning feature 
representation method for time series. We also include fully supervised methods in deep learning 
-ResNet, CNN \cite{ismail2019deep}- and machine learning: Catch22 \cite{lubba2019catch22}, Rocket
\cite{dempster2020rocket}, Dynamic Time Wrapping k-Nearest Neighbors (DTW-kNN) 
\cite{muller2007dynamic}. Methods are compared using f1-score on several shape-based UCR/UEA datasets 
\cite{dau2019ucr,bagnall2018uea} introduced in \Cref{appendix:classification_dataset}. All implementation details are given 
in \Cref{appendix:classification_implementation}.
\Cref{table:classification} presents the reuslts. TS-LDDMM-SVC consistently outperforms the other unsupervised methods. It is ranked 1,3,4,3 for all methods combined, 
demonstrating its competitiveness as an unsupervised method on time series dataset homogeneous regarding shape.
%TS-LDDMM representation can be used in a Support Vector Classifier (SVC) \cite{hsu2003practical} using the kernel of the space of initial velocity $\msv$.
%We compare its capacity of representation to an SVC using T-Loss representation \cite{franceschi2019unsupervised}, which is an unsupervised representation deep learning method for time series,
%and to others fully supervised methods using deep learning -ResNet, CNN \cite{ismail2019deep}- or not: Catch22 \cite{lubba2019catch22}, Rocket \cite{dempster2020rocket}, Dynamic Time Wrapping \cite{muller2007dynamic} k-Nearest Neighbors (DTW-kNN).
%These methods are compared using f1-score on several datasets introduced in \Cref{appendix:classification_dataset}. All details of implementations are given in \Cref{appendix:classification_implementation}.
%TS-LDDMM-SVC is always ranked 2 or 3 behind a fully supervised method, demonstrating its competitiveness as an unsupervised method.
%The more homogeneous the shapes in the dataset are, the better TS-LDDMM-SVC performs.


\begin{table}[t]
  \vspace{-0.5em}
  \caption{Classification results in f1-score (U: unsupervised, S: supervised, DL: deep learning, ML: machine learning). \textbf{x} best unsupervised method, \underline{x} best supervised method.}
  \resizebox{\linewidth}{!}{%
  \begin{tabular}{lllrrrr}
    \toprule
     &  &  & ArrowHead & ECG200 & GunPoint & NATOPS \\
    \midrule
    \multirow[m]{3}{*}{U} & \multirow[t]{3}{*}{} & TS-LDDMM-SVC & \textbf{0.84} & \textbf{0.82} & \textbf{0.94} & \textbf{0.93} \\
     &  & T-loss-SVC & 0.57 & 0.76 & 0.82 & 0.88 \\
     &  & DTW-kNN & 0.70 & 0.75 & 0.91 & 0.88 \\
    \cline{1-7} 
    \multirow[m]{5}{*}{S} & \multirow[m]{2}{*}{DL} & CNN & 0.70 & 0.79 & 0.85 & \underline{0.96} \\
     &  & ResNet & 0.77 & 0.87 & 0.97 & 0.95 \\
    \cline{2-7}
     & \multirow[m]{2}{*}{ML} & Catch22 & 0.73 & 0.81 & 0.96 & 0.89 \\
     &  & Rocket & \underline{0.81} & \underline{0.91} & \underline{1.00} & 0.88 \\
    \bottomrule
    \end{tabular}
    \label{table:classification}
  %
}
\vspace{-2em}
  \end{table}
\vspace{-1ex}

\section{Related Works}
\vspace{-1ex}
Shape analysis focuses on statistical analysis of mathematical objects invariant under some deformations like rotations, dilations, or time parameterization.
 The main idea is to represent these objects in a complete Riemannian manifold $(\mathcal{M},\mathbf{g})$ with a metric $\mathbf{g}$ adapted to the geometry of the problem \cite{miller2006geodesic}.
 Then, any set of points in $\mathcal{M}$ can be represented as points in the tangent space of their Frechet mean $\mathbf{m}_0$ \cite{pal2017riemannian,le2001locating} by considering their logarithms.
 The goal is to find a well-suited Riemannian structure according to the nature of the studied object.
 
 LDDMM framework is a relevant shape analysis tool to represent curves as depicted in \cite{glaunes2008large}. However, graphs of time series are a well-structured type of curve due to the inclusion of the temporal dimension that requires specific care (\Cref{fig:diffeo}).
 In a similar vein, Qiu \textit{et al} \cite{qiu2009time} proposes a method for tracking anatomical shape changes in serial images using LDDMM. They include temporal evolution, but not for the same purpose: the aim is to perform longitudinal modeling of brain images.

 Leaving the LDDMM representation, the results of \cite{srivastava2010shape,heo2024logistic} address the representation of curves with the Square-Root Velocity (SRV) representation.
 However, the SRV representation is applied after reparametrization of the temporal dimension of the unit length segment.
 Consequently, the graph structure of the time series is not respected, and the original time evolution of the time series is not encoded in the final representation.
 Very recently, in a functional data analysis framework, a paper \cite{wu2024shape} (Shape-FPCA) improved by representing the original time evolution.
 Nevertheless, this method is made for \textit{continuous objects} and only applies to time series of \textit{same length}, making the estimation more sensitive to noise.

 Balancing between discrete and continuous elements is a challenging task.
 In the deep learning literature \cite{chen2018neural, kidger2020neural, tzen2019neural, jia2019neural, liu2019neural, ansari2023neural}, Neural Ordinary Differential Equations (Neural ODEs) \cite{chen2018neural} learn continuous latent representations using a vector field parameterized by a neural network, serving as a continuous analog to Residual Networks \cite{zagoruyko2016wide}.
 This approach was further enhanced by Neural Controlled Differential Equations (Neural CDEs) \cite{kidger2020neural} for handling irregular time series, functioning as continuous-time analogs of RNNs \cite{schuster1997bidirectional}.
 Extending Neural ODEs, Neural Stochastic Differential Equations (Neural SDEs) introduce regularization effects \cite{liu2019neural}, although optimization remains challenging.
 Leveraging techniques from continuous-discrete filtering theory, Ansari et al. \cite{ansari2023neural} applied successfully Neural SDEs to irregular time series.
 Oh \textit{et al.} \cite{oh2024stable} improved these results by incorporating the concept of controlled paths into the drift term, similar to how Neural CDEs outperform Neural ODEs.
 With TS-LDDMM, the representation is also derived from an ODE, but the velocity field is parameterized with kernels and optimized to have a minimal norm, which enhances interpretability.

 All these state-of-the-art methods previously mentionned \cite{glaunes2008large,oh2024stable,wu2024shape,heo2024logistic} are compared to TS-LDDMM in \Cref{appendix: robustness} and \Cref{appendix: shape_classification}.
 \vspace{-1ex}
  \section{Limitations and conclusion}
  \vspace{-1ex}
  \label{sec:limitations}
  
  This paper proposes a feature representation method, TS-LDDMM, designed for 
shape comparison on homogeneous time series datasets. We show on a real dataset 
its ability to study, with high interpretability, the inter-individual shape 
variability. As an unsupervised approach, it is user-friendly and enables knowledge 
transfer for different supervised tasks such as classification.

Although TS-LDDMM 
is already competitive for classification, its performances can be leveraged on 
more heterogeneous datasets using a hierarchical clustering extension, which is relegated for future work. 
%   While TS-LDDMM performs well on shape-based datasets, it assumes a certain degree of homogeneity with the existence of a reference time series graph$\msg_0$.
%  Extending TS-LDDMM to more heterogeneous datasets using clustering is ongoing work.

 TS-LDDMM employs kernel computations, which require specific libraries (e.g., KeOps \cite{charlier2021kernel}) to be efficient and scalable.
 However, in our experiments, the time complexity of TS-LDDMM is comparable to that of competitors.
 It is clear that TS-LDDMM needs to be extended to handle very large datasets with high-dimensional time series (such as videos).
 
 Additionally, TS-LDDMM requires tuning several hyperparameters, though this is a common requirement among competitors \cite{glaunes2008large, oh2024stable, wu2024shape, heo2024logistic}.
 In future work, adaptive methods are expected to be developed to provide a more user-friendly interface.
  %\color{red} rappeler les limites de la méthodes : petite dimension à cause des kernels, tuner les hyper paramètres correctement, version adaptative}
% \paragraph*{Conclusion}
% \vspace{-1ex}




%\section{Societal impact}
%This paper presents work whose goal is to advance the field of Machine Learning. There are many potential societal consequences of our work, none which we feel must be specifically highlighted here.
% \section{Acknowledgements}

%The TS-LDDMM feature representation proposed in this paper enables us to study the inter-individual variability of shapes in a time series dataset with high interpretability.
%Its unsupervised aspect is user-friendly and enables knowledge transfer for different supervised tasks such as classification.
%Although the method is already competitive,
% its capacity can be leveraged on more heterogeneous datasets using a hierarchical clustering extension.
% This is relegated to future works.

%Applying NF or CNF to our
 %  framework with high dimensional time series is relagated for future works.

% \begin{figure}[ht]
% \vskip 0.2in
% \begin{center}
% \centerline{\includegraphics[width=\columnwidth]{icml_numpapers}}
% \caption{Historical locations and number of accepted papers for International
% Machine Learning Conferences (ICML 1993 -- ICML 2008) and International
% Workshops on Machine Learning (ML 1988 -- ML 1992). At the time this figure was
% produced, the number of accepted papers for ICML 2008 was unknown and instead
% estimated.}
% \label{icml-historical}
% \end{center}
% \vskip -0.2in
% \end{figure}


% \begin{algorithm}[tb]
%    \caption{Bubble Sort}
%    \label{alg:example}
% \begin{algorithmic}
%    \STATE {\bfseries Input:} data $x_i$, size $m$
%    \REPEAT
%    \STATE Initialize $noChange = true$.
%    \FOR{$i=1$ {\bfseries to} $m-1$}
%    \IF{$x_i > x_{i+1}$}
%    \STATE Swap $x_i$ and $x_{i+1}$
%    \STATE $noChange = false$
%    \ENDIF
%    \ENDFOR
%    \UNTIL{$noChange$ is $true$}
% \end{algorithmic}
% \end{algorithm}







\begin{ack}
This work was supported by grants from Région Ile-de-France (DIM MathInnov).
Charles Truong is funded by the PhLAMES chair of ENS Paris-Saclay.
\end{ack}

\bibliographystyle{plain}
\bibliography{ref.bib}

\newpage

%%%%%%%%%%%%%%%%%%%%%%%%%%%%%%%%%%%%%%%%%%%%%%%%%%%%%%%%%%%%

\appendix





%%%%%%%%%%%%%%%%%%%%%%%%%%%%%%%%%%%%%%%%%%%%%%%%%%%%%%%%%%%%%%%%%%%%%%%%%%%%%%%
%%%%%%%%%%%%%%%%%%%%%%%%%%%%%%%%%%%%%%%%%%%%%%%%%%%%%%%%%%%%%%%%%%%%%%%%%%%%%%%
% APPENDIX
%%%%%%%%%%%%%%%%%%%%%%%%%%%%%%%%%%%%%%%%%%%%%%%%%%%%%%%%%%%%%%%%%%%%%%%%%%%%%%%
%%%%%%%%%%%%%%%%%%%%%%%%%%%%%%%%%%%%%%%%%%%%%%%%%%%%%%%%%%%%%%%%%%%%%%%%%%%%%%%

\section{Societal impact}
\label{appendix:societal_impact}
We believe that the paper has a positive societal impact for the following reasons:
\begin{itemize}
\item TS-LDDMM is an interpretable method for understanding inter-individual variability in biomedical datasets, potentially offering new insights in medicine.
\item TS-LDDMM bridges the gap between the shape analysis community and the unsupervised representation learning (URL) community, fostering potential future collaborations between these fields.
\end{itemize}
However, the computational cost of the method may raise environmental concerns similar to those associated with deep learning \cite{oh2024stable}.
 Additionally, while TS-LDDMM has promising biomedical applications, it could also be misused for creating poison.
\section{Proofs}
\label{appendix:proofs}
Denote by $\msg(s)\triangleq \{ (t,s(t)): \eqsp t\in \msi \} $ the graph of a time series $s: \msi \to \Rset^d$ and $ \phi.\msg(s)\triangleq\{ \phi(t,s(t)): \eqsp t\in \msi\} $ the action of  $\phi\in \mcd(\Rset^{d+1}) $ on $\msg(s)$.
\begin{theorem}
    \label{theorem:representation_proof}
Let $s:  \msj \to \Rset^d  $ and $\mathbf{s}_0: \msi\to \Rset^d $ be two continuously differentiable time seriess with $\msi,\msj$ two intervals of $\Rset$.
 There exist $f\in \rmC^1(\Rset^{d+1},\Rset^d)$ and $\gamma\in  \mcd(\Rset) $ such that $\gamma(\msi)=\msj $ and $\Phi_f\in \mcd(\Rset^{d+1})$,
 \begin{equation}% TO add more detail on functionnal space
    \msg(s)= \Pi_{\gamma,f}.\msg(\mathbf{s}_0),\eqsp \Pi_{\gamma,f}=\Psi_\gamma\circ\Phi_f.
 \end{equation}
 Moreover, for any $\bar{f}\in \rmC^1(\Rset^{d+1},\Rset^d)$ and $\bar{\gamma}\in  \mcd(\Rset) $, there exists a continously differentiable time series $\bar{s}$ such that 
 $\msg(\bar{s})= \Pi_{\bar{\gamma},\bar{f}}.\msg(\mathbf{s}_0)$
\end{theorem}
%\sam{preuve todo, classement des variétés à bord, donne un homeomorphisme sur la restriction}
\begin{proof}
  Let $s:  \msj \to \Rset^d  $ and $\mathbf{s}_0: \msi\to \Rset^d $ be two continuously differentiable time seriess with $\msi=(a,b),\msj=(\alpha,\beta)$ two intervals of $\Rset$.
  By setting $\gamma: t\in \Rset \mapsto (\beta-\alpha)(t-a)/(b-a)+\alpha\in \Rset $, we have $ \gamma(\msi)=\msj$ and $\gamma \in \mcd(\Rset) $.
   By defining $f:(t,x)\in\Rset^{d+1}\mapsto x-\mathbf{s}_0(t)+s\circ \gamma(t) $, the map $\Phi_f\in \mcd(\Rset^{d+1})$,
    indeed, its inverse is $\Phi_f^{-1}:(t,x)\in\Rset^{d+1}\mapsto (t,x+\mathbf{s}_0(t)-s(t)) $ and is continuously differentiable.
     Moreover, we have $\Pi_{\gamma,f}.\msg(\mathbf{s}_0)=\{(\gamma(t),s\circ \gamma(t)):\eqsp t\in\msi \}=\msg(s) $.

    

    Let $\bar{f}\in \rmC^1(\Rset^{d+1},\Rset^d)$, $\bar{\gamma}\in  \mcd(\Rset) $ and $\mathbf{s}_0\in \rmC^1(\msi,\Rset^d)$ with $\msi$ an interval of $\Rset$.
    We have :
    \begin{align}
      \Pi_{\gamma,f}.\msg(\mathbf{s}_0)&=\{(\gamma(t),f(t,\mathbf{s}_0(t))),\eqsp t\in \msi \} \\
      &\label{eq:proof1_last_eq}=\{(t,f\left(\gamma^{-1}(t),\mathbf{s}_0(\gamma^{-1}(t))\right),\eqsp t\in \gamma(\msi) \} \eqsp .
    \end{align}
    By defining $\bar{s}:t\in \gamma(\msi)\to f\left(\gamma^{-1}(t),\mathbf{s}_0(\gamma^{-1}(t))\right) $, we have $\bar{s}\in \rmC^1(\gamma(\msi), \Rset^d) $ by composition of $C^1$ functions
    and $ \msg(\bar{s})= \Pi_{\gamma,f}.\msg(\mathbf{s}_0)$ by \eqref{eq:proof1_last_eq}, which concludes the proof.
\end{proof}
\begin{lemma}
  If we denote by $\msv$ the RKHS associated with the kernel $K_{\msg}$, then for any vector field $v$ generated by \eqref{eq:integration} with $v_0$ satisfying \eqref{eq:def_v0},
   there exist $\gamma \in \msd(\Rset) $ and $f\in \rmC^1(\Rset^{d+1},\Rset^d)$ such that $\phi^v=\Psi_\gamma\circ\Phi_f $.
\end{lemma}
\begin{proof}
  Let $v$ be a vector field generated by \eqref{eq:integration} with $v_0$ satisfying \eqref{eq:def_v0}.
 We remark that the first coordinate of the velocity field $v_\tau$ denoted by $v_\tau^{\text{time}}$ only depends on the time variable $t$ for any $\tau\in[0,1]$.
 Thus, when computing the first coordinate of the deformation $\phi^v$, denoted by $\gamma$, we integrate \eqref{eq:LDDMM_dynamic} with $v_\tau$ replaced by $v_\tau^{\text{time}}$,
  thus $\gamma$ is independant of the variable $x$. Moreover, $\gamma\in \mcd(\Rset)$ since a Gaussian kernel induced an Hilbert space $\msv$ satisfying $|f|_V\leq |f|_\infty+ |\dd f|_\infty  $ for any $f\in \msv$ by \citep[Theorem 9]{glaunes2005transport}.
  For the same reason, we have $\phi^v\in \mcd(\Rset^{d+1})$, and thus its last coordinates denoted by $f$ belongs to $\rmC^1(\Rset^{d+1},\Rset^d)$, and by construction $\phi^v=\Psi_\gamma\circ\Phi_f $.
\end{proof}




\section{Oriented varifold}

\label{appendix:varifold}
In this section, we introduce the \textit{oriented varifold} associated with curves.
For further readings on curves and surfaces representation as varifolds, readers can refer to \cite{kaltenmark2017general,charon2013varifold}. 
We associate to $\gamma\in \rmC^1((a,b),\Rset^{d+1})$ an \textit{oriented varifold} $\mu_\gamma$, i.e. a distribution on the space $\Rset^{d+1}\times \mathbb{S}^{d}$ defined as follows, for any smooth test function $\omega :\Rset^{d+1}\times \mathbb{S}^{d}\to \Rset  $,
\begin{equation}
  \mathbb{E}_{Y\sim \mu_\gamma}\left[\omega(Y)\right]=\mu_\gamma(\omega)=\int_a^b \omega\left(\gamma(t),\frac{\dot{\gamma}(t)}{|\dot{\gamma}(t)|}\right)|\dot{\gamma}(t)|\dd t \eqsp .
\end{equation}
Denoting by $\msw$ the space of smooth test function, we have that $\mu_\gamma $ belongs to its dual $\msw^*$.
 Thus, a distance on $\msw^*$ is sufficient to set a distance on oriented varifolds associated to curve and thus on $\rmC^1((a,b),\Rset^{d+1})$ by the identification $\gamma\to \mu_\gamma $.
Remark that in (TS-LDDMM), $\gamma$ should be the parametrization of a time series' graph $\msg(s)$, i.e. $\gamma: t\in \msi \to (t,s(t))\in\Rset^{d+1} $ denoting by $s:\msi \to \Rset^d$ the time series.
However, in practice, we work with discrete objects.
 That is why, we set $W$ as an RKHS to use its representation theorem.
 More specifically \citep[Proposition 2 \& 4]{kaltenmark2017general} encourages us to consider a kernel $k:(\Rset^{d+1} \times \mathbb{S}^d)^2\to \Rset$ such that there exist two positive and continuously differentiable kernels $k_{\pos}$ and $k_{\dir}$, 
 such that for any $(x,\overrightarrow{u}),(y,\overrightarrow{v}) \in (\Rset^{d+1} \times \mathbb{S}^d)^2$
 \begin{equation}
     k((x,\overrightarrow{u}),(y,\overrightarrow{v})) = k_{\pos}(x,y)k_{\dir}(\overrightarrow{u},\overrightarrow{v}) \eqsp,
   \end{equation}
 with moreover $k_\dir>0$ and $ k_\pos$ which admits an RKHS $\msw_\pos$ dense in the space of continous function on $\Rset^{d+1}$ vanishing at infinite \cite{carmeli2010vector}.

Given such a kernel $k:(\Rset^{d+1} \times \mathbb{S}^d)^2\to \Rset$
verifying \citep[Proposition 2 \& 4]{kaltenmark2017general}, we have that for any $(x,v)\in \Rset^{d+1}\times \mathbb{S}^{d}$, $\delta_{(x,\overrightarrow{v})}$ belongs to $\msw^*$ as a distribution and that the dual metric $\langle\cdot,\cdot \rangle_{\msw^*} $ satisfies for any $(x_1,v_1),(x_2,v_2)\in \left(\Rset^{d+1}\times \mathbb{S}^{d}\right)^2$,
\begin{equation}
  \langle\delta_{(x_1,\overrightarrow{v}_1)},\delta_{(x_2,\overrightarrow{v}_2)} \rangle_{\msw^*} =k((x_1,\overrightarrow{v}_1),(x_2,\overrightarrow{v}_2)) \eqsp .
\end{equation}
Thus, given two sets of triplets $X=(l_i,x_i,\overrightarrow{v}_i)_{i\in[T_0-1]}\in (\Rset\times \Rset^{d+1}\times \mathbb{S}^d)^{T_0-1},Y=(l_i',y_i,\overrightarrow{w}_i)_{i\in[T_1]}\in (\Rset\times \Rset^{d+1}\times \mathbb{S}^d)^{T_1-1}$ and denoting by 
\begin{equation}
  \label{eq:def_mu_X}
  \mu_X=\sum_{i=1}^{T_0} l_i \delta_{(x_i,\overrightarrow{v}_i)},\mu_Y=\sum_{i=1}^{T_1} l_i' \delta_{(y_i,\overrightarrow{w}_i)} \eqsp,
\end{equation}
 we have,
\begin{equation}
  \begin{array}{ll}
    |\mu_X-\mu_Y |_{\msw^*}^2 = & \sum_{i,j = 1}^{T_0-1}l_i k((x_i,\overrightarrow{v_i}),(x_i,\overrightarrow{v_i}^0))l_j + \sum_{i,j = 1}^{T_1-1}l_i' k((y_i,\overrightarrow{w_i}),(y_i,\overrightarrow{w_i}))l_j' \\
    & - 2 \sum_{i=1}^{T_0-1}\sum_{j=1}^{T_1-1}l_i k((x_i,\overrightarrow{v_i}),(y_i,\overrightarrow{w_i}))l_j' \eqsp.
  \end{array}
\end{equation}
Then, using the identification $X\to \mu_X, Y\to \mu_Y$, we can define a distance on sets of triplets as $d_{\msw^*,3}(X,Y)=|\mu_X-\mu_Y|_{\msw^*}^2$.

Now, we aim to discretize the oriented varifold $\mu_\msg $ related to a time series' graph $\msg(s)$ by using a set of triplets.
This is carried out by using a discretized version of $\msg(s) $, i.e. $\tilde{\msg}=(g_i=(t_i,s(t_i)))_{i\in[T]}\in(\Rset^{d+1})^T$, in the following way: 
For any  $i\in[T-1]$, denoting the center and length of the $i^{th}$ segment $[g_i,g_{i+1}]$ by
$c_i = (g_i + g_{i+1})/2$, $l_i = \| g_{i+1}-g_{i}\|$, and the unit norm vector of direction $\overrightarrow{g_i g_{i+1}}$ by
 $\overrightarrow{v_i} = (g_{i+1}-g_{i})/l_i$, we define the set of triplets $X(\tilde{\msg})=(l_i,c_i,\overrightarrow{v_i})_{i\in[T-1]}$ and its related oriented varifold $\mu_{X(\tilde{\msg})}= \sum_{i=1}^{T-1}l_i \delta_{c_i,\overrightarrow{v_i}}$ as in \eqref{eq:def_mu_X}.
 This is a valid discretization of the oriented varifold $\mu_\msg$ according to \citep[Proposition 1]{kaltenmark2017general}: $\mu_{X(\tilde{\msg})}$ converges towards $\mu_\msg$ as the size of the descretization mesh $\sup_{i\in[T-1]} |t_{i+1}-t_i| $ converges to 0.

 Finally, we define a distance on discretized time series' graphs $\tilde{\msg}_1,\tilde{\msg}_2$ as $d_{\msw^*}(\tilde{\msg}_1,\tilde{\msg}_2)=d_{\msw^*,3}(X(\tilde{\msg}_1),X(\tilde{\msg}_2)) $.


\subsection{Varifold kernels}
\label{appendix:kernel_implementation}
Denote the one-dimensional Gaussian kernel by $K_\sigma^{(a)}(x,y)=\exp(-|x-y|^2/\sigma)$ for any $(x,y)\in (\Rset^a)^2$, $a\in \Nset$ and $\sigma>0$.
In the implementation, we use the following kernels, for any $((t_1,x_1),(t_2,x_2))\in (\Rset^{d+1})^2, ((w_1,v_1),(w_2,v_2))\in (\mathbb{S}^{d})^2 $,
\begin{equation}
  k_{\pos}(x,y)=K_{\sigma_{\pos,t}}^{(1)}(t_1,t_2)K_{\sigma_{\pos,x}}^{(d)}(x_1,x_2),\quad k_{\pos}(x,y)=K_{\sigma_{\dir,t}}^{(1)}(w_1,w_2)K_{\sigma_{\dir,x}}^{(d)}(v_1,v_2)\eqsp ,
\end{equation}
where $\sigma_{\pos,t}, \sigma_{\pos,x}, \sigma_{\dir,t}, \sigma_{\dir,x}>0 $ are hyperparameters.
 In practice, we select $\sigma_{\pos,x}\approx \sigma_{\dir,x} \approx 1$ when the times series are centered and normalized. Otherwise we select $\sigma_{\pos,x}\approx \sigma_{\dir,x} \approx \bar{\sigma}_{s}$ with $\bar{\sigma}_{s}$ the average standard deviation of the time series.
  We choose $\sigma_{\pos,t}\approx \sigma_{\dir,t} = m f_e$ with $f_e$ the sampling frequency of the time series and $m\in [5]$ an integer depending on the time change between the starting and the target time series graph.
  The more significant the time change, the higher $m$ should be. The intuition comes from the fact that the width $\sigma_{\pos,t}, \sigma_{\dir,t}$ rules the time windows used to perform the comparison, and $\sigma_{\pos,x}, \sigma_{\dir,x}$ affects the space window.
   The size of the windows should be selected depending on the variations in the data.

\section{Tuning the hyperparameters of the TS-LDDMM velocity field kernel}
\label{appendix:kernel_TS_LDDMM}
The parameter $\sigma_{T,0}$ should be chosen \textit{large} compared the sampling frequency $f_e$ and compared to average standard deviation $\bar{\sigma}_s$ of the time series, e.g $\sigma_{T,0}=100$ as $\bar{\sigma}_s\approx f_e\approx  1 $.
It makes the time transformation smoother. If $\sigma_{T,0}$ is too small, for instance, $\sigma_{T,0}=f_e $, the effect of the time deformation is too localized, and there are not enough samples to make it visible.

The parameter $\sigma_{T,1}$ should be of the same order as $f_e$: two different points in time can have various space transformations.
 $\sigma_x$ should be of the same order of $\bar{\sigma}_s$: two points with a big difference regarding space compared to $\bar{\sigma}_s$ can have very different space transformations.

 We take $c_0\approx 10 c_1 $, we want to encourage time transformation before space transformation. We take $(c_0,c_1)=(1,0.1)$ in all experiments.


\section{Experimental settings}
\label{appendix:settings}
All experiments were performed on a Debian 6.1.69-1 server with NVIDIA RTX A2000 12GB GPU, Intel(R) Xeon(R) Gold 5220R CPU @ 2.20GHz, and 250 GB of RAM. The source code is available on Github\footnote{\url{https://github.com/thibaut-germain/TSLDDMM}}.

\subsection{Optimization details of TS-LDDMM \& LDDMM}
\label{appendix:optimizers_details}

We implemented TS-LDDMM in Python with the JAX library \footnote{https://github.com/google/jax}.

\paragraph{Initialization.}
As initialization of \eqref{eq:general_optimization_problem}, all momentum parameters are set to $0$, and the initial graph of reference is picked from the dataset such that its length is equal to the median length observed in the dataset.

\paragraph{Gradient descent.}
The chosen gradient descent method is "adabelief" \cite{zhuang2020adabelief} implemented in the OPTAX library~\footnote{https://optax.readthedocs.io/en/latest/}.
The gradient descent has two main parameters: the number of steps (nb\_steps) and the maximum stepsize value ($\eta_M$).
The stepsize has a scheduling scheme: 
\begin{itemize}
  \item Warmup period on $0.1 \times$ nb\_steps steps: the stepsize increases linearly from $0$ to $\eta_M$. The goal is to learn progressively the parameters. If the step size is too large at the start, smaller steps at the end cannot make up for the mistakes made at the beginning. 
  \item Fine tuning periode on $ 0.9  \times$ nb\_steps : the stepsize decreases from $\eta_M$ to $0$ with a cosine decay implemented in the OPTAX scheduler, i.e. the decreasing factor as the form $0.5  (1 + \cos(\pi  t/T))$. 
\end{itemize}
By default, we set nb\_steps to 400 and $\eta_M$ to 0.1.

\section{Datasets}
\subsection{Mouse respiratory cycle dataset}

\begin{figure}
  \centering
  \includegraphics[width = \linewidth]{pictures/mice_exp.png}
  \caption{A: Illustration of a double-chamber plethysmograph. The term \textit{dpt} stands for differential 
  pressure transducer which measures the pressure in each compartment, the pressure then being converted to flow. 
  B: Nasal airflow (top) and lung volume (bottom). During inspiration, airflow is positive (grey) and during
  expiration, airflow is negative (pink).}
  \label{fig:mice_exp}
\end{figure}

\label{appendix:mouse_dataset}
Ventilation is a simple physiological function that ensures a vital supply of oxygen and the elimination of CO2. 
Acetylcholine (Ach) is a neurotransmitter that plays an important role in muscular activity, notably for breathing. 
Indeed, muscle contraction information passes from the brain to the muscle through the nervous system. Achs are located 
in synapses of the nervous system (central and peripheral) and skeletal muscles. They ensure the information transmission 
from nerve to nerve. However, the transmission cannot end without the hydrolysis of Ach by the enzyme Acetylcholinesterase 
(AchE), allowing nerves to return to their resting state. Inhibition of (AchE) with, for instance, nerve gas, pesticide, 
or drug intoxication leads to respiratory arrests. 

The dataset comes from the experiment \cite{nervo2019respiratory}, where they studied the consequences of partial 
deficits in AChE and AChE inhibition on mice respiration. AchE inhibition was induced with an 
irritant molecule called physostigmine (an AchE inhibitor). Mice nasal airflows were sampled at 
2000Hz with a Double Chamber plethysmograph \cite{hoymann2012lung}, as depicted in \Cref{fig:mice_exp}-A). The flow is expressed in 
$ml.s^{-1}$; it has a positive value during inspiration and a negative value expiration \Cref{fig:mice_exp}-B). 
Among the mice population, we selected 7 control mice (\textbf{wt}) and 7 ColQ mice (\textbf{colq}), which do not have 
AChE anchoring in muscles and some tissues. 
As described in \cite{nervo2019respiratory}, mice experiments were as follows:
\begin{enumerate}
  \item The mouse is placed in a DCP for 15 or 20 min to serve as an internal control.
  \item The mouse is removed from the DCP and injected with physostigmine.
  \item The mouse is placed back into the DCP, and its nasal flow is recorded for 35 or 40 min.
\end{enumerate}

Respiratory cycles were extracted following procedure \cite{germain2023unsupervised}. We removed 
respiratory cycles whose duration exceeds 1 second; the average respiratory cycle duration is 
300 ms. We randomly sampled 10 respiratory cycles per minute and mouse. It leads to a dataset of 
12,732 (time, genotype)-annotated respiratory cycles. 

\subsection{Shape-based UCR/UEA time series classification datasets}
\label{appendix:classification_dataset}
We selected 15 shape-based datasets (7 univariates and 8 multivariates) from the from the University of East Anglia (UEA) and the University of California Riverside (UCR) Time Series Classification Repository\footnote{https://timeseriesclassification.com} \cite{dau2019ucr,bagnall2018uea}. All datasets were downloaded with the python package aeon\footnote{https://www.aeon-toolkit.org/en/stable/}. Essential datasets information are summarized in \Cref{appendix:table:datasets} and further can be found in \cite{dau2019ucr,bagnall2018uea}.

\begin{table}[hbt!]
  \centering
  \caption{UCR/UEA shape-based time series datasets for classification.}
  \resizebox{\columnwidth}{!}{%
  \begin{tabular}{lllllll}
    \toprule
    & \textbf{Dataset} &  \textbf{Size} & \textbf{Lengh} & \textbf{Number of classes} & \textbf{Number of dimensions} & \textbf{Type} \\
    \midrule
    \multirow[c]{7}{*}{Univariate} &  ArrowHead & 211 & 251 & 3 & 1 & IMAGE \\
    & BME & 180 & 128 & 3 & 1 & SIMULATED \\
    & ECG200 & 200 & 96 & 2 & 1 & ECG \\
    & FacesUCR & 2250 & 131 & 14 & 1 & IMAGE \\
    & GunPoint & 200 & 150 & 2 & 1 & MOTION \\
    & PhalangesOutlinesCorrect & 2658 & 80 & 2 & 1 & IMAGE \\
    & Trace & 200 & 275 & 4 & 1 & SENSOR \\
    \cline{1-7}
    \multirow[c]{8}{*}{Multivariate}& ArticularyWordRecognition & 575 & 144 & 25 & 9 & SENSOR \\
    & Cricket & 180 & 1197 & 12 & 6 & MOTION \\
    & ERing & 60 & 65 & 6 & 4 & SENSOR \\
    & Handwriting & 1000 & 152 & 26 & 3 & MOTION \\
    & Libras & 360 & 45 & 15 & 2 & VIDEO \\ 
    & NATOPS & 360 & 51 & 6 & 24 & MOTION \\
    & RacketSports & 303 & 30 & 4 & 6 & SENSOR \\
    & UWaveGestureLibrary & 240 & 315 & 8 &3 & SENSOR \\
    \bottomrule
  \end{tabular}
  %
  }
  \label{appendix:table:datasets}
\end{table}

\section{Appendix for experiment: TS-LDDMM representation identifiability}
\label{appendix: identifiability}
In this experiment, we evaluate the ability of TS-LDDMM to retrieve the parameter $v_0^*$ that encodes the deformation $\varphi^{\{v_0^*\}}$ acting on a time series graph $\msg$ by solving the geodesic shooting problem \eqref{eq:relaxation} between $\msg$ and $\varphi^{\{v_0^*\}}.\msg$. Parameter identifiability is an important property for subsequent statistical analysis. Results show that TS-LDDMM representations are identifiable or weakly identifiable depending on the velocity field kernel $K_G$ specification.

\subsection{Settings}
\label{appendix: settings_identifiability}
This experiment only involves the TS-LDDMM method in two different settings: 
\begin{itemize}
  \item \textbf{The velocity field kernel $K_G$ is well-specified:} The velocity field kernel $K_G$ is set to $ (c_0,c_1,\sigma_{T,0},\sigma_{T,1},\sigma_x) = (1,0.1,100,1,1)$, the varifold loss kernels $(k_{pos},k_{dir})$ are set to $(\sigma_{\pos,t}, \sigma_{\pos,t}, \sigma_{\dir,t}, \sigma_{\dir,x}) = (2,1,2,0.6)$, and the optimizer has 400 steps with a maximum stepsize $\eta_M$ of 0.05.
  \item \textbf{The velocity field kernel $K_G$ is missspecified:} The velocity field kernel $K_G$ is set with  $(c_0,c_1,\sigma_{T,1}) = (1,0.1,1)$, $\sigma_{T,0}$ ranging in $(1,5,10,50,100,200,300)$, and $\sigma_x$  ranging in $(0.1,1,10,100)$. The varifold loss kernels $(k_{pos},k_{dir})$ are set to $(\sigma_{\pos,t}, \sigma_{\pos,t}, \sigma_{\dir,t}, \sigma_{\dir,x}) = (2,1,2,0.6)$, and the optimizer has 400 steps with a maximum stepsize $\eta_M$ of 0.05.
\end{itemize}

\begin{figure}[t]
  \centering
  \includegraphics[width=0.5\linewidth]{pictures/samples.pdf}
  \caption{Plots of $\varphi^{\{v_0(\mathbf{\alpha}^*,\msx)\}}.\msx$ for different values of $\mathbf{\alpha}^*$ according to its sampling parameter $t_a,s_a,m_s $, taking $\msx=\msg(s_0)$ with $s_0:k\in [300]\to \sin(2\pi k/300) $.}
  \label{fig:exemple_synthetic}
\end{figure}

\begin{table}
  \caption{Values of $\scrl(\varphi^{\{v_0(\mathbf{\alpha}^*,\msx)\}}.\msx,\varphi^{\{\hat{v}_0\}}.\msx)$ as $\mathbf{\alpha}^*$ is sampled according to Gen(10,10,50) and $\hat{v}_0$ is estimated using $K_\msg$ with varying parameters $\sigma_{T,1},\sigma_x$.}
    \centering
       \begin{tabular}{lrrrrrrr}
       \toprule
       $\sigma_{T,0} \backslash \sigma_x$  & 1 & 10 & 50 & 100 & 200 & 300 \\
       \midrule
       0.1 & 2e+0 & 3e-4  & 1e-5&4e-6&7e-4&4e-3 \\
      1 & 4e-2 & 1e-4  & 1e-5&4e-6&7e-4 &4e-3  \\
       100 & 4e-2 & 2e-4  & 1e-5&4e-6&7e-4&4e-3  \\
       \bottomrule
       \end{tabular}
    \label{table:synthetic2}
\end{table}

provided that the hyperparameters and the reference graph are wisely selected, i.e., the parameter $v_0^*$ generating a deformation $\varphi^{\{v_0^*\}}$ of a time series graph $\msg$ can be estimated from the data $\msg,\varphi^{\{v_0^*\}}.\msg$ by solving the geodesic shooting problem \eqref{eq:relaxation}. 

\paragraph{The velocity field kernel $K_G$ is well specified.}

First, we show the model identifiability when the kernel $K_G$ is well specified: the estimated parameter is a good approximation of the generating parameter when the generation and the estimation procedure use the same hyperparameters for the RKHS kernel $K_\msg$.
All the hyperparameter values for generation and estimation are given in \Cref{appendix: settings_identifiability}.

We fix the initial control points as $\msx=\left(x_k=(k,\sin(2\pi k/300))\right)_{k\in[300]} $.
Given $m_{s}\in \Nset_{>0}$ and $t_{a},s_{a}>0$, we randomly generate initial momentums $\mathbf{\alpha}^*=(\alpha_k^*)_{k\in[\mathbf{n}_0]}$ with the following sampling, called Gen($m_s,t_a,s_a$):
For any $k\in[\mathbf{n}_0]$, $\mathbf{\alpha}_k'$ is sampled according to a Gaussian normal distribution $\mathcal{N}(0_{d+1},I_{d+1})$.
Then, $(\alpha_k')_{k\in[\mathbf{n}_0]}$ is regularized by a rolling average of size $m_{s}$, we get $\bar{\mathbf{\alpha}}'=(\bar{\alpha}_k')_{k\in[\mathbf{n}_0]}$.
Finally, we normalize $\bar{\mathbf{\alpha}}'$ to derive $\mathbf{\alpha}^*$ such that $|([\alpha_k^*]_t)_{k\in[\mathbf{n}_0]}|=t_{\text{amp}}$ and $|([\alpha_k^*]_s)_{k\in[\mathbf{n}_0]}|=s_{\text{amp}}$ for any $k\in[\mathbf{n}_0]$, denoting by $[\alpha_k^*]_t,[\alpha_k^*]_s$ the time and space coordinates of $\alpha_k^*$ respectively.
Note that the regularizing step $(\mathbf{\alpha}_k')_{k\in[\mathbf{n}_0]}\to \bar{\mathbf{\alpha}}' $ is necessary to obtain realistic deformations which take into account the regularity induced by the RKHS $\msv$.

Then, using $v_0(\mathbf{\alpha}^*,\msx)$ as defined in \eqref{eq:def_v0} with initial momentums $\mathbf{\alpha}^*$ and control points $\msx$, we apply the induced deformation $\varphi^{\{v_0\}} $ by \eqref{eq:integration} to $\msx$ and obtain $\varphi^{\{v_0\}}.\msx$.
Finally, we solve \eqref{eq:relaxation} to recover an estimation $\hat{\mathbf{\alpha}}$ of $\mathbf{\alpha}^*$ and report the average relative error (ARE) $|v_0(\hat{\mathbf{\alpha}},\msx)-v_0(\mathbf{\alpha}^*,\msx)|_\msv/|v_0(\mathbf{\alpha}^*,\msx)|_\msv$ on 50 repetitions.
This procedure is performed for any $m_{s},t_{a},s_{a}\in \{10,50,100\}\times \{5,10,15,20\}^2 $.
Mean, standard deviation, and maximum of the ARE on all these hyperparameters choices are respectively $\mathbf{0.10, 0.03, 0.17}$.
Therefore, the estimation procedure \eqref{eq:relaxation} offers a good approximation of the true parameter when the kernel $K_\msg$ is well specified.
We observe that the estimation is difficult when $t_a\ll s_a$ because the time series can be very noisy as illustrated in \Cref{fig:exemple_synthetic}: this impacts the Varifold loss which is sensitive to tangents.

\paragraph{The velocity field kernel $K_G$ is misspecified.}

We demonstrate a weak identifiability when the kernel $K_\msg$ is misspecified: we can reconstruct the graph time series' after deformations even if the hyperparameters of $K_\msg$
are different during the generation and the estimation.
 The hyperparameters of $K_\msg$ during generation are $(c_0,c_1,\sigma_{T,0},\sigma_{T,1},\sigma_x)=(1,0.1,100,1,1)$ and we fix $\sigma_{T,1},c_0,c_1=(1,1,0.1) $ for $K_\msg$ during estimation.
 We aim to understand the impact of $\sigma_{T,1},\sigma_x$ on the reconstruction since they are encoding the smoothness of the transformation according to time and space.  

For any choice of the hyperparameters $\sigma_{T,1},\sigma_x\in \{1,10,50,100,200,300 \}\times \{0.1,1,100\}$ related to $K_\msg$ in the estimation,
we average $\scrl(\varphi^{\{v_0(\mathbf{\alpha}^*,\msx)\}}.\msx,\varphi^{\{\hat{v}_0\}}.\msx)$ on 50 repetitions when $\mathbf{\alpha}^*$ is sampled according to Gen$(10,10,50)$ and $\hat{v}_0=v_0(\hat{\alpha},\msx)$ denoting by $\hat{\alpha}$ the result of the minimization \eqref{eq:relaxation}.
We observe in \Cref{table:synthetic2} that the reconstruction is almost perfect except in the case when $\sigma_{t,0}=1$ during estimation, while $ \sigma_{t,0}=100$ during generation.
Compared to $\sigma_{T,0}$, $\sigma_x$ has nearly no impact on the reconstruction.
In \Cref{appendix:kernel_implementation}-\ref{appendix:kernel_TS_LDDMM}, we propose guidelines to drive future hyperparameters tuning and further discussions related to $\sigma_{T,1},c_0,c_1$. 

\section{Appendix for experiment: Robustness to irregular sampling}
\label{appendix: robustness}

This experiment is inspired by \cite{oh2024stable} where the authors perform an extensive comparison of Neural Ordinary Differential Equations (Neural ODEs) methods~\cite{kidger2020neural}. We assess the classification
performances of several methods under regular sampling (0\% missing rate) and three irregular sampling regimes on 15 shape-based datasets (7 univariate \& 8 multivariate). Methods and training strategy are taken from its associated Github\footnote{\url{https://github.com/yongkyung-oh/Stable-Neural-SDEs}} and described in what follows. We conclude with the results, which show that our method, TS-LDDMM, outperforms all methods for sampling regimes with missing rates: 0\%, 30\%, and 50\%.

\subsection{Benchmark methods}
\label{appendix: benchmark robustness}
In related work, we give an overview of Neurals ODEs methods and their relation with TS-LDDMM.  
\begin{itemize}
  \item RNN-based methods: Baseline reccurent neural networks including RNN~\cite{medsker1999recurrent}, LSTM~\cite{hochreiter1997long}, and GRU~\cite{chung2014empirical}.

  \item Attention-based methods: Multi-Time Attention Networks (MTAN)~\cite{shukla2021multi} and Multi-Integration Attention Module (MIAM)~\cite{lee2022multi}. Both handle multivariate time series irregularly sampled with attention mechanisms.
  \item Neural ODEs: ODE-LSTM~\cite{lechner2020learning} a form of Neural-ODEs used to learn continuous latent representations. 
  \item Neural SDEs:  Neural SDE~\cite{liu2019neural} and Neural LNSDE~\cite{oh2024stable} have been proposed to model randomness in time-series using drift and diffusion terms as an extension of Neural-ODEs. 
  \item Shape-Analysis methods: TS-LDDMM (ours) and LDDMM~\cite{glaunes2008large}. From shape analysis, both methods learn representations by solving ODEs parametrized with Kernels. While both methods handle multivariate signals irregularly sampled, TS-LDDMM is specifically designed for time series.
\end{itemize}

\subsection{Model architecture}

\paragraph{Neural ODEs methods}
As depicted in~\cite{oh2024stable}, any Neural ODEs layer in \Cref{appendix: benchmark robustness} is followed by an MLP with two fully connected layers with \texttt{ReLU} activations. The risk of overfitting and the model regularization are handled with a dropout rate of 10\% and an early-stopping mechanism, ceasing the training when the validation loss does not improve for 10 successive epochs. 

For each method and dataset, the learning rate, the hidden vector dimensions, and the number of layers are optimized to minimize the CrossEntropy loss on a validation set using the \texttt{Ray}~\footnote{\url{https://github.com/ray- project/ray}} Python library. 
The learning rate varies from $10^{-4}$ to $10^{-1}$ using log uniform search, the hidden vector dimension ranges from ${16, 32, 64, 128}$ using grid search, and the number of layers ranges from ${1, 2, 3, 4}$ using grid search. The batch size was selected from ${16, 32, 64, 128}$ according to the size of the dataset. All methods were trained for 100 epochs, and the best method was selected based on the lowest validation loss. 

\paragraph{TS-LDDMM and LDDMM}
Representations learned with TS-LDDMM or LDDMM are fed to a Support Vector Classifier (\href{https://scikit-learn.org/stable/modules/generated/sklearn.svm.SVC.html#sklearn.svm.SVC}{SVC}) from \texttt{scikit-learn}~\footnote{\url{https://scikit-learn.org/stable/}}. All SVC's hyperparameters are set to default except the regularization term C, which is set through grid search on a validation set with the macro f1-score~\footnote{\url{https://scikit-learn.org/stable/modules/generated/sklearn.metrics.f1_score.html}}. 

To learn TS-LDDMM (resp. LDDMM) representations, the velocity field kernel $K_G$ is set to $ (c_0,c_1,\sigma_{T,0},\sigma_{T,1},\sigma_x) = (1,0.1,0.33\bar{l},1,n_d)$, (resp. $ (\sigma_{T},\sigma_x) = (0.33\bar{l},n_d)$) where $\bar{l}$ is the average time series length and $n_d$ the number of dimensions. For both methods and all datasets, the varifold loss kernels $(k_{pos},k_{dir})$ are identical and set to $(\sigma_{\pos,t}, \sigma_{\pos,t}, \sigma_{\dir,t}, \sigma_{\dir,x}) = (2,n_d,2,n_d)$. For TS-LDDMM (resp. LDDMM), the optimizer is set with 400 epochs (resp. 400) and a maximum learning rate $\eta_M = 0.1$ (resp. $\eta_M = 0.01$). In all cases, the initial reference graph is selected in the dataset as a time series with the median length.


\subsection{Protocol}

In this experiment, we investigate the robustness to missing samples and the classification performance of TS-LDDMM compared to Neural ODEs on 15 datasets described in \Cref{appendix:classification_dataset}. For fairness between methods of different architectures, the evaluation protocol on each dataset and method is as follows: 
\begin{enumerate}
  \item Spilt the dataset in train 75\%, validation 15\%, and test 15\%.
  \item Tune hyperparameters with train and validation sets and a missing rate of 0\%.
  \item For each missing rate in [0\%,30\%,50\%,70\%]
    \begin{itemize}
      \item Remove samples in time series in the train and test sets according to the missing rate and the drop procedure described in~\cite{kidger2020neural}.
      \item Train the model on the train set
      \item Evaluate the macro f1-score on the test set
    \end{itemize}
\end{enumerate}




\subsection{Results}
In this experiment, we investigate the robstuness to missing samples and the classification performance of TS-LDDMM representations. We compare TS-LDDMM with LDDMM and 8 neural ODEs networks. Performances are evaluated in terms of average macro f1-score and rank on four different regimes of missing rate 0\%,30\%,50\%, and 70\%. Results are aggregated in \Cref{table:robstuness}. 

On three out of four regimes (0\%,30\%, and 50\%) TS-LDDMM classifier is the best performer in terms of f1-score and rank. For missing rates of 0\% and 30\%, the score increases by 10\% compared to the second-best performer, LDDMM. However, LDDMM is not the second-best performer in rank (Neural LNSDE), showing its sensitivity to parameterization, unlike TS-LDDMM, which remains consistent. Performances of Neural LNSDE remain constant with the increase of the missing rate as observed in~\cite{oh2024stable}, and it becomes the best performer for missing rate 70\%. The decrease in TS-LDDMM performances with the increasing missing rate is due to the varifold loss, which poorly approximates the time series shape. Other losses might be more relevant for high missing rates.

Overall, TS-LDDMM is a relevant and consistent shape-based representation for irregularly sampled multivariate time series for missing rates up to 50\% . 

\begin{table}[hbt!]
  \centering
  \caption{Comparison of average macro f1-score and rank as the sample dropping rate increases. \textbf{First} \& \underline{second} best performers. TS-LDDMM is the best performer on three out of four regimes.}
  \label{table:robstuness}
  \resizebox{\columnwidth}{!}{%
  \begin{tabular}{lcccccccc}
    \toprule
    \multirow[c]{2}{*}{\textbf{Methods}} & \multicolumn{2}{c}{\textbf{Regular}} & \multicolumn{2}{c}{\textbf{30 \% dropped}} &  \multicolumn{2}{c}{\textbf{50 \% dropped}} & \multicolumn{2}{c}{\textbf{70 \% dropped}} \\
    \cline{2-9}
     &  \textbf{F1-score} & \textbf{Rank} &  \textbf{F1-score} & \textbf{Rank} &  \textbf{F1-score} & \textbf{Rank} &  \textbf{F1-score} & \textbf{Rank} \\
    \midrule
    RNN (1999) & $0.64 \pm 0.21$  & 6.2 & $0.53 \pm 0.23$  & 6.6 & $0.48 \pm 0.21$  & 7.2 & $0.44 \pm 0.21$  & 6.07 \\
    LSTM (1997) & $0.61 \pm 0.29$  & 6.0 & $0.57 \pm 0.29$  & 6.27 & $0.53 \pm 0.25$  & 6.07 & $0.51 \pm 0.29$  & 5.27 \\
    GRU (2014) & $0.71 \pm 0.26$  & 4.2 & $0.68 \pm 0.28$  & 4.27 & $0.66 \pm 0.28$  & 3.73 & $\underline{0.59 \pm 0.28}$  & \underline{3.67} \\
    MTAN (2021) & $0.59 \pm 0.28$  & 7.13 & $0.58 \pm 0.28$  & 5.8 & $0.54 \pm 0.29$  & 5.33 & $0.51 \pm 0.28$  & 5.0 \\
    MIAM (2022) & $0.48 \pm 0.35$  & 6.93 & $0.42 \pm 0.33$  & 8.27 & $0.47 \pm 0.31$  & 6.93 & $0.35 \pm 0.31$  & 7.6 \\
    ODE-LSTM (2020) & $0.63 \pm 0.24$  & 6.0 & $0.57 \pm 0.25$  & 6.53 & $0.51 \pm 0.24$  & 7.27 & $0.45 \pm 0.23$  & 6.73 \\
    Neural SDE (2019) & $0.48 \pm 0.28$  & 7.67 & $0.47 \pm 0.26$  & 7.47 & $0.45 \pm 0.27$  & 7.13 & $0.45 \pm 0.25$  & 6.0 \\
    Neural LNSDE (2024) & $0.7 \pm 0.27$  & \underline{3.87} & $0.68 \pm 0.29$  & \underline{4.0} & $\underline{0.67 \pm 0.25}$  & \underline{3.53} & $\mathbf{0.66 \pm 0.23}$  & \textbf{2.47} \\
    LDDMM (2008) & $\underline{0.72 \pm 0.2}$  & 4.53 & $\underline{0.7 \pm 0.21}$  & 4.2 & $0.57 \pm 0.25$  & 5.0 & $0.4 \pm 0.25$  & 7.13 \\
    TS-LDDMM (ours) & $\mathbf{0.83 \pm 0.18}$  & \textbf{2.93} & $\mathbf{0.8 \pm 0.18}$  & \textbf{2.07} & $\mathbf{0.7 \pm 0.26}$  & \textbf{3.33} & $0.51 \pm 0.27$  & 5.67 \\
    \bottomrule
  \end{tabular}  
  %
  }
\end{table}







\section{Appendix for experiment: Classification benchmark on regularly sampled datasets}
\label{appendix: shape_classification}

In this section, we compare the classification performances of TS-LDDMM with other methods from shape analysis on 15 shape-based datasets of time series regularly sampled. TS-LDDMM outperforms other methods on 12 out of 15, highlighting its relevance for shape analysis when dealing with time series. 


\subsection{Benchmark methods}
\begin{itemize}
  \item SRV-based method: we include TCLR~\cite{heo2024logistic} a logistic regression on the tangent space of the Frechet mean with Square Root Velocity (SRV representation). We also include Shape-FPCA~\cite{wu2024shape} that encodes both the time series and its time parameterization. 
  \item LDDMM-Based : TS-LDDMM (ours) and LDDMM~\cite{glaunes2008large}. Both methods learn representations by solving ODEs parametrized with Kernels. While both methods handle multivariate signals, TS-LDDMM is specifically designed for time series.
 \end{itemize}

\subsection{Model settings}

\paragraph{TCLR \& Shape-FPCA}
Shape-FPCA is available in the Python library \texttt{FDASRSF}~\footnote{https://fdasrsf-python.readthedocs.io/en/latest/}. Once the shape-FPCA representations are learned, they are fed to an \href{https://scikit-learn.org/stable/modules/generated/sklearn.svm.SVC.html#sklearn.svm.SVC}{SVC} from \texttt{scikit-learn}. \texttt{FDASRSF} provides SRV representation methods that we combined with a \href{https://scikit-learn.org/stable/modules/generated/sklearn.linear_model.LogisticRegressionCV.html}{logistic regression} from \texttt{scikit-learn} to implement TCLR. For both methods, the number of steps to learn the Frechet mean is set to 50, and the regularization hyperparameter C is set through grid search on a validation set with the macro f1-score. Other parameters are set to default.  

\paragraph{TS-LDDMM \& LDDMM}
Representations learned with TS-LDDMM or LDDMM are fed to an SVC from \texttt{scikit-learn}. All SVC's hyperparameters are set to default except the regularization term C, which is set through grid search on a validation set with the macro f1-score. 

To learn TS-LDDMM (resp. LDDMM) representations, the velocity field kernel $K_G$ is set to $ (c_0,c_1,\sigma_{T,0},\sigma_{T,1},\sigma_x) = (1,0.1,0.33\bar{l},1,n_d)$, (resp. $ (\sigma_{T},\sigma_x) = (0.33\bar{l},n_d)$) where $\bar{l}$ is the average time series length and $n_d$ the number of dimensions. For both methods and all datasets, the varifold loss kernels $(k_{pos},k_{dir})$ are identical and set to $(\sigma_{\pos,t}, \sigma_{\pos,t}, \sigma_{\dir,t}, \sigma_{\dir,x}) = (2,n_d,2,n_d)$. For TS-LDDMM (resp. LDDMM), the optimizer is set with 400 epochs (resp. 400) and a maximum learning rate $\eta_M = 0.1$ (resp. $\eta_M = 0.01$). In all cases, the initial reference graph is selected in the dataset as a time series with the median length.


\subsection{Protocol}
For each dataset and method, the evaluation protocol is a simple train,validation test with hyperparameter tuning:
\begin{enumerate}
  \item Split The dataset in train 75\%, validation 15\%, and test 15\%.
  \item Training and hyperparameters tuning with train and validation sets
  \item Evaluate the macro f1-score on the test set 
\end{enumerate}


\subsection{Results}
In this experiment, we investigate the classification performances of several methods from shape analysis on 15 shape-based time series datasets (7 univariate and 8 multivariate). The performances are evaluated in terms of macro f1-score. Results are aggregated in \Cref{table: shape classif}. 

The TS-LDDMM-based classifier outperforms other methods on 12 out of 15 datasets. TCLR is the second-best performer on univariate datasets; however, its current implementation with \texttt{FDASRSF} does not extend to the multivariate case, which limits usage. LDDMM performances are lower than TCLR, and Shape-FPCA is the worst performer. 

Overall, TS-LDDMM representations are well suited for shape-based time series classification, and its extension to multivariate irregularly sampled time series makes it a relevant option for time series shape analysis.

\begin{table}[hbt!]
  \centering
  \caption{F1-score comparison between methods from shape analysis on 15 datasets. \textbf{First} and \underline{second} best performers.}
  \label{table: shape classif}
  \resizebox{\columnwidth}{!}{%
  \begin{tabular}{llrrrr}
    \toprule
     & \textbf{Dataset} & \textbf{Shape-FPCA (2024)} & \textbf{TCLR (2024)} & \textbf{LDDMM (2008)} & \textbf{TS-LDDMM (ours)} \\
    \midrule
    \multirow[c]{7}{*}{Univariate} & ArrowHead & 0.18 & 0.75 & \underline{0.84} & \textbf{0.91} \\
     & BME & 0.16 & \underline{1.00} & 0.82 & \textbf{1.00} \\
     & ECG200 & 0.40 & 0.67 & \textbf{0.81} & \underline{0.79} \\
     & FacesUCR & 0.08 & \underline{0.73} & 0.69 & \textbf{0.86} \\
     & GunPoint & 0.93 & \underline{0.97} & 0.83 & \textbf{1.00} \\
     & PhalangesOutlinesCorrect & 0.39 & \textbf{0.63} & \underline{0.53} & 0.52 \\
     & Trace & 0.55 & \underline{1.00} & 0.46 & \textbf{1.00} \\
    \cline{1-6}
    \multirow[c]{8}{*}{Multivariate} & ArticularyWordRecognition & -- & -- & \underline{0.98} & \textbf{1.00} \\
     & Cricket & -- & -- & \underline{0.77} & \textbf{0.93} \\
     & ERing & -- & -- & \underline{0.95} & \textbf{0.98} \\
     & Handwriting & -- & -- & \underline{0.22} & \textbf{0.44} \\
     & Libras & -- & -- & \underline{0.56} & \textbf{0.60} \\
     & NATOPS & -- & -- & \underline{0.82} & \textbf{0.82} \\
     & RacketSports & -- & -- & \textbf{0.83} & \underline{0.79} \\
     & UWaveGestureLibrary & -- & -- & \underline{0.72} & \textbf{0.81} \\
    \bottomrule
    \end{tabular}
    %
  }
    
\end{table}


\section{Appendix for the experiment: Noise sensitivity}
\label{appendix: noise sensitivity}
This experiment evaluates the influence of noise on the learning of the reference sequence for TS-LDDMM and SRVF Kacher-mean, a subroutine of Shape-FPCA~\cite{wu2024shape}. 

\subsection{Protocol}
The dataset includes 100 sine waves with randomly generated time parametrization by following the procedure described in \Cref{appendix: settings_identifiability} with $\text{Gen}(50,1,0)$ and uniformly resampled. The dataset has been altered under four scenarios with an additive Gaussian noise centered and with standard deviation $\sigma_\epsilon \in \{0,0.05,0.1,0.2\}$. The referent sequence is learned for each scenario, and the $L_2$-norm error between the exact and the learned barycenter is computed. 

\subsection{Method settings}
For SRVF, the number of steps to learn the Kacher-mean is set to 20. Regarding TS-LDDMM, the velocity field kernel $K_G$ is set to $ (c_0,c_1,\sigma_{T,0},\sigma_{T,1},\sigma_x) = (1,0.1,65,1,1)$, and the varifold loss kernels $(k_{pos},k_{dir})$ are set to $(\sigma_{\pos,t}, \sigma_{\pos,t}, \sigma_{\dir,t}, \sigma_{\dir,x}) = (5,1,1,1)$. For TS-LDDMM, the optimizer is set with 400 epochs and a maximum learning rate $\eta_M = 0.1$. In all cases, the initial reference graph is selected in the dataset as a time series with the median length.

\begin{figure}
  \centering
  \includegraphics[width = 0.9\linewidth]{pictures/noise sensitivity/noise sensitivity.pdf}
  \caption{Illustration of the learned barycenter (red) compared to the exact barycenter (green) for both TS-LDDMM \textbf{(a)} and Shape-FPCA \textbf{(b)}. The computation has been done for different level of noise $\epsilon \sim \mathcal{N}(0,\sigma_\epsilon)$ with $\sigma_\epsilon \in \{0,0.05,0.1,0.2\}$.}
  \label{fig: noise sensitivity}
\end{figure}

\subsection{Results}
\Cref{fig: noise sensitivity} illustrates the results. Noise level affects the learning reference graph in both cases, as depicted by the increasing error and the illustrations. However, the overall sine wave shape is better preserved by TS-LDDMM compared to SRVF Kacher-mean, for which the sine wave amplitude decreases as the noise increases. In addition, for TS-LDDMM, the regularity of the reference graph can be controlled by penalizing the norm of the velocity fields in the loss function. Further work on penalization will be conducted to handle noisy data better.


\section{Appendix for experiment: Analysis of respiratory behavior in mice}
\label{appendix: mice_exp_setting}

\subsection{Settings}

This experiment involves TS-LDDMM, LDDMM~\cite{glaunes2008large} and Shape-FPCA~\cite{wu2024shape} methods. Two scenarios are investigated: before drug exposure and before/after drug exposure. All methods are investigated on both scenarios. 

\paragraph{TS-LDDMM parameters.}
\begin{itemize}
  \item \textbf{Before exposure:} The velocity field kernel $K_G$ is set to $ (c_0,c_1,\sigma_{T,0},\sigma_{T,1},\sigma_x) = (1,0.1,150,1,2)$. The varifold loss is the sum of three varifolds to capture shapes variations at different scales with parameters: (Varifold 1,Varifold 2,Varifold 3): $\left((5,2,5,1),(2,1,2,0.6),(1,0.6,1,0.6)\right)$ and the mapper $(\sigma_{\pos,t}, \sigma_{\pos,t}, \sigma_{\dir,t}, \sigma_{\dir,x})$. The optimizer has 800 steps with a maximum stepsize $\eta_M$ of 0.3.
  \item \textbf{Before/after exposure:} The velocity field kernel $K_G$ is set to $ (c_0,c_1,\sigma_{T,0},\sigma_{T,1},\sigma_x) = (1,0.1,220,1,2)$. The varifold loss is the sum of four varifolds to capture shapes variations at different scales with parameters: (Varifold 1,Varifold 2,Varifold 3, Varifold 4): $\left((30,2,30,1),(5,2,5,1),(2,1,2,0.6),(1,0.1,1,0.1)\right)$ and the mapper $(\sigma_{\pos,t}, \sigma_{\pos,t}, \sigma_{\dir,t}, \sigma_{\dir,x})$. The optimizer has 800 steps with a maximum stepsize $\eta_M$ of 0.3.
\end{itemize}

\paragraph{LDDMM parameters.} Note that varifold losses are unchanged between TS-LDDMM and LDDMM. Compared to TS-LDDMM, the convergence of LDDMM is more sensitive to the maximum stepsize $\eta_m$, which must remain small for LDDMM to guarantee the convergence.
\begin{itemize}
  \item \textbf{Before exposure:} The velocity field kernel $K_G$ is an anysotropic Gaussian kernel with parameters $\sigma_{T} =150$ for the time dimension and $\sigma_x = 2$ for space dimensions. 
  The varifold loss is the sum of three varifolds to capture shapes variations at different scales with parameters: (Varifold 1,Varifold 2,Varifold 3): $\left((5,2,5,1),(2,1,2,0.6),(1,0.6,1,0.6)\right)$ and the mapper $(\sigma_{\pos,t}, \sigma_{\pos,t}, \sigma_{\dir,t}, \sigma_{\dir,x})$. The optimizer has 800 steps with a maximum stepsize $\eta_M$ of 0.01.
  \item \textbf{Before/after exposure:} The velocity field kernel $K_G$ is an anysotropic Gaussian kernel with parameters $\sigma_{T} =220$ for the time dimension and $\sigma_x = 2$ for space dimensions.
  The varifold loss is the sum of four varifolds to capture shapes variations at different scales with parameters: (Varifold 1,Varifold 2,Varifold 3, Varifold 4): $\left((30,2,30,1),(5,2,5,1),(2,1,2,0.6),(1,0.1,1,0.1)\right)$ and the mapper $(\sigma_{\pos,t}, \sigma_{\pos,t}, \sigma_{\dir,t}, \sigma_{\dir,x})$. The optimizer has 800 steps with a maximum stepsize $\eta_M$ of 0.01.
\end{itemize}

\paragraph{Shape-FPCA parameters.}
For both scenarios, respiratory cycles are linearly interpolated and resampled to 200 points, and the length of the original time interval is kept. The computation of the Kacher-mean is done in a maximum of 50 iterations, and srv representations of the realigned time series and time parametrization are concatenated with cycle durations. When concatenating these vectors, the choice of amplitude factors is made to minimize the reconstruction error from the principal components analysis by following the procedure described in \cite{wu2024shape}. Shape-FPCA does not handle multivariate data, and we only kept the nasal airflow for this method. 

\subsection{Addiotinal results}
\Cref{fig: after tslddmm} presents results for TS-LDDMM and \Cref{fig: after shapefpca}  presents results for Shape-FPCA.
The main components look similar. However, a subtle difference, yet important, can be noticed. With Shape-FPCA, the deformation tends to be a uniform time scaling, whereas, with TS-LDDMM, the time dilatation mainly occurs during the pause between inspiration and expiration. Qualitatively, this last deformation fits the physiological phenomenon: Mice's muscles cannot relax after exposure to the irritant molecule, leading to pauses between inspiration and expiration~\cite{nervo2019respiratory}. Qualitatively, contrary to Shape-FPCA, which manages to represent the main phenomena in the data, the deformations of TS-LDDMM capture subtle physiological behaviors essential for understanding the phenomenon at hand.

\begin{figure}
  \centering
  \includegraphics[width = \linewidth]{pictures/exp after/exp after.pdf}
  \caption{Analysis of the first Principal Component (PC1) related to mice ventilation before and after exposure with TS-LDDMM representations. \textbf{(a)} displays PC densities per mice genotype, \textbf{(b)} illustrates deformations of the reference respiratory cycle $\mathbf{c}_0$ along PC1, and \textbf{(c)} displays all respiratory cycles with respect to time in PC1 and PC3 coordinates}
  \label{fig: after tslddmm}
\end{figure}

\begin{figure}
  \centering
  \includegraphics[width = \linewidth]{pictures/exp after/exp after fpca.pdf}
  \caption{Analysis of the first Principal Component (PC1) related to mice ventilation before and after exposure with shape-FPCA representations. \textbf{(a)} displays PC densities per mice genotype, \textbf{(b)} illustrates deformations of the reference respiratory cycle $\mathbf{c}_0$ along PC1, and \textbf{(c)} displays all respiratory cycles with respect to time in PC1 and PC2 coordinates}
  \label{fig: after shapefpca}
\end{figure}


\input{checklist.tex}


\end{document}